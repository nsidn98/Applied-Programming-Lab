
% Default to the notebook output style

    


% Inherit from the specified cell style.




    
\documentclass[11pt]{article}

    
    
    \usepackage[T1]{fontenc}
    % Nicer default font (+ math font) than Computer Modern for most use cases
    \usepackage{mathpazo}

    % Basic figure setup, for now with no caption control since it's done
    % automatically by Pandoc (which extracts ![](path) syntax from Markdown).
    \usepackage{graphicx}
    % We will generate all images so they have a width \maxwidth. This means
    % that they will get their normal width if they fit onto the page, but
    % are scaled down if they would overflow the margins.
    \makeatletter
    \def\maxwidth{\ifdim\Gin@nat@width>\linewidth\linewidth
    \else\Gin@nat@width\fi}
    \makeatother
    \let\Oldincludegraphics\includegraphics
    % Set max figure width to be 80% of text width, for now hardcoded.
    \renewcommand{\includegraphics}[1]{\Oldincludegraphics[width=.8\maxwidth]{#1}}
    % Ensure that by default, figures have no caption (until we provide a
    % proper Figure object with a Caption API and a way to capture that
    % in the conversion process - todo).
    \usepackage{caption}
    \DeclareCaptionLabelFormat{nolabel}{}
    \captionsetup{labelformat=nolabel}

    \usepackage{adjustbox} % Used to constrain images to a maximum size 
    \usepackage{xcolor} % Allow colors to be defined
    \usepackage{enumerate} % Needed for markdown enumerations to work
    \usepackage{geometry} % Used to adjust the document margins
    \usepackage{amsmath} % Equations
    \usepackage{amssymb} % Equations
    \usepackage{textcomp} % defines textquotesingle
    % Hack from http://tex.stackexchange.com/a/47451/13684:
    \AtBeginDocument{%
        \def\PYZsq{\textquotesingle}% Upright quotes in Pygmentized code
    }
    \usepackage{upquote} % Upright quotes for verbatim code
    \usepackage{eurosym} % defines \euro
    \usepackage[mathletters]{ucs} % Extended unicode (utf-8) support
    \usepackage[utf8x]{inputenc} % Allow utf-8 characters in the tex document
    \usepackage{fancyvrb} % verbatim replacement that allows latex
    \usepackage{grffile} % extends the file name processing of package graphics 
                         % to support a larger range 
    % The hyperref package gives us a pdf with properly built
    % internal navigation ('pdf bookmarks' for the table of contents,
    % internal cross-reference links, web links for URLs, etc.)
    \usepackage{hyperref}
    \usepackage{longtable} % longtable support required by pandoc >1.10
    \usepackage{booktabs}  % table support for pandoc > 1.12.2
    \usepackage[inline]{enumitem} % IRkernel/repr support (it uses the enumerate* environment)
    \usepackage[normalem]{ulem} % ulem is needed to support strikethroughs (\sout)
                                % normalem makes italics be italics, not underlines
    

    
    
    % Colors for the hyperref package
    \definecolor{urlcolor}{rgb}{0,.145,.698}
    \definecolor{linkcolor}{rgb}{.71,0.21,0.01}
    \definecolor{citecolor}{rgb}{.12,.54,.11}

    % ANSI colors
    \definecolor{ansi-black}{HTML}{3E424D}
    \definecolor{ansi-black-intense}{HTML}{282C36}
    \definecolor{ansi-red}{HTML}{E75C58}
    \definecolor{ansi-red-intense}{HTML}{B22B31}
    \definecolor{ansi-green}{HTML}{00A250}
    \definecolor{ansi-green-intense}{HTML}{007427}
    \definecolor{ansi-yellow}{HTML}{DDB62B}
    \definecolor{ansi-yellow-intense}{HTML}{B27D12}
    \definecolor{ansi-blue}{HTML}{208FFB}
    \definecolor{ansi-blue-intense}{HTML}{0065CA}
    \definecolor{ansi-magenta}{HTML}{D160C4}
    \definecolor{ansi-magenta-intense}{HTML}{A03196}
    \definecolor{ansi-cyan}{HTML}{60C6C8}
    \definecolor{ansi-cyan-intense}{HTML}{258F8F}
    \definecolor{ansi-white}{HTML}{C5C1B4}
    \definecolor{ansi-white-intense}{HTML}{A1A6B2}

    % commands and environments needed by pandoc snippets
    % extracted from the output of `pandoc -s`
    \providecommand{\tightlist}{%
      \setlength{\itemsep}{0pt}\setlength{\parskip}{0pt}}
    \DefineVerbatimEnvironment{Highlighting}{Verbatim}{commandchars=\\\{\}}
    % Add ',fontsize=\small' for more characters per line
    \newenvironment{Shaded}{}{}
    \newcommand{\KeywordTok}[1]{\textcolor[rgb]{0.00,0.44,0.13}{\textbf{{#1}}}}
    \newcommand{\DataTypeTok}[1]{\textcolor[rgb]{0.56,0.13,0.00}{{#1}}}
    \newcommand{\DecValTok}[1]{\textcolor[rgb]{0.25,0.63,0.44}{{#1}}}
    \newcommand{\BaseNTok}[1]{\textcolor[rgb]{0.25,0.63,0.44}{{#1}}}
    \newcommand{\FloatTok}[1]{\textcolor[rgb]{0.25,0.63,0.44}{{#1}}}
    \newcommand{\CharTok}[1]{\textcolor[rgb]{0.25,0.44,0.63}{{#1}}}
    \newcommand{\StringTok}[1]{\textcolor[rgb]{0.25,0.44,0.63}{{#1}}}
    \newcommand{\CommentTok}[1]{\textcolor[rgb]{0.38,0.63,0.69}{\textit{{#1}}}}
    \newcommand{\OtherTok}[1]{\textcolor[rgb]{0.00,0.44,0.13}{{#1}}}
    \newcommand{\AlertTok}[1]{\textcolor[rgb]{1.00,0.00,0.00}{\textbf{{#1}}}}
    \newcommand{\FunctionTok}[1]{\textcolor[rgb]{0.02,0.16,0.49}{{#1}}}
    \newcommand{\RegionMarkerTok}[1]{{#1}}
    \newcommand{\ErrorTok}[1]{\textcolor[rgb]{1.00,0.00,0.00}{\textbf{{#1}}}}
    \newcommand{\NormalTok}[1]{{#1}}
    
    % Additional commands for more recent versions of Pandoc
    \newcommand{\ConstantTok}[1]{\textcolor[rgb]{0.53,0.00,0.00}{{#1}}}
    \newcommand{\SpecialCharTok}[1]{\textcolor[rgb]{0.25,0.44,0.63}{{#1}}}
    \newcommand{\VerbatimStringTok}[1]{\textcolor[rgb]{0.25,0.44,0.63}{{#1}}}
    \newcommand{\SpecialStringTok}[1]{\textcolor[rgb]{0.73,0.40,0.53}{{#1}}}
    \newcommand{\ImportTok}[1]{{#1}}
    \newcommand{\DocumentationTok}[1]{\textcolor[rgb]{0.73,0.13,0.13}{\textit{{#1}}}}
    \newcommand{\AnnotationTok}[1]{\textcolor[rgb]{0.38,0.63,0.69}{\textbf{\textit{{#1}}}}}
    \newcommand{\CommentVarTok}[1]{\textcolor[rgb]{0.38,0.63,0.69}{\textbf{\textit{{#1}}}}}
    \newcommand{\VariableTok}[1]{\textcolor[rgb]{0.10,0.09,0.49}{{#1}}}
    \newcommand{\ControlFlowTok}[1]{\textcolor[rgb]{0.00,0.44,0.13}{\textbf{{#1}}}}
    \newcommand{\OperatorTok}[1]{\textcolor[rgb]{0.40,0.40,0.40}{{#1}}}
    \newcommand{\BuiltInTok}[1]{{#1}}
    \newcommand{\ExtensionTok}[1]{{#1}}
    \newcommand{\PreprocessorTok}[1]{\textcolor[rgb]{0.74,0.48,0.00}{{#1}}}
    \newcommand{\AttributeTok}[1]{\textcolor[rgb]{0.49,0.56,0.16}{{#1}}}
    \newcommand{\InformationTok}[1]{\textcolor[rgb]{0.38,0.63,0.69}{\textbf{\textit{{#1}}}}}
    \newcommand{\WarningTok}[1]{\textcolor[rgb]{0.38,0.63,0.69}{\textbf{\textit{{#1}}}}}
    
    
    % Define a nice break command that doesn't care if a line doesn't already
    % exist.
    \def\br{\hspace*{\fill} \\* }
    % Math Jax compatability definitions
    \def\gt{>}
    \def\lt{<}
    % Document parameters
    \title{Assignment 6}
    \author{Siddharth Nayak EE16B073}
    
    

    % Pygments definitions
    
\makeatletter
\def\PY@reset{\let\PY@it=\relax \let\PY@bf=\relax%
    \let\PY@ul=\relax \let\PY@tc=\relax%
    \let\PY@bc=\relax \let\PY@ff=\relax}
\def\PY@tok#1{\csname PY@tok@#1\endcsname}
\def\PY@toks#1+{\ifx\relax#1\empty\else%
    \PY@tok{#1}\expandafter\PY@toks\fi}
\def\PY@do#1{\PY@bc{\PY@tc{\PY@ul{%
    \PY@it{\PY@bf{\PY@ff{#1}}}}}}}
\def\PY#1#2{\PY@reset\PY@toks#1+\relax+\PY@do{#2}}

\expandafter\def\csname PY@tok@w\endcsname{\def\PY@tc##1{\textcolor[rgb]{0.73,0.73,0.73}{##1}}}
\expandafter\def\csname PY@tok@c\endcsname{\let\PY@it=\textit\def\PY@tc##1{\textcolor[rgb]{0.25,0.50,0.50}{##1}}}
\expandafter\def\csname PY@tok@cp\endcsname{\def\PY@tc##1{\textcolor[rgb]{0.74,0.48,0.00}{##1}}}
\expandafter\def\csname PY@tok@k\endcsname{\let\PY@bf=\textbf\def\PY@tc##1{\textcolor[rgb]{0.00,0.50,0.00}{##1}}}
\expandafter\def\csname PY@tok@kp\endcsname{\def\PY@tc##1{\textcolor[rgb]{0.00,0.50,0.00}{##1}}}
\expandafter\def\csname PY@tok@kt\endcsname{\def\PY@tc##1{\textcolor[rgb]{0.69,0.00,0.25}{##1}}}
\expandafter\def\csname PY@tok@o\endcsname{\def\PY@tc##1{\textcolor[rgb]{0.40,0.40,0.40}{##1}}}
\expandafter\def\csname PY@tok@ow\endcsname{\let\PY@bf=\textbf\def\PY@tc##1{\textcolor[rgb]{0.67,0.13,1.00}{##1}}}
\expandafter\def\csname PY@tok@nb\endcsname{\def\PY@tc##1{\textcolor[rgb]{0.00,0.50,0.00}{##1}}}
\expandafter\def\csname PY@tok@nf\endcsname{\def\PY@tc##1{\textcolor[rgb]{0.00,0.00,1.00}{##1}}}
\expandafter\def\csname PY@tok@nc\endcsname{\let\PY@bf=\textbf\def\PY@tc##1{\textcolor[rgb]{0.00,0.00,1.00}{##1}}}
\expandafter\def\csname PY@tok@nn\endcsname{\let\PY@bf=\textbf\def\PY@tc##1{\textcolor[rgb]{0.00,0.00,1.00}{##1}}}
\expandafter\def\csname PY@tok@ne\endcsname{\let\PY@bf=\textbf\def\PY@tc##1{\textcolor[rgb]{0.82,0.25,0.23}{##1}}}
\expandafter\def\csname PY@tok@nv\endcsname{\def\PY@tc##1{\textcolor[rgb]{0.10,0.09,0.49}{##1}}}
\expandafter\def\csname PY@tok@no\endcsname{\def\PY@tc##1{\textcolor[rgb]{0.53,0.00,0.00}{##1}}}
\expandafter\def\csname PY@tok@nl\endcsname{\def\PY@tc##1{\textcolor[rgb]{0.63,0.63,0.00}{##1}}}
\expandafter\def\csname PY@tok@ni\endcsname{\let\PY@bf=\textbf\def\PY@tc##1{\textcolor[rgb]{0.60,0.60,0.60}{##1}}}
\expandafter\def\csname PY@tok@na\endcsname{\def\PY@tc##1{\textcolor[rgb]{0.49,0.56,0.16}{##1}}}
\expandafter\def\csname PY@tok@nt\endcsname{\let\PY@bf=\textbf\def\PY@tc##1{\textcolor[rgb]{0.00,0.50,0.00}{##1}}}
\expandafter\def\csname PY@tok@nd\endcsname{\def\PY@tc##1{\textcolor[rgb]{0.67,0.13,1.00}{##1}}}
\expandafter\def\csname PY@tok@s\endcsname{\def\PY@tc##1{\textcolor[rgb]{0.73,0.13,0.13}{##1}}}
\expandafter\def\csname PY@tok@sd\endcsname{\let\PY@it=\textit\def\PY@tc##1{\textcolor[rgb]{0.73,0.13,0.13}{##1}}}
\expandafter\def\csname PY@tok@si\endcsname{\let\PY@bf=\textbf\def\PY@tc##1{\textcolor[rgb]{0.73,0.40,0.53}{##1}}}
\expandafter\def\csname PY@tok@se\endcsname{\let\PY@bf=\textbf\def\PY@tc##1{\textcolor[rgb]{0.73,0.40,0.13}{##1}}}
\expandafter\def\csname PY@tok@sr\endcsname{\def\PY@tc##1{\textcolor[rgb]{0.73,0.40,0.53}{##1}}}
\expandafter\def\csname PY@tok@ss\endcsname{\def\PY@tc##1{\textcolor[rgb]{0.10,0.09,0.49}{##1}}}
\expandafter\def\csname PY@tok@sx\endcsname{\def\PY@tc##1{\textcolor[rgb]{0.00,0.50,0.00}{##1}}}
\expandafter\def\csname PY@tok@m\endcsname{\def\PY@tc##1{\textcolor[rgb]{0.40,0.40,0.40}{##1}}}
\expandafter\def\csname PY@tok@gh\endcsname{\let\PY@bf=\textbf\def\PY@tc##1{\textcolor[rgb]{0.00,0.00,0.50}{##1}}}
\expandafter\def\csname PY@tok@gu\endcsname{\let\PY@bf=\textbf\def\PY@tc##1{\textcolor[rgb]{0.50,0.00,0.50}{##1}}}
\expandafter\def\csname PY@tok@gd\endcsname{\def\PY@tc##1{\textcolor[rgb]{0.63,0.00,0.00}{##1}}}
\expandafter\def\csname PY@tok@gi\endcsname{\def\PY@tc##1{\textcolor[rgb]{0.00,0.63,0.00}{##1}}}
\expandafter\def\csname PY@tok@gr\endcsname{\def\PY@tc##1{\textcolor[rgb]{1.00,0.00,0.00}{##1}}}
\expandafter\def\csname PY@tok@ge\endcsname{\let\PY@it=\textit}
\expandafter\def\csname PY@tok@gs\endcsname{\let\PY@bf=\textbf}
\expandafter\def\csname PY@tok@gp\endcsname{\let\PY@bf=\textbf\def\PY@tc##1{\textcolor[rgb]{0.00,0.00,0.50}{##1}}}
\expandafter\def\csname PY@tok@go\endcsname{\def\PY@tc##1{\textcolor[rgb]{0.53,0.53,0.53}{##1}}}
\expandafter\def\csname PY@tok@gt\endcsname{\def\PY@tc##1{\textcolor[rgb]{0.00,0.27,0.87}{##1}}}
\expandafter\def\csname PY@tok@err\endcsname{\def\PY@bc##1{\setlength{\fboxsep}{0pt}\fcolorbox[rgb]{1.00,0.00,0.00}{1,1,1}{\strut ##1}}}
\expandafter\def\csname PY@tok@kc\endcsname{\let\PY@bf=\textbf\def\PY@tc##1{\textcolor[rgb]{0.00,0.50,0.00}{##1}}}
\expandafter\def\csname PY@tok@kd\endcsname{\let\PY@bf=\textbf\def\PY@tc##1{\textcolor[rgb]{0.00,0.50,0.00}{##1}}}
\expandafter\def\csname PY@tok@kn\endcsname{\let\PY@bf=\textbf\def\PY@tc##1{\textcolor[rgb]{0.00,0.50,0.00}{##1}}}
\expandafter\def\csname PY@tok@kr\endcsname{\let\PY@bf=\textbf\def\PY@tc##1{\textcolor[rgb]{0.00,0.50,0.00}{##1}}}
\expandafter\def\csname PY@tok@bp\endcsname{\def\PY@tc##1{\textcolor[rgb]{0.00,0.50,0.00}{##1}}}
\expandafter\def\csname PY@tok@fm\endcsname{\def\PY@tc##1{\textcolor[rgb]{0.00,0.00,1.00}{##1}}}
\expandafter\def\csname PY@tok@vc\endcsname{\def\PY@tc##1{\textcolor[rgb]{0.10,0.09,0.49}{##1}}}
\expandafter\def\csname PY@tok@vg\endcsname{\def\PY@tc##1{\textcolor[rgb]{0.10,0.09,0.49}{##1}}}
\expandafter\def\csname PY@tok@vi\endcsname{\def\PY@tc##1{\textcolor[rgb]{0.10,0.09,0.49}{##1}}}
\expandafter\def\csname PY@tok@vm\endcsname{\def\PY@tc##1{\textcolor[rgb]{0.10,0.09,0.49}{##1}}}
\expandafter\def\csname PY@tok@sa\endcsname{\def\PY@tc##1{\textcolor[rgb]{0.73,0.13,0.13}{##1}}}
\expandafter\def\csname PY@tok@sb\endcsname{\def\PY@tc##1{\textcolor[rgb]{0.73,0.13,0.13}{##1}}}
\expandafter\def\csname PY@tok@sc\endcsname{\def\PY@tc##1{\textcolor[rgb]{0.73,0.13,0.13}{##1}}}
\expandafter\def\csname PY@tok@dl\endcsname{\def\PY@tc##1{\textcolor[rgb]{0.73,0.13,0.13}{##1}}}
\expandafter\def\csname PY@tok@s2\endcsname{\def\PY@tc##1{\textcolor[rgb]{0.73,0.13,0.13}{##1}}}
\expandafter\def\csname PY@tok@sh\endcsname{\def\PY@tc##1{\textcolor[rgb]{0.73,0.13,0.13}{##1}}}
\expandafter\def\csname PY@tok@s1\endcsname{\def\PY@tc##1{\textcolor[rgb]{0.73,0.13,0.13}{##1}}}
\expandafter\def\csname PY@tok@mb\endcsname{\def\PY@tc##1{\textcolor[rgb]{0.40,0.40,0.40}{##1}}}
\expandafter\def\csname PY@tok@mf\endcsname{\def\PY@tc##1{\textcolor[rgb]{0.40,0.40,0.40}{##1}}}
\expandafter\def\csname PY@tok@mh\endcsname{\def\PY@tc##1{\textcolor[rgb]{0.40,0.40,0.40}{##1}}}
\expandafter\def\csname PY@tok@mi\endcsname{\def\PY@tc##1{\textcolor[rgb]{0.40,0.40,0.40}{##1}}}
\expandafter\def\csname PY@tok@il\endcsname{\def\PY@tc##1{\textcolor[rgb]{0.40,0.40,0.40}{##1}}}
\expandafter\def\csname PY@tok@mo\endcsname{\def\PY@tc##1{\textcolor[rgb]{0.40,0.40,0.40}{##1}}}
\expandafter\def\csname PY@tok@ch\endcsname{\let\PY@it=\textit\def\PY@tc##1{\textcolor[rgb]{0.25,0.50,0.50}{##1}}}
\expandafter\def\csname PY@tok@cm\endcsname{\let\PY@it=\textit\def\PY@tc##1{\textcolor[rgb]{0.25,0.50,0.50}{##1}}}
\expandafter\def\csname PY@tok@cpf\endcsname{\let\PY@it=\textit\def\PY@tc##1{\textcolor[rgb]{0.25,0.50,0.50}{##1}}}
\expandafter\def\csname PY@tok@c1\endcsname{\let\PY@it=\textit\def\PY@tc##1{\textcolor[rgb]{0.25,0.50,0.50}{##1}}}
\expandafter\def\csname PY@tok@cs\endcsname{\let\PY@it=\textit\def\PY@tc##1{\textcolor[rgb]{0.25,0.50,0.50}{##1}}}

\def\PYZbs{\char`\\}
\def\PYZus{\char`\_}
\def\PYZob{\char`\{}
\def\PYZcb{\char`\}}
\def\PYZca{\char`\^}
\def\PYZam{\char`\&}
\def\PYZlt{\char`\<}
\def\PYZgt{\char`\>}
\def\PYZsh{\char`\#}
\def\PYZpc{\char`\%}
\def\PYZdl{\char`\$}
\def\PYZhy{\char`\-}
\def\PYZsq{\char`\'}
\def\PYZdq{\char`\"}
\def\PYZti{\char`\~}
% for compatibility with earlier versions
\def\PYZat{@}
\def\PYZlb{[}
\def\PYZrb{]}
\makeatother


    % Exact colors from NB
    \definecolor{incolor}{rgb}{0.0, 0.0, 0.5}
    \definecolor{outcolor}{rgb}{0.545, 0.0, 0.0}



    
    % Prevent overflowing lines due to hard-to-break entities
    \sloppy 
    % Setup hyperref package
    \hypersetup{
      breaklinks=true,  % so long urls are correctly broken across lines
      colorlinks=true,
      urlcolor=urlcolor,
      linkcolor=linkcolor,
      citecolor=citecolor,
      }
    % Slightly bigger margins than the latex defaults
    
    \geometry{verbose,tmargin=1in,bmargin=1in,lmargin=1in,rmargin=1in}
    
    

    \begin{document}
    
    
    \maketitle
    
    

    
    \hypertarget{introduction}{%
\section{Introduction}\label{introduction}}

In this assignment a tubelight is simulated. Whenever a tubelight is
used the electrons move from the cathode with zero energy at the cathode
to the anode.The electric field in between accelerates the electrons and
thus energizes it.Therefore if some electron crosses a threshold energy,
then it can excite the atom which it hits and thus it can emit light.
But there is some probability with which the electron hits an atom.

\hypertarget{assumptions-in-the-simulation}{%
\subsection{Assumptions in the
simulation:}\label{assumptions-in-the-simulation}}

1:The electron comes to rest (i.e.~it loses all of it's energy) after
collision and is made to accelerate again starting with zero velocity.\\
2:The electron acceleration due to the electron field is equal to
1\(m/s^2\). \\3:The time between each update of displacement,velocity and
acceleration is 1 sec. \\4:Five new electrons with standard deviation of 2
are added into the tubelight after every turn of the simulation.\\

    \hypertarget{import-libraries}{%
\section{Import Libraries}\label{import-libraries}}

    \begin{Verbatim}[commandchars=\\\{\}]
{\color{incolor}In [{\color{incolor}1}]:} \PY{k+kn}{from} \PY{n+nn}{pylab} \PY{k}{import} \PY{o}{*}
        \PY{k+kn}{import} \PY{n+nn}{sys}
\end{Verbatim}


    \hypertarget{take-input-from-user-or-use-a-set-of-predefined-parameters}{%
\section{Take input from user or use a set of predefined
parameters}\label{take-input-from-user-or-use-a-set-of-predefined-parameters}}

    \begin{Verbatim}[commandchars=\\\{\}]
{\color{incolor}In [{\color{incolor}2}]:} \PY{k}{if} \PY{n+nb}{len}\PY{p}{(}\PY{n}{sys}\PY{o}{.}\PY{n}{argv}\PY{p}{)}\PY{o}{==}\PY{l+m+mi}{6}\PY{p}{:}
            \PY{n}{n}\PY{o}{=}\PY{n}{sys}\PY{o}{.}\PY{n}{argv}\PY{p}{[}\PY{l+m+mi}{0}\PY{p}{]}  \PY{c+c1}{\PYZsh{} spatial grid size.}
            \PY{n}{M}\PY{o}{=}\PY{n}{sys}\PY{o}{.}\PY{n}{argv}\PY{p}{[}\PY{l+m+mi}{1}\PY{p}{]}    \PY{c+c1}{\PYZsh{} number of electrons injected per turn.}
            \PY{n}{nk}\PY{o}{=}\PY{n}{sys}\PY{o}{.}\PY{n}{argv}\PY{p}{[}\PY{l+m+mi}{2}\PY{p}{]} \PY{c+c1}{\PYZsh{} number of turns to simulate.}
            \PY{n}{u0}\PY{o}{=}\PY{n}{sys}\PY{o}{.}\PY{n}{argv}\PY{p}{[}\PY{l+m+mi}{3}\PY{p}{]}  \PY{c+c1}{\PYZsh{} threshold velocity.}
            \PY{n}{p}\PY{o}{=}\PY{n}{sys}\PY{o}{.}\PY{n}{argv}\PY{p}{[}\PY{l+m+mi}{4}\PY{p}{]} \PY{c+c1}{\PYZsh{} probability that ionization will occur}
            \PY{n}{Msig}\PY{o}{=}\PY{n}{sys}\PY{o}{.}\PY{n}{argv}\PY{p}{[}\PY{l+m+mi}{5}\PY{p}{]} \PY{c+c1}{\PYZsh{}deviation of elctrons injected per turn}
            \PY{n}{params}\PY{o}{=}\PY{p}{[}\PY{n}{n}\PY{p}{,}\PY{n}{M}\PY{p}{,}\PY{n}{nk}\PY{p}{,}\PY{n}{u0}\PY{p}{,}\PY{n}{p}\PY{p}{,}\PY{n}{Msig}\PY{p}{]}
        \PY{k}{else}\PY{p}{:}
            \PY{n}{params}\PY{o}{=}\PY{p}{[}\PY{p}{[}\PY{l+m+mi}{100}\PY{p}{,}\PY{l+m+mi}{5}\PY{p}{,}\PY{l+m+mi}{500}\PY{p}{,}\PY{l+m+mi}{5}\PY{p}{,}\PY{l+m+mf}{0.25}\PY{p}{,}\PY{l+m+mi}{2}\PY{p}{]}\PY{p}{,}\PY{p}{[}\PY{l+m+mi}{100}\PY{p}{,}\PY{l+m+mi}{5}\PY{p}{,}\PY{l+m+mi}{500}\PY{p}{,}\PY{l+m+mi}{5}\PY{p}{,}\PY{l+m+mf}{0.50}\PY{p}{,}\PY{l+m+mi}{2}\PY{p}{]}\PY{p}{,}
                \PY{p}{[}\PY{l+m+mi}{100}\PY{p}{,}\PY{l+m+mi}{5}\PY{p}{,}\PY{l+m+mi}{500}\PY{p}{,}\PY{l+m+mi}{5}\PY{p}{,}\PY{l+m+mi}{1}\PY{p}{,}\PY{l+m+mi}{2}\PY{p}{]}\PY{p}{,}\PY{p}{[}\PY{l+m+mi}{100}\PY{p}{,}\PY{l+m+mi}{5}\PY{p}{,}\PY{l+m+mi}{500}\PY{p}{,}\PY{l+m+mi}{10}\PY{p}{,}\PY{l+m+mf}{0.25}\PY{p}{,}\PY{l+m+mi}{2}\PY{p}{]}\PY{p}{,}
                \PY{p}{[}\PY{l+m+mi}{100}\PY{p}{,}\PY{l+m+mi}{5}\PY{p}{,}\PY{l+m+mi}{500}\PY{p}{,}\PY{l+m+mi}{10}\PY{p}{,}\PY{l+m+mf}{0.50}\PY{p}{,}\PY{l+m+mi}{2}\PY{p}{]}\PY{p}{,}\PY{p}{[}\PY{l+m+mi}{100}\PY{p}{,}\PY{l+m+mi}{5}\PY{p}{,}\PY{l+m+mi}{500}\PY{p}{,}\PY{l+m+mi}{10}\PY{p}{,}\PY{l+m+mi}{1}\PY{p}{,}\PY{l+m+mi}{2}\PY{p}{]}\PY{p}{]}
\end{Verbatim}


    \hypertarget{the-loop-of-the-simulation}{%
\section{The loop of the simulation}\label{the-loop-of-the-simulation}}

Here a function is defined which runs through the main loop of the
simulation. \\Here the algorithm used is as defined in the problem.\\ *
Note: The algorithm is explained in the comments

    \begin{Verbatim}[commandchars=\\\{\}]
{\color{incolor}In [{\color{incolor}3}]:} \PY{k}{def} \PY{n+nf}{loop}\PY{p}{(}\PY{n}{params}\PY{p}{,}\PY{n}{quadratic}\PY{p}{)}\PY{p}{:}
            \PY{n}{n}\PY{o}{=}\PY{n}{params}\PY{p}{[}\PY{l+m+mi}{0}\PY{p}{]}
            \PY{n}{M}\PY{o}{=}\PY{n}{params}\PY{p}{[}\PY{l+m+mi}{1}\PY{p}{]}
            \PY{n}{nk}\PY{o}{=}\PY{n}{params}\PY{p}{[}\PY{l+m+mi}{2}\PY{p}{]}
            \PY{n}{u0}\PY{o}{=}\PY{n}{params}\PY{p}{[}\PY{l+m+mi}{3}\PY{p}{]}
            \PY{n}{p}\PY{o}{=}\PY{n}{params}\PY{p}{[}\PY{l+m+mi}{4}\PY{p}{]}
            \PY{n}{Msig}\PY{o}{=}\PY{n}{params}\PY{p}{[}\PY{l+m+mi}{5}\PY{p}{]}
            \PY{n}{xx}\PY{o}{=}\PY{n}{np}\PY{o}{.}\PY{n}{zeros}\PY{p}{(}\PY{p}{(}\PY{n}{n}\PY{o}{*}\PY{n}{M}\PY{p}{)}\PY{p}{)}  \PY{c+c1}{\PYZsh{} electron position}
            \PY{n}{u}\PY{o}{=}\PY{n}{np}\PY{o}{.}\PY{n}{zeros}\PY{p}{(}\PY{p}{(}\PY{n}{n}\PY{o}{*}\PY{n}{M}\PY{p}{)}\PY{p}{)}   \PY{c+c1}{\PYZsh{}electron velocity}
            \PY{n}{dx}\PY{o}{=}\PY{n}{np}\PY{o}{.}\PY{n}{zeros}\PY{p}{(}\PY{p}{(}\PY{n}{n}\PY{o}{*}\PY{n}{M}\PY{p}{)}\PY{p}{)}  \PY{c+c1}{\PYZsh{}displacement in current turn}
        
            \PY{n}{I}\PY{o}{=}\PY{p}{[}\PY{p}{]}
            \PY{n}{V}\PY{o}{=}\PY{p}{[}\PY{p}{]}
            \PY{n}{X}\PY{o}{=}\PY{p}{[}\PY{p}{]}
            
            \PY{k}{for} \PY{n}{i} \PY{o+ow}{in} \PY{n+nb}{range}\PY{p}{(}\PY{l+m+mi}{1}\PY{p}{,}\PY{n}{nk}\PY{p}{)}\PY{p}{:}
                \PY{n}{ii}\PY{o}{=}\PY{n}{where}\PY{p}{(}\PY{n}{xx}\PY{o}{\PYZgt{}}\PY{l+m+mi}{0}\PY{p}{)} \PY{c+c1}{\PYZsh{}get the indices of positions greater than zero}
                \PY{n}{dx}\PY{p}{[}\PY{n}{ii}\PY{p}{]}\PY{o}{=}\PY{n}{u}\PY{p}{[}\PY{n}{ii}\PY{p}{]}\PY{o}{+}\PY{l+m+mf}{0.5} \PY{c+c1}{\PYZsh{}increase the displacement}
                \PY{n}{xx}\PY{p}{[}\PY{n}{ii}\PY{p}{]}\PY{o}{+}\PY{o}{=}\PY{n}{dx}\PY{p}{[}\PY{n}{ii}\PY{p}{]} \PY{c+c1}{\PYZsh{}increase the position}
                \PY{n}{u}\PY{p}{[}\PY{n}{ii}\PY{p}{]}\PY{o}{+}\PY{o}{=}\PY{l+m+mi}{1} \PY{c+c1}{\PYZsh{}increase the velocity}
                \PY{n}{reached}\PY{o}{=}\PY{n}{where}\PY{p}{(}\PY{n}{xx}\PY{p}{[}\PY{n}{ii}\PY{p}{]}\PY{o}{\PYZgt{}}\PY{n}{n}\PY{p}{)}\PY{c+c1}{\PYZsh{}contains the indices }
                \PY{c+c1}{\PYZsh{}set position,velocities,displacements to zero}
                \PY{n}{xx}\PY{p}{[}\PY{n}{ii}\PY{p}{[}\PY{l+m+mi}{0}\PY{p}{]}\PY{p}{[}\PY{n}{reached}\PY{p}{]}\PY{p}{]}\PY{o}{=}\PY{n}{u}\PY{p}{[}\PY{n}{ii}\PY{p}{[}\PY{l+m+mi}{0}\PY{p}{]}\PY{p}{[}\PY{n}{reached}\PY{p}{]}\PY{p}{]}\PY{o}{=}\PY{n}{dx}\PY{p}{[}\PY{n}{ii}\PY{p}{[}\PY{l+m+mi}{0}\PY{p}{]}\PY{p}{[}\PY{n}{reached}\PY{p}{]}\PY{p}{]}\PY{o}{=}\PY{l+m+mi}{0}
                \PY{n}{kk}\PY{o}{=}\PY{n}{where}\PY{p}{(}\PY{n}{u}\PY{o}{\PYZgt{}}\PY{o}{=}\PY{n}{u0}\PY{p}{)}
                \PY{n}{ll}\PY{o}{=}\PY{n}{where}\PY{p}{(}\PY{n}{rand}\PY{p}{(}\PY{n+nb}{len}\PY{p}{(}\PY{n}{kk}\PY{p}{[}\PY{l+m+mi}{0}\PY{p}{]}\PY{p}{)}\PY{p}{)}\PY{o}{\PYZlt{}}\PY{o}{=}\PY{n}{p}\PY{p}{)}
                \PY{n}{kl}\PY{o}{=}\PY{n}{kk}\PY{p}{[}\PY{l+m+mi}{0}\PY{p}{]}\PY{p}{[}\PY{n}{ll}\PY{p}{]}\PY{c+c1}{\PYZsh{}contains the indices }
                \PY{c+c1}{\PYZsh{}of energetic electrons that suffer collision}
                \PY{n}{u}\PY{p}{[}\PY{n}{kl}\PY{p}{]}\PY{o}{=}\PY{l+m+mi}{0} \PY{c+c1}{\PYZsh{} reset the velocity after collision}
                \PY{k}{if} \PY{n}{quadratic}\PY{o}{==}\PY{k+kc}{False}\PY{p}{:}
                    \PY{n}{rho}\PY{o}{=}\PY{n}{rand}\PY{p}{(}\PY{n+nb}{len}\PY{p}{(}\PY{n}{kl}\PY{p}{)}\PY{p}{)} \PY{c+c1}{\PYZsh{}get random number }
                \PY{k}{if} \PY{n}{quadratic}\PY{o}{==}\PY{k+kc}{True}\PY{p}{:}    
                    \PY{n}{rho}\PY{o}{=}\PY{n}{power}\PY{p}{(}\PY{l+m+mf}{0.5}\PY{p}{,}\PY{n+nb}{len}\PY{p}{(}\PY{n}{kl}\PY{p}{)}\PY{p}{)} \PY{c+c1}{\PYZsh{} a quadratic probability distribution}
                \PY{n}{xx}\PY{p}{[}\PY{n}{kl}\PY{p}{]}\PY{o}{=}\PY{n}{xx}\PY{p}{[}\PY{n}{kl}\PY{p}{]}\PY{o}{\PYZhy{}}\PY{n}{dx}\PY{p}{[}\PY{n}{kl}\PY{p}{]}\PY{o}{*}\PY{n}{rho} \PY{c+c1}{\PYZsh{}get the actual value of x where it collides}
                \PY{n}{I}\PY{o}{.}\PY{n}{extend}\PY{p}{(}\PY{n}{xx}\PY{p}{[}\PY{n}{kl}\PY{p}{]}\PY{o}{.}\PY{n}{tolist}\PY{p}{(}\PY{p}{)}\PY{p}{)}
                \PY{n}{m}\PY{o}{=}\PY{n+nb}{int}\PY{p}{(}\PY{n}{rand}\PY{p}{(}\PY{p}{)}\PY{o}{*}\PY{n}{Msig}\PY{o}{+}\PY{n}{M}\PY{p}{)} \PY{c+c1}{\PYZsh{}get the (random)number of new electrons to be added}
                \PY{n}{empty}\PY{o}{=}\PY{n}{where}\PY{p}{(}\PY{n}{xx}\PY{o}{==}\PY{l+m+mi}{0}\PY{p}{)} \PY{c+c1}{\PYZsh{}get empty spaces where electrons can be injected}
                \PY{n}{nv}\PY{o}{=}\PY{p}{(}\PY{n+nb}{min}\PY{p}{(}\PY{n}{n}\PY{o}{*}\PY{n}{M}\PY{o}{\PYZhy{}}\PY{n+nb}{len}\PY{p}{(}\PY{n}{empty}\PY{p}{)}\PY{p}{,}\PY{n}{m}\PY{p}{)}\PY{p}{)} \PY{c+c1}{\PYZsh{}if no empty spaces are left}
                \PY{n}{xx}\PY{p}{[}\PY{n}{empty}\PY{p}{[}\PY{p}{:}\PY{n}{nv}\PY{p}{]}\PY{p}{]}\PY{o}{=}\PY{l+m+mi}{1} \PY{c+c1}{\PYZsh{}inject the new electrons}
                \PY{n}{u}\PY{p}{[}\PY{n}{empty}\PY{p}{[}\PY{l+m+mi}{0}\PY{p}{]}\PY{p}{[}\PY{p}{:}\PY{n}{nv}\PY{p}{]}\PY{p}{]}\PY{o}{=}\PY{l+m+mi}{0} \PY{c+c1}{\PYZsh{}with velocity zero}
                \PY{n}{dx}\PY{p}{[}\PY{n}{empty}\PY{p}{[}\PY{l+m+mi}{0}\PY{p}{]}\PY{p}{[}\PY{p}{:}\PY{n}{nv}\PY{p}{]}\PY{p}{]}\PY{o}{=}\PY{l+m+mi}{0} \PY{c+c1}{\PYZsh{}and displacement zero}
                \PY{n}{X}\PY{o}{.}\PY{n}{extend}\PY{p}{(}\PY{n}{xx}\PY{o}{.}\PY{n}{tolist}\PY{p}{(}\PY{p}{)}\PY{p}{)}
                \PY{n}{V}\PY{o}{.}\PY{n}{extend}\PY{p}{(}\PY{n}{u}\PY{o}{.}\PY{n}{tolist}\PY{p}{(}\PY{p}{)}\PY{p}{)}
                        
            \PY{k}{return} \PY{n}{X}\PY{p}{,}\PY{n}{V}\PY{p}{,}\PY{n}{I}
\end{Verbatim}


    \hypertarget{the-functions-to-plot-the-graphs}{%
\section{The functions to plot the
graphs}\label{the-functions-to-plot-the-graphs}}

\texttt{plot\_no\_of\_elec} plots the number of electrons vs x.\\
\texttt{plot\_intensity} plots the intensity vs x.\\
\texttt{plot\_intensity\_map} plots the relative brightness of the
tubelight in grayscale.\\ \texttt{plot\_phase} plots the phase space of
the electrons.

    \begin{Verbatim}[commandchars=\\\{\}]
{\color{incolor}In [{\color{incolor}6}]:} \PY{k}{def} \PY{n+nf}{plot\PYZus{}no\PYZus{}of\PYZus{}elec}\PY{p}{(}\PY{n}{X}\PY{p}{,}\PY{n}{u0}\PY{p}{,}\PY{n}{p}\PY{p}{)}\PY{p}{:}
        \PY{c+c1}{\PYZsh{}plot the number of electrons vs x}
            \PY{n}{figure}\PY{p}{(}\PY{l+m+mi}{1}\PY{p}{)}
            \PY{n}{hist}\PY{p}{(}\PY{n}{X}\PY{p}{,}\PY{n}{bins}\PY{o}{=}\PY{n}{np}\PY{o}{.}\PY{n}{arange}\PY{p}{(}\PY{l+m+mi}{0}\PY{p}{,}\PY{l+m+mi}{101}\PY{p}{,}\PY{l+m+mf}{0.5}\PY{p}{)}\PY{p}{,}\PY{n}{rwidth}\PY{o}{=}\PY{l+m+mf}{0.8}\PY{p}{,}\PY{n}{color}\PY{o}{=}\PY{l+s+s1}{\PYZsq{}}\PY{l+s+s1}{g}\PY{l+s+s1}{\PYZsq{}}\PY{p}{)}
            \PY{n}{title}\PY{p}{(}\PY{l+s+s1}{\PYZsq{}}\PY{l+s+s1}{Number of Electrons vs \PYZdl{}x\PYZdl{} with \PYZdl{}u\PYZus{}0=\PYZdl{}}\PY{l+s+si}{\PYZpc{}f}\PY{l+s+s1}{ and p=}\PY{l+s+si}{\PYZpc{}f}\PY{l+s+s1}{\PYZsq{}}\PY{o}{\PYZpc{}}\PY{p}{(}\PY{n}{u0}\PY{p}{,}\PY{n}{p}\PY{p}{)}\PY{p}{)}
            \PY{n}{xlabel}\PY{p}{(}\PY{l+s+s1}{\PYZsq{}}\PY{l+s+s1}{\PYZdl{}x\PYZdl{}}\PY{l+s+s1}{\PYZsq{}}\PY{p}{)}
            \PY{n}{ylabel}\PY{p}{(}\PY{l+s+s1}{\PYZsq{}}\PY{l+s+s1}{Number of electrons}\PY{l+s+s1}{\PYZsq{}}\PY{p}{)}
            \PY{n}{show}\PY{p}{(}\PY{p}{)}
        
        \PY{k}{def} \PY{n+nf}{plot\PYZus{}intensity\PYZus{}map}\PY{p}{(}\PY{n}{I}\PY{p}{,}\PY{n}{u0}\PY{p}{,}\PY{n}{p}\PY{p}{)}\PY{p}{:}
        \PY{c+c1}{\PYZsh{}plot the intensity map}
            \PY{n}{histogram\PYZus{}}\PY{o}{=}\PY{n}{hist}\PY{p}{(}\PY{n}{I}\PY{p}{,}\PY{n}{bins}\PY{o}{=}\PY{n}{np}\PY{o}{.}\PY{n}{arange}\PY{p}{(}\PY{l+m+mi}{0}\PY{p}{,}\PY{l+m+mi}{101}\PY{p}{,}\PY{l+m+mi}{1}\PY{p}{)}\PY{p}{,}\PY{n}{rwidth}\PY{o}{=}\PY{l+m+mf}{0.8}\PY{p}{,}\PY{n}{color}\PY{o}{=}\PY{l+s+s1}{\PYZsq{}}\PY{l+s+s1}{r}\PY{l+s+s1}{\PYZsq{}}\PY{p}{)}
            \PY{n}{x}\PY{o}{=}\PY{n}{histogram\PYZus{}}\PY{p}{[}\PY{l+m+mi}{1}\PY{p}{]}\PY{p}{[}\PY{l+m+mi}{1}\PY{p}{:}\PY{p}{]}
            \PY{n}{y}\PY{o}{=}\PY{n}{histogram\PYZus{}}\PY{p}{[}\PY{l+m+mi}{0}\PY{p}{]}
            \PY{n}{fig}\PY{p}{,} \PY{p}{(}\PY{n}{ax}\PY{p}{)} \PY{o}{=} \PY{n}{plt}\PY{o}{.}\PY{n}{subplots}\PY{p}{(}\PY{n}{nrows}\PY{o}{=}\PY{l+m+mi}{1}\PY{p}{,} \PY{n}{sharex}\PY{o}{=}\PY{k+kc}{True}\PY{p}{)}
            \PY{n}{extent} \PY{o}{=} \PY{p}{[}\PY{n}{x}\PY{p}{[}\PY{l+m+mi}{0}\PY{p}{]}\PY{o}{\PYZhy{}}\PY{p}{(}\PY{n}{x}\PY{p}{[}\PY{l+m+mi}{1}\PY{p}{]}\PY{o}{\PYZhy{}}\PY{n}{x}\PY{p}{[}\PY{l+m+mi}{0}\PY{p}{]}\PY{p}{)}\PY{o}{/}\PY{l+m+mf}{2.}\PY{p}{,} \PY{n}{x}\PY{p}{[}\PY{o}{\PYZhy{}}\PY{l+m+mi}{1}\PY{p}{]}\PY{o}{+}\PY{p}{(}\PY{n}{x}\PY{p}{[}\PY{l+m+mi}{1}\PY{p}{]}\PY{o}{\PYZhy{}}\PY{n}{x}\PY{p}{[}\PY{l+m+mi}{0}\PY{p}{]}\PY{p}{)}\PY{o}{/}\PY{l+m+mf}{2.}\PY{p}{,}\PY{l+m+mi}{0}\PY{p}{,}\PY{l+m+mi}{1}\PY{p}{]}
            \PY{n}{intensity}\PY{o}{=}\PY{n}{ax}\PY{o}{.}\PY{n}{imshow}\PY{p}{(}\PY{n}{y}\PY{p}{[}\PY{n}{np}\PY{o}{.}\PY{n}{newaxis}\PY{p}{,}\PY{p}{:}\PY{p}{]}\PY{p}{,} \PY{n}{cmap}\PY{o}{=}\PY{l+s+s2}{\PYZdq{}}\PY{l+s+s2}{gray}\PY{l+s+s2}{\PYZdq{}}\PY{p}{,} \PY{n}{aspect}\PY{o}{=}\PY{l+s+s2}{\PYZdq{}}\PY{l+s+s2}{auto}\PY{l+s+s2}{\PYZdq{}}\PY{p}{,} \PY{n}{extent}\PY{o}{=}\PY{n}{extent}\PY{p}{)}
            \PY{n}{ax}\PY{o}{.}\PY{n}{set\PYZus{}yticks}\PY{p}{(}\PY{p}{[}\PY{p}{]}\PY{p}{)}
            \PY{n}{ax}\PY{o}{.}\PY{n}{set\PYZus{}xlim}\PY{p}{(}\PY{n}{extent}\PY{p}{[}\PY{l+m+mi}{0}\PY{p}{]}\PY{p}{,} \PY{n}{extent}\PY{p}{[}\PY{l+m+mi}{1}\PY{p}{]}\PY{p}{)}
            \PY{n}{plt}\PY{o}{.}\PY{n}{title}\PY{p}{(}\PY{l+s+s1}{\PYZsq{}}\PY{l+s+s1}{Intensity map with \PYZdl{}u\PYZus{}0=\PYZdl{}}\PY{l+s+si}{\PYZpc{}f}\PY{l+s+s1}{ and p=}\PY{l+s+si}{\PYZpc{}f}\PY{l+s+s1}{\PYZsq{}}\PY{o}{\PYZpc{}}\PY{p}{(}\PY{n}{u0}\PY{p}{,}\PY{n}{p}\PY{p}{)}\PY{p}{)}
            \PY{n}{plt}\PY{o}{.}\PY{n}{xlabel}\PY{p}{(}\PY{l+s+s1}{\PYZsq{}}\PY{l+s+s1}{\PYZdl{}x\PYZdl{}}\PY{l+s+s1}{\PYZsq{}}\PY{p}{)}
            \PY{n}{plt}\PY{o}{.}\PY{n}{colorbar}\PY{p}{(}\PY{n}{intensity}\PY{p}{)}
            \PY{n}{plt}\PY{o}{.}\PY{n}{tight\PYZus{}layout}\PY{p}{(}\PY{p}{)}
            \PY{n}{show}\PY{p}{(}\PY{p}{)}
            
        \PY{k}{def} \PY{n+nf}{plot\PYZus{}intensity}\PY{p}{(}\PY{n}{X}\PY{p}{,}\PY{n}{V}\PY{p}{,}\PY{n}{u0}\PY{p}{,}\PY{n}{p}\PY{p}{)}\PY{p}{:}
            \PY{c+c1}{\PYZsh{}plot the histogram of intensity}
            \PY{n}{figure}\PY{p}{(}\PY{l+m+mi}{0}\PY{p}{)}
            \PY{n}{histogram}\PY{o}{=}\PY{n}{hist}\PY{p}{(}\PY{n}{I}\PY{p}{,}\PY{n}{bins}\PY{o}{=}\PY{n}{np}\PY{o}{.}\PY{n}{arange}\PY{p}{(}\PY{l+m+mi}{0}\PY{p}{,}\PY{l+m+mi}{101}\PY{p}{,}\PY{l+m+mf}{0.5}\PY{p}{)}\PY{p}{,}\PY{n}{rwidth}\PY{o}{=}\PY{l+m+mf}{0.8}\PY{p}{,}\PY{n}{color}\PY{o}{=}\PY{l+s+s1}{\PYZsq{}}\PY{l+s+s1}{r}\PY{l+s+s1}{\PYZsq{}}\PY{p}{)}
            \PY{n}{title}\PY{p}{(}\PY{l+s+s1}{\PYZsq{}}\PY{l+s+s1}{Intensity histogram with \PYZdl{}u\PYZus{}0=\PYZdl{}}\PY{l+s+si}{\PYZpc{}f}\PY{l+s+s1}{ and p=}\PY{l+s+si}{\PYZpc{}f}\PY{l+s+s1}{\PYZsq{}}\PY{o}{\PYZpc{}}\PY{p}{(}\PY{n}{u0}\PY{p}{,}\PY{n}{p}\PY{p}{)}\PY{p}{)}
            \PY{n}{xlabel}\PY{p}{(}\PY{l+s+s1}{\PYZsq{}}\PY{l+s+s1}{\PYZdl{}x\PYZdl{}}\PY{l+s+s1}{\PYZsq{}}\PY{p}{)}
            \PY{n}{ylabel}\PY{p}{(}\PY{l+s+s1}{\PYZsq{}}\PY{l+s+s1}{Intensity}\PY{l+s+s1}{\PYZsq{}}\PY{p}{)}
            \PY{n}{show}\PY{p}{(}\PY{p}{)}
            \PY{k}{return} \PY{n}{histogram}
        
        \PY{k}{def} \PY{n+nf}{plot\PYZus{}phase}\PY{p}{(}\PY{n}{X}\PY{p}{,}\PY{n}{V}\PY{p}{,}\PY{n}{u0}\PY{p}{,}\PY{n}{p}\PY{p}{)}\PY{p}{:}
        \PY{c+c1}{\PYZsh{}plot the phase space}
            \PY{n}{figure}\PY{p}{(}\PY{l+m+mi}{2}\PY{p}{)}
            \PY{n}{plt}\PY{o}{.}\PY{n}{plot}\PY{p}{(}\PY{n}{X}\PY{p}{,}\PY{n}{V}\PY{p}{,}\PY{l+s+s1}{\PYZsq{}}\PY{l+s+s1}{bo}\PY{l+s+s1}{\PYZsq{}}\PY{p}{)}
            \PY{n}{title}\PY{p}{(}\PY{l+s+s1}{\PYZsq{}}\PY{l+s+s1}{Electron Phase Space with \PYZdl{}u\PYZus{}0=\PYZdl{}}\PY{l+s+si}{\PYZpc{}f}\PY{l+s+s1}{ and p=}\PY{l+s+si}{\PYZpc{}f}\PY{l+s+s1}{\PYZsq{}}\PY{o}{\PYZpc{}}\PY{p}{(}\PY{n}{u0}\PY{p}{,}\PY{n}{p}\PY{p}{)}\PY{p}{)}
            \PY{n}{xlabel}\PY{p}{(}\PY{l+s+s1}{\PYZsq{}}\PY{l+s+s1}{\PYZdl{}x\PYZdl{}}\PY{l+s+s1}{\PYZsq{}}\PY{p}{)}
            \PY{n}{ylabel}\PY{p}{(}\PY{l+s+s1}{\PYZsq{}}\PY{l+s+s1}{Velocity\PYZhy{}\PYZdl{}v\PYZdl{}}\PY{l+s+s1}{\PYZsq{}}\PY{p}{)}
            \PY{n}{show}\PY{p}{(}\PY{p}{)}
            
      
\end{Verbatim}


    \hypertarget{plot-of-the-graphs-with-uniform-probability-of-electron.}{%
\section{Plot of the graphs with uniform probability of
electron.}\label{plot-of-the-graphs-with-uniform-probability-of-electron.}}

In the intensity vs x graph the intensity reaches a maximum at around
x=15 and stays like that for around 4-5 bins and then it decreases.This
is because of the fact that the electron comes to rest after
collision.So it has to gain energy again from zero to be able to excite
the atom for emitting light. \\In the electron phase space the graph shows
the allowed velocities at a particular value of x thus we can say that
the velocities are quantized. \\The number of electrons vs x graph shows
the number of electrons which got excited at that value of x.

    \begin{Verbatim}[commandchars=\\\{\}]
{\color{incolor}In [{\color{incolor}7}]:} \PY{n}{param}\PY{o}{=}\PY{p}{[}\PY{l+m+mi}{100}\PY{p}{,}\PY{l+m+mi}{5}\PY{p}{,}\PY{l+m+mi}{500}\PY{p}{,}\PY{l+m+mi}{5}\PY{p}{,}\PY{l+m+mf}{0.25}\PY{p}{,}\PY{l+m+mi}{2}\PY{p}{]}
        \PY{n}{X}\PY{p}{,}\PY{n}{V}\PY{p}{,}\PY{n}{I}\PY{o}{=}\PY{n}{loop}\PY{p}{(}\PY{n}{param}\PY{p}{,}\PY{k+kc}{False}\PY{p}{)}
        \PY{n}{histo}\PY{o}{=}\PY{n}{plot\PYZus{}intensity}\PY{p}{(}\PY{n}{X}\PY{p}{,}\PY{n}{V}\PY{p}{,}\PY{n}{param}\PY{p}{[}\PY{l+m+mi}{3}\PY{p}{]}\PY{p}{,}\PY{n}{param}\PY{p}{[}\PY{l+m+mi}{4}\PY{p}{]}\PY{p}{)}
        \PY{n}{plot\PYZus{}no\PYZus{}of\PYZus{}elec}\PY{p}{(}\PY{n}{X}\PY{p}{,}\PY{n}{param}\PY{p}{[}\PY{l+m+mi}{3}\PY{p}{]}\PY{p}{,}\PY{n}{param}\PY{p}{[}\PY{l+m+mi}{4}\PY{p}{]}\PY{p}{)}
        \PY{n}{plot\PYZus{}phase}\PY{p}{(}\PY{n}{X}\PY{p}{,}\PY{n}{V}\PY{p}{,}\PY{n}{param}\PY{p}{[}\PY{l+m+mi}{3}\PY{p}{]}\PY{p}{,}\PY{n}{param}\PY{p}{[}\PY{l+m+mi}{4}\PY{p}{]}\PY{p}{)}
        \PY{n}{plot\PYZus{}intensity\PYZus{}map}\PY{p}{(}\PY{n}{I}\PY{p}{,}\PY{n}{param}\PY{p}{[}\PY{l+m+mi}{3}\PY{p}{]}\PY{p}{,}\PY{n}{param}\PY{p}{[}\PY{l+m+mi}{4}\PY{p}{]}\PY{p}{)}
\end{Verbatim}


    \begin{center}
    \adjustimage{max size={0.9\linewidth}{0.9\paperheight}}{output_10_0.png}
    \end{center}
    { \hspace*{\fill} \\}
    
    \begin{center}
    \adjustimage{max size={0.9\linewidth}{0.9\paperheight}}{output_10_1.png}
    \end{center}
    { \hspace*{\fill} \\}
    
    \begin{center}
    \adjustimage{max size={0.9\linewidth}{0.9\paperheight}}{output_10_2.png}
    \end{center}
    { \hspace*{\fill} \\}
    
    \begin{center}
    \adjustimage{max size={0.9\linewidth}{0.9\paperheight}}{output_10_4.png}
    \end{center}
    { \hspace*{\fill} \\}
    
    \hypertarget{table-of-population-vs-xpos}{%
\section{Table of Population vs
xpos}\label{table-of-population-vs-xpos}}

    


\begin{Verbatim}[commandchars=\\\{\}]
        Sr.No.  xpos  Population
         0      0.0         0.0
         2      1.0         0.0
         4      2.0         0.0
         6      3.0         0.0
         8      4.0         0.0
         10     5.0         0.0
         12     6.0         0.0
         14     7.0         0.0
         16     8.0         0.0
         18     9.0       248.0
         20    10.0       258.0
         22    11.0       256.0
         24    12.0       239.0
         26    13.0       227.0
         28    14.0       141.0
         30    15.0       146.0
         32    16.0       142.0
         34    17.0       147.0
         36    18.0       163.0
         38    19.0       128.0
         40    20.0       145.0
         42    21.0       157.0
         44    22.0       165.0
         46    23.0       149.0
         48    24.0       166.0
         50    25.0       172.0
         52    26.0       118.0
         54    27.0       149.0
         56    28.0       157.0
         58    29.0       153.0
         ..     {\ldots}         {\ldots}
         142   71.0       146.0
         144   72.0       133.0
         146   73.0       138.0
         148   74.0       133.0
         150   75.0       142.0
         152   76.0       140.0
         154   77.0       141.0
         156   78.0       119.0
         158   79.0       143.0
         160   80.0       147.0
         162   81.0       126.0
         164   82.0       114.0
         166   83.0       121.0
         168   84.0       117.0
         170   85.0       127.0
         172   86.0       149.0
         174   87.0       132.0
         176   88.0       132.0
         178   89.0       132.0
         180   90.0       135.0
         182   91.0       146.0
         184   92.0       120.0
         186   93.0       129.0
         188   94.0        99.0
         190   95.0       100.0
         192   96.0        92.0
         194   97.0        74.0
         196   98.0        34.0
         198   99.0        16.0
         200  100.0         0.0
        
 Note:For the purpose of showing the table every alternate element is shown in the table.
\end{Verbatim}
            

            
    \hypertarget{plot-of-the-graphs-with-quadratic-probability-of-electron.}{%
\section{Plot of the graphs with quadratic probability of
electron.}\label{plot-of-the-graphs-with-quadratic-probability-of-electron.}}

In the intensity vs x graph the intensity increases and then suddenly
drops.This is different than the previous model as the electron getting
excited is more towards the right side as it's velocity increases.

    \begin{Verbatim}[commandchars=\\\{\}]
{\color{incolor}In [{\color{incolor}15}]:} \PY{n}{param}\PY{o}{=}\PY{p}{[}\PY{l+m+mi}{100}\PY{p}{,}\PY{l+m+mi}{5}\PY{p}{,}\PY{l+m+mi}{500}\PY{p}{,}\PY{l+m+mi}{5}\PY{p}{,}\PY{l+m+mf}{0.25}\PY{p}{,}\PY{l+m+mi}{2}\PY{p}{]}
         \PY{n}{X}\PY{p}{,}\PY{n}{V}\PY{p}{,}\PY{n}{I}\PY{o}{=}\PY{n}{loop}\PY{p}{(}\PY{n}{param}\PY{p}{,}\PY{k+kc}{True}\PY{p}{)}
         \PY{n}{plot\PYZus{}intensity}\PY{p}{(}\PY{n}{X}\PY{p}{,}\PY{n}{V}\PY{p}{,}\PY{n}{param}\PY{p}{[}\PY{l+m+mi}{3}\PY{p}{]}\PY{p}{,}\PY{n}{param}\PY{p}{[}\PY{l+m+mi}{4}\PY{p}{]}\PY{p}{)}
         \PY{n}{plot\PYZus{}no\PYZus{}of\PYZus{}elec}\PY{p}{(}\PY{n}{X}\PY{p}{,}\PY{n}{param}\PY{p}{[}\PY{l+m+mi}{3}\PY{p}{]}\PY{p}{,}\PY{n}{param}\PY{p}{[}\PY{l+m+mi}{4}\PY{p}{]}\PY{p}{)}
         \PY{n}{plot\PYZus{}phase}\PY{p}{(}\PY{n}{X}\PY{p}{,}\PY{n}{V}\PY{p}{,}\PY{n}{param}\PY{p}{[}\PY{l+m+mi}{3}\PY{p}{]}\PY{p}{,}\PY{n}{param}\PY{p}{[}\PY{l+m+mi}{4}\PY{p}{]}\PY{p}{)}
         \PY{n}{plot\PYZus{}intensity\PYZus{}map}\PY{p}{(}\PY{n}{I}\PY{p}{,}\PY{n}{param}\PY{p}{[}\PY{l+m+mi}{3}\PY{p}{]}\PY{p}{,}\PY{n}{param}\PY{p}{[}\PY{l+m+mi}{4}\PY{p}{]}\PY{p}{)}
\end{Verbatim}


    \begin{center}
    \adjustimage{max size={0.9\linewidth}{0.9\paperheight}}{output_15_0.png}
    \end{center}
    { \hspace*{\fill} \\}
    
    \begin{center}
    \adjustimage{max size={0.9\linewidth}{0.9\paperheight}}{output_15_1.png}
    \end{center}
    { \hspace*{\fill} \\}
    
    \begin{center}
    \adjustimage{max size={0.9\linewidth}{0.9\paperheight}}{output_15_2.png}
    \end{center}
    { \hspace*{\fill} \\}
    
    \begin{center}
    \adjustimage{max size={0.9\linewidth}{0.9\paperheight}}{output_15_3.png}
    \end{center}
    { \hspace*{\fill} \\}
    
    \begin{center}
    \adjustimage{max size={0.9\linewidth}{0.9\paperheight}}{output_15_4.png}
    \end{center}
    { \hspace*{\fill} \\}
    
    \hypertarget{iterating-through-the-different-parameters}{%
\section{Iterating through the different
parameters}\label{iterating-through-the-different-parameters}}

The variation in the graphs when the cutoff velocity and probability are
captured here.\\ One specific observation is that as the probability
increases the graphs become more variate i.e.~the superposition of the
probability graphs become more seperated.\\Also the maximum intensity also
increases as the electrons get inonized more often.\\ Also as the cutoff
velocity (i.e.~the gas inside the tubelight is changed) is increased
they initial excitation happens at a higher value of x.This is because
the elctron has to travel longer distances to be able to reach cutoff
frequency.

    \begin{Verbatim}[commandchars=\\\{\}]
{\color{incolor}In [{\color{incolor}50}]:} \PY{k}{for} \PY{n}{param} \PY{o+ow}{in} \PY{n}{params}\PY{p}{:}
             \PY{n}{X}\PY{p}{,}\PY{n}{V}\PY{p}{,}\PY{n}{I}\PY{o}{=}\PY{n}{loop}\PY{p}{(}\PY{n}{param}\PY{p}{,}\PY{k+kc}{False}\PY{p}{)}
             \PY{n}{plot\PYZus{}intensity}\PY{p}{(}\PY{n}{X}\PY{p}{,}\PY{n}{V}\PY{p}{,}\PY{n}{param}\PY{p}{[}\PY{l+m+mi}{3}\PY{p}{]}\PY{p}{,}\PY{n}{param}\PY{p}{[}\PY{l+m+mi}{4}\PY{p}{]}\PY{p}{)}
             \PY{n}{plot\PYZus{}no\PYZus{}of\PYZus{}elec}\PY{p}{(}\PY{n}{X}\PY{p}{,}\PY{n}{param}\PY{p}{[}\PY{l+m+mi}{3}\PY{p}{]}\PY{p}{,}\PY{n}{param}\PY{p}{[}\PY{l+m+mi}{4}\PY{p}{]}\PY{p}{)}
             \PY{n}{plot\PYZus{}phase}\PY{p}{(}\PY{n}{X}\PY{p}{,}\PY{n}{V}\PY{p}{,}\PY{n}{param}\PY{p}{[}\PY{l+m+mi}{3}\PY{p}{]}\PY{p}{,}\PY{n}{param}\PY{p}{[}\PY{l+m+mi}{4}\PY{p}{]}\PY{p}{)}
             \PY{n}{plot\PYZus{}intensity\PYZus{}map}\PY{p}{(}\PY{n}{I}\PY{p}{,}\PY{n}{param}\PY{p}{[}\PY{l+m+mi}{3}\PY{p}{]}\PY{p}{,}\PY{n}{param}\PY{p}{[}\PY{l+m+mi}{4}\PY{p}{]}\PY{p}{)}
\end{Verbatim}


    \begin{center}
    \adjustimage{max size={0.9\linewidth}{0.9\paperheight}}{output_17_0.png}
    \end{center}
    { \hspace*{\fill} \\}
    
    \begin{center}
    \adjustimage{max size={0.9\linewidth}{0.9\paperheight}}{output_17_1.png}
    \end{center}
    { \hspace*{\fill} \\}
    
    \begin{center}
    \adjustimage{max size={0.9\linewidth}{0.9\paperheight}}{output_17_2.png}
    \end{center}
    { \hspace*{\fill} \\}
    
    \begin{center}
    \adjustimage{max size={0.9\linewidth}{0.9\paperheight}}{output_17_3.png}
    \end{center}
    { \hspace*{\fill} \\}
    
    \begin{center}
    \adjustimage{max size={0.9\linewidth}{0.9\paperheight}}{output_17_4.png}
    \end{center}
    { \hspace*{\fill} \\}
    
    \begin{center}
    \adjustimage{max size={0.9\linewidth}{0.9\paperheight}}{output_17_5.png}
    \end{center}
    { \hspace*{\fill} \\}
    
    \begin{center}
    \adjustimage{max size={0.9\linewidth}{0.9\paperheight}}{output_17_6.png}
    \end{center}
    { \hspace*{\fill} \\}
    
    \begin{center}
    \adjustimage{max size={0.9\linewidth}{0.9\paperheight}}{output_17_7.png}
    \end{center}
    { \hspace*{\fill} \\}
    
    \begin{center}
    \adjustimage{max size={0.9\linewidth}{0.9\paperheight}}{output_17_8.png}
    \end{center}
    { \hspace*{\fill} \\}
    
    \begin{center}
    \adjustimage{max size={0.9\linewidth}{0.9\paperheight}}{output_17_9.png}
    \end{center}
    { \hspace*{\fill} \\}
    
    \begin{center}
    \adjustimage{max size={0.9\linewidth}{0.9\paperheight}}{output_17_10.png}
    \end{center}
    { \hspace*{\fill} \\}
    
    \begin{center}
    \adjustimage{max size={0.9\linewidth}{0.9\paperheight}}{output_17_11.png}
    \end{center}
    { \hspace*{\fill} \\}
    
    \begin{center}
    \adjustimage{max size={0.9\linewidth}{0.9\paperheight}}{output_17_12.png}
    \end{center}
    { \hspace*{\fill} \\}
    
    \begin{center}
    \adjustimage{max size={0.9\linewidth}{0.9\paperheight}}{output_17_13.png}
    \end{center}
    { \hspace*{\fill} \\}
    
    \begin{center}
    \adjustimage{max size={0.9\linewidth}{0.9\paperheight}}{output_17_14.png}
    \end{center}
    { \hspace*{\fill} \\}
    
    \begin{center}
    \adjustimage{max size={0.9\linewidth}{0.9\paperheight}}{output_17_15.png}
    \end{center}
    { \hspace*{\fill} \\}
    
    \begin{center}
    \adjustimage{max size={0.9\linewidth}{0.9\paperheight}}{output_17_16.png}
    \end{center}
    { \hspace*{\fill} \\}
    
    \begin{center}
    \adjustimage{max size={0.9\linewidth}{0.9\paperheight}}{output_17_17.png}
    \end{center}
    { \hspace*{\fill} \\}
    
    \begin{center}
    \adjustimage{max size={0.9\linewidth}{0.9\paperheight}}{output_17_18.png}
    \end{center}
    { \hspace*{\fill} \\}
    
    \begin{center}
    \adjustimage{max size={0.9\linewidth}{0.9\paperheight}}{output_17_19.png}
    \end{center}
    { \hspace*{\fill} \\}
    
    \begin{center}
    \adjustimage{max size={0.9\linewidth}{0.9\paperheight}}{output_17_20.png}
    \end{center}
    { \hspace*{\fill} \\}
    
    \begin{center}
    \adjustimage{max size={0.9\linewidth}{0.9\paperheight}}{output_17_21.png}
    \end{center}
    { \hspace*{\fill} \\}
    
    \begin{center}
    \adjustimage{max size={0.9\linewidth}{0.9\paperheight}}{output_17_22.png}
    \end{center}
    { \hspace*{\fill} \\}
    
    \begin{center}
    \adjustimage{max size={0.9\linewidth}{0.9\paperheight}}{output_17_23.png}
    \end{center}
    { \hspace*{\fill} \\}
    
    \begin{center}
    \adjustimage{max size={0.9\linewidth}{0.9\paperheight}}{output_17_24.png}
    \end{center}
    { \hspace*{\fill} \\}
    
    \begin{center}
    \adjustimage{max size={0.9\linewidth}{0.9\paperheight}}{output_17_25.png}
    \end{center}
    { \hspace*{\fill} \\}
    
    \begin{center}
    \adjustimage{max size={0.9\linewidth}{0.9\paperheight}}{output_17_26.png}
    \end{center}
    { \hspace*{\fill} \\}
    
    \begin{center}
    \adjustimage{max size={0.9\linewidth}{0.9\paperheight}}{output_17_27.png}
    \end{center}
    { \hspace*{\fill} \\}
    
    \begin{center}
    \adjustimage{max size={0.9\linewidth}{0.9\paperheight}}{output_17_28.png}
    \end{center}
    { \hspace*{\fill} \\}
    
    \begin{center}
    \adjustimage{max size={0.9\linewidth}{0.9\paperheight}}{output_17_29.png}
    \end{center}
    { \hspace*{\fill} \\}
    

    % Add a bibliography block to the postdoc
    
    
    
    \end{document}
