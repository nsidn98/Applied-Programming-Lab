
% Default to the notebook output style

    


% Inherit from the specified cell style.




    
\documentclass[11pt]{article}

    
    
    \usepackage[T1]{fontenc}
    % Nicer default font (+ math font) than Computer Modern for most use cases
    \usepackage{mathpazo}

    % Basic figure setup, for now with no caption control since it's done
    % automatically by Pandoc (which extracts ![](path) syntax from Markdown).
    \usepackage{graphicx}
    % We will generate all images so they have a width \maxwidth. This means
    % that they will get their normal width if they fit onto the page, but
    % are scaled down if they would overflow the margins.
    \makeatletter
    \def\maxwidth{\ifdim\Gin@nat@width>\linewidth\linewidth
    \else\Gin@nat@width\fi}
    \makeatother
    \let\Oldincludegraphics\includegraphics
    % Set max figure width to be 80% of text width, for now hardcoded.
    \renewcommand{\includegraphics}[1]{\Oldincludegraphics[width=.8\maxwidth]{#1}}
    % Ensure that by default, figures have no caption (until we provide a
    % proper Figure object with a Caption API and a way to capture that
    % in the conversion process - todo).
    \usepackage{caption}
    \DeclareCaptionLabelFormat{nolabel}{}
    \captionsetup{labelformat=nolabel}

    \usepackage{adjustbox} % Used to constrain images to a maximum size 
    \usepackage{xcolor} % Allow colors to be defined
    \usepackage{enumerate} % Needed for markdown enumerations to work
    \usepackage{geometry} % Used to adjust the document margins
    \usepackage{amsmath} % Equations
    \usepackage{amssymb} % Equations
    \usepackage{textcomp} % defines textquotesingle
    % Hack from http://tex.stackexchange.com/a/47451/13684:
    \AtBeginDocument{%
        \def\PYZsq{\textquotesingle}% Upright quotes in Pygmentized code
    }
    \usepackage{upquote} % Upright quotes for verbatim code
    \usepackage{eurosym} % defines \euro
    \usepackage[mathletters]{ucs} % Extended unicode (utf-8) support
    \usepackage[utf8x]{inputenc} % Allow utf-8 characters in the tex document
    \usepackage{fancyvrb} % verbatim replacement that allows latex
    \usepackage{grffile} % extends the file name processing of package graphics 
                         % to support a larger range 
    % The hyperref package gives us a pdf with properly built
    % internal navigation ('pdf bookmarks' for the table of contents,
    % internal cross-reference links, web links for URLs, etc.)
    \usepackage{hyperref}
    \usepackage{longtable} % longtable support required by pandoc >1.10
    \usepackage{booktabs}  % table support for pandoc > 1.12.2
    \usepackage[inline]{enumitem} % IRkernel/repr support (it uses the enumerate* environment)
    \usepackage[normalem]{ulem} % ulem is needed to support strikethroughs (\sout)
                                % normalem makes italics be italics, not underlines
    

    
    
    % Colors for the hyperref package
    \definecolor{urlcolor}{rgb}{0,.145,.698}
    \definecolor{linkcolor}{rgb}{.71,0.21,0.01}
    \definecolor{citecolor}{rgb}{.12,.54,.11}

    % ANSI colors
    \definecolor{ansi-black}{HTML}{3E424D}
    \definecolor{ansi-black-intense}{HTML}{282C36}
    \definecolor{ansi-red}{HTML}{E75C58}
    \definecolor{ansi-red-intense}{HTML}{B22B31}
    \definecolor{ansi-green}{HTML}{00A250}
    \definecolor{ansi-green-intense}{HTML}{007427}
    \definecolor{ansi-yellow}{HTML}{DDB62B}
    \definecolor{ansi-yellow-intense}{HTML}{B27D12}
    \definecolor{ansi-blue}{HTML}{208FFB}
    \definecolor{ansi-blue-intense}{HTML}{0065CA}
    \definecolor{ansi-magenta}{HTML}{D160C4}
    \definecolor{ansi-magenta-intense}{HTML}{A03196}
    \definecolor{ansi-cyan}{HTML}{60C6C8}
    \definecolor{ansi-cyan-intense}{HTML}{258F8F}
    \definecolor{ansi-white}{HTML}{C5C1B4}
    \definecolor{ansi-white-intense}{HTML}{A1A6B2}

    % commands and environments needed by pandoc snippets
    % extracted from the output of `pandoc -s`
    \providecommand{\tightlist}{%
      \setlength{\itemsep}{0pt}\setlength{\parskip}{0pt}}
    \DefineVerbatimEnvironment{Highlighting}{Verbatim}{commandchars=\\\{\}}
    % Add ',fontsize=\small' for more characters per line
    \newenvironment{Shaded}{}{}
    \newcommand{\KeywordTok}[1]{\textcolor[rgb]{0.00,0.44,0.13}{\textbf{{#1}}}}
    \newcommand{\DataTypeTok}[1]{\textcolor[rgb]{0.56,0.13,0.00}{{#1}}}
    \newcommand{\DecValTok}[1]{\textcolor[rgb]{0.25,0.63,0.44}{{#1}}}
    \newcommand{\BaseNTok}[1]{\textcolor[rgb]{0.25,0.63,0.44}{{#1}}}
    \newcommand{\FloatTok}[1]{\textcolor[rgb]{0.25,0.63,0.44}{{#1}}}
    \newcommand{\CharTok}[1]{\textcolor[rgb]{0.25,0.44,0.63}{{#1}}}
    \newcommand{\StringTok}[1]{\textcolor[rgb]{0.25,0.44,0.63}{{#1}}}
    \newcommand{\CommentTok}[1]{\textcolor[rgb]{0.38,0.63,0.69}{\textit{{#1}}}}
    \newcommand{\OtherTok}[1]{\textcolor[rgb]{0.00,0.44,0.13}{{#1}}}
    \newcommand{\AlertTok}[1]{\textcolor[rgb]{1.00,0.00,0.00}{\textbf{{#1}}}}
    \newcommand{\FunctionTok}[1]{\textcolor[rgb]{0.02,0.16,0.49}{{#1}}}
    \newcommand{\RegionMarkerTok}[1]{{#1}}
    \newcommand{\ErrorTok}[1]{\textcolor[rgb]{1.00,0.00,0.00}{\textbf{{#1}}}}
    \newcommand{\NormalTok}[1]{{#1}}
    
    % Additional commands for more recent versions of Pandoc
    \newcommand{\ConstantTok}[1]{\textcolor[rgb]{0.53,0.00,0.00}{{#1}}}
    \newcommand{\SpecialCharTok}[1]{\textcolor[rgb]{0.25,0.44,0.63}{{#1}}}
    \newcommand{\VerbatimStringTok}[1]{\textcolor[rgb]{0.25,0.44,0.63}{{#1}}}
    \newcommand{\SpecialStringTok}[1]{\textcolor[rgb]{0.73,0.40,0.53}{{#1}}}
    \newcommand{\ImportTok}[1]{{#1}}
    \newcommand{\DocumentationTok}[1]{\textcolor[rgb]{0.73,0.13,0.13}{\textit{{#1}}}}
    \newcommand{\AnnotationTok}[1]{\textcolor[rgb]{0.38,0.63,0.69}{\textbf{\textit{{#1}}}}}
    \newcommand{\CommentVarTok}[1]{\textcolor[rgb]{0.38,0.63,0.69}{\textbf{\textit{{#1}}}}}
    \newcommand{\VariableTok}[1]{\textcolor[rgb]{0.10,0.09,0.49}{{#1}}}
    \newcommand{\ControlFlowTok}[1]{\textcolor[rgb]{0.00,0.44,0.13}{\textbf{{#1}}}}
    \newcommand{\OperatorTok}[1]{\textcolor[rgb]{0.40,0.40,0.40}{{#1}}}
    \newcommand{\BuiltInTok}[1]{{#1}}
    \newcommand{\ExtensionTok}[1]{{#1}}
    \newcommand{\PreprocessorTok}[1]{\textcolor[rgb]{0.74,0.48,0.00}{{#1}}}
    \newcommand{\AttributeTok}[1]{\textcolor[rgb]{0.49,0.56,0.16}{{#1}}}
    \newcommand{\InformationTok}[1]{\textcolor[rgb]{0.38,0.63,0.69}{\textbf{\textit{{#1}}}}}
    \newcommand{\WarningTok}[1]{\textcolor[rgb]{0.38,0.63,0.69}{\textbf{\textit{{#1}}}}}
    
    
    % Define a nice break command that doesn't care if a line doesn't already
    % exist.
    \def\br{\hspace*{\fill} \\* }
    % Math Jax compatability definitions
    \def\gt{>}
    \def\lt{<}
    % Document parameters
    \title{Assignment 4}
    \author{Siddharth Nayak EE16B073}
    
    
    

    % Pygments definitions
    
\makeatletter
\def\PY@reset{\let\PY@it=\relax \let\PY@bf=\relax%
    \let\PY@ul=\relax \let\PY@tc=\relax%
    \let\PY@bc=\relax \let\PY@ff=\relax}
\def\PY@tok#1{\csname PY@tok@#1\endcsname}
\def\PY@toks#1+{\ifx\relax#1\empty\else%
    \PY@tok{#1}\expandafter\PY@toks\fi}
\def\PY@do#1{\PY@bc{\PY@tc{\PY@ul{%
    \PY@it{\PY@bf{\PY@ff{#1}}}}}}}
\def\PY#1#2{\PY@reset\PY@toks#1+\relax+\PY@do{#2}}

\expandafter\def\csname PY@tok@w\endcsname{\def\PY@tc##1{\textcolor[rgb]{0.73,0.73,0.73}{##1}}}
\expandafter\def\csname PY@tok@c\endcsname{\let\PY@it=\textit\def\PY@tc##1{\textcolor[rgb]{0.25,0.50,0.50}{##1}}}
\expandafter\def\csname PY@tok@cp\endcsname{\def\PY@tc##1{\textcolor[rgb]{0.74,0.48,0.00}{##1}}}
\expandafter\def\csname PY@tok@k\endcsname{\let\PY@bf=\textbf\def\PY@tc##1{\textcolor[rgb]{0.00,0.50,0.00}{##1}}}
\expandafter\def\csname PY@tok@kp\endcsname{\def\PY@tc##1{\textcolor[rgb]{0.00,0.50,0.00}{##1}}}
\expandafter\def\csname PY@tok@kt\endcsname{\def\PY@tc##1{\textcolor[rgb]{0.69,0.00,0.25}{##1}}}
\expandafter\def\csname PY@tok@o\endcsname{\def\PY@tc##1{\textcolor[rgb]{0.40,0.40,0.40}{##1}}}
\expandafter\def\csname PY@tok@ow\endcsname{\let\PY@bf=\textbf\def\PY@tc##1{\textcolor[rgb]{0.67,0.13,1.00}{##1}}}
\expandafter\def\csname PY@tok@nb\endcsname{\def\PY@tc##1{\textcolor[rgb]{0.00,0.50,0.00}{##1}}}
\expandafter\def\csname PY@tok@nf\endcsname{\def\PY@tc##1{\textcolor[rgb]{0.00,0.00,1.00}{##1}}}
\expandafter\def\csname PY@tok@nc\endcsname{\let\PY@bf=\textbf\def\PY@tc##1{\textcolor[rgb]{0.00,0.00,1.00}{##1}}}
\expandafter\def\csname PY@tok@nn\endcsname{\let\PY@bf=\textbf\def\PY@tc##1{\textcolor[rgb]{0.00,0.00,1.00}{##1}}}
\expandafter\def\csname PY@tok@ne\endcsname{\let\PY@bf=\textbf\def\PY@tc##1{\textcolor[rgb]{0.82,0.25,0.23}{##1}}}
\expandafter\def\csname PY@tok@nv\endcsname{\def\PY@tc##1{\textcolor[rgb]{0.10,0.09,0.49}{##1}}}
\expandafter\def\csname PY@tok@no\endcsname{\def\PY@tc##1{\textcolor[rgb]{0.53,0.00,0.00}{##1}}}
\expandafter\def\csname PY@tok@nl\endcsname{\def\PY@tc##1{\textcolor[rgb]{0.63,0.63,0.00}{##1}}}
\expandafter\def\csname PY@tok@ni\endcsname{\let\PY@bf=\textbf\def\PY@tc##1{\textcolor[rgb]{0.60,0.60,0.60}{##1}}}
\expandafter\def\csname PY@tok@na\endcsname{\def\PY@tc##1{\textcolor[rgb]{0.49,0.56,0.16}{##1}}}
\expandafter\def\csname PY@tok@nt\endcsname{\let\PY@bf=\textbf\def\PY@tc##1{\textcolor[rgb]{0.00,0.50,0.00}{##1}}}
\expandafter\def\csname PY@tok@nd\endcsname{\def\PY@tc##1{\textcolor[rgb]{0.67,0.13,1.00}{##1}}}
\expandafter\def\csname PY@tok@s\endcsname{\def\PY@tc##1{\textcolor[rgb]{0.73,0.13,0.13}{##1}}}
\expandafter\def\csname PY@tok@sd\endcsname{\let\PY@it=\textit\def\PY@tc##1{\textcolor[rgb]{0.73,0.13,0.13}{##1}}}
\expandafter\def\csname PY@tok@si\endcsname{\let\PY@bf=\textbf\def\PY@tc##1{\textcolor[rgb]{0.73,0.40,0.53}{##1}}}
\expandafter\def\csname PY@tok@se\endcsname{\let\PY@bf=\textbf\def\PY@tc##1{\textcolor[rgb]{0.73,0.40,0.13}{##1}}}
\expandafter\def\csname PY@tok@sr\endcsname{\def\PY@tc##1{\textcolor[rgb]{0.73,0.40,0.53}{##1}}}
\expandafter\def\csname PY@tok@ss\endcsname{\def\PY@tc##1{\textcolor[rgb]{0.10,0.09,0.49}{##1}}}
\expandafter\def\csname PY@tok@sx\endcsname{\def\PY@tc##1{\textcolor[rgb]{0.00,0.50,0.00}{##1}}}
\expandafter\def\csname PY@tok@m\endcsname{\def\PY@tc##1{\textcolor[rgb]{0.40,0.40,0.40}{##1}}}
\expandafter\def\csname PY@tok@gh\endcsname{\let\PY@bf=\textbf\def\PY@tc##1{\textcolor[rgb]{0.00,0.00,0.50}{##1}}}
\expandafter\def\csname PY@tok@gu\endcsname{\let\PY@bf=\textbf\def\PY@tc##1{\textcolor[rgb]{0.50,0.00,0.50}{##1}}}
\expandafter\def\csname PY@tok@gd\endcsname{\def\PY@tc##1{\textcolor[rgb]{0.63,0.00,0.00}{##1}}}
\expandafter\def\csname PY@tok@gi\endcsname{\def\PY@tc##1{\textcolor[rgb]{0.00,0.63,0.00}{##1}}}
\expandafter\def\csname PY@tok@gr\endcsname{\def\PY@tc##1{\textcolor[rgb]{1.00,0.00,0.00}{##1}}}
\expandafter\def\csname PY@tok@ge\endcsname{\let\PY@it=\textit}
\expandafter\def\csname PY@tok@gs\endcsname{\let\PY@bf=\textbf}
\expandafter\def\csname PY@tok@gp\endcsname{\let\PY@bf=\textbf\def\PY@tc##1{\textcolor[rgb]{0.00,0.00,0.50}{##1}}}
\expandafter\def\csname PY@tok@go\endcsname{\def\PY@tc##1{\textcolor[rgb]{0.53,0.53,0.53}{##1}}}
\expandafter\def\csname PY@tok@gt\endcsname{\def\PY@tc##1{\textcolor[rgb]{0.00,0.27,0.87}{##1}}}
\expandafter\def\csname PY@tok@err\endcsname{\def\PY@bc##1{\setlength{\fboxsep}{0pt}\fcolorbox[rgb]{1.00,0.00,0.00}{1,1,1}{\strut ##1}}}
\expandafter\def\csname PY@tok@kc\endcsname{\let\PY@bf=\textbf\def\PY@tc##1{\textcolor[rgb]{0.00,0.50,0.00}{##1}}}
\expandafter\def\csname PY@tok@kd\endcsname{\let\PY@bf=\textbf\def\PY@tc##1{\textcolor[rgb]{0.00,0.50,0.00}{##1}}}
\expandafter\def\csname PY@tok@kn\endcsname{\let\PY@bf=\textbf\def\PY@tc##1{\textcolor[rgb]{0.00,0.50,0.00}{##1}}}
\expandafter\def\csname PY@tok@kr\endcsname{\let\PY@bf=\textbf\def\PY@tc##1{\textcolor[rgb]{0.00,0.50,0.00}{##1}}}
\expandafter\def\csname PY@tok@bp\endcsname{\def\PY@tc##1{\textcolor[rgb]{0.00,0.50,0.00}{##1}}}
\expandafter\def\csname PY@tok@fm\endcsname{\def\PY@tc##1{\textcolor[rgb]{0.00,0.00,1.00}{##1}}}
\expandafter\def\csname PY@tok@vc\endcsname{\def\PY@tc##1{\textcolor[rgb]{0.10,0.09,0.49}{##1}}}
\expandafter\def\csname PY@tok@vg\endcsname{\def\PY@tc##1{\textcolor[rgb]{0.10,0.09,0.49}{##1}}}
\expandafter\def\csname PY@tok@vi\endcsname{\def\PY@tc##1{\textcolor[rgb]{0.10,0.09,0.49}{##1}}}
\expandafter\def\csname PY@tok@vm\endcsname{\def\PY@tc##1{\textcolor[rgb]{0.10,0.09,0.49}{##1}}}
\expandafter\def\csname PY@tok@sa\endcsname{\def\PY@tc##1{\textcolor[rgb]{0.73,0.13,0.13}{##1}}}
\expandafter\def\csname PY@tok@sb\endcsname{\def\PY@tc##1{\textcolor[rgb]{0.73,0.13,0.13}{##1}}}
\expandafter\def\csname PY@tok@sc\endcsname{\def\PY@tc##1{\textcolor[rgb]{0.73,0.13,0.13}{##1}}}
\expandafter\def\csname PY@tok@dl\endcsname{\def\PY@tc##1{\textcolor[rgb]{0.73,0.13,0.13}{##1}}}
\expandafter\def\csname PY@tok@s2\endcsname{\def\PY@tc##1{\textcolor[rgb]{0.73,0.13,0.13}{##1}}}
\expandafter\def\csname PY@tok@sh\endcsname{\def\PY@tc##1{\textcolor[rgb]{0.73,0.13,0.13}{##1}}}
\expandafter\def\csname PY@tok@s1\endcsname{\def\PY@tc##1{\textcolor[rgb]{0.73,0.13,0.13}{##1}}}
\expandafter\def\csname PY@tok@mb\endcsname{\def\PY@tc##1{\textcolor[rgb]{0.40,0.40,0.40}{##1}}}
\expandafter\def\csname PY@tok@mf\endcsname{\def\PY@tc##1{\textcolor[rgb]{0.40,0.40,0.40}{##1}}}
\expandafter\def\csname PY@tok@mh\endcsname{\def\PY@tc##1{\textcolor[rgb]{0.40,0.40,0.40}{##1}}}
\expandafter\def\csname PY@tok@mi\endcsname{\def\PY@tc##1{\textcolor[rgb]{0.40,0.40,0.40}{##1}}}
\expandafter\def\csname PY@tok@il\endcsname{\def\PY@tc##1{\textcolor[rgb]{0.40,0.40,0.40}{##1}}}
\expandafter\def\csname PY@tok@mo\endcsname{\def\PY@tc##1{\textcolor[rgb]{0.40,0.40,0.40}{##1}}}
\expandafter\def\csname PY@tok@ch\endcsname{\let\PY@it=\textit\def\PY@tc##1{\textcolor[rgb]{0.25,0.50,0.50}{##1}}}
\expandafter\def\csname PY@tok@cm\endcsname{\let\PY@it=\textit\def\PY@tc##1{\textcolor[rgb]{0.25,0.50,0.50}{##1}}}
\expandafter\def\csname PY@tok@cpf\endcsname{\let\PY@it=\textit\def\PY@tc##1{\textcolor[rgb]{0.25,0.50,0.50}{##1}}}
\expandafter\def\csname PY@tok@c1\endcsname{\let\PY@it=\textit\def\PY@tc##1{\textcolor[rgb]{0.25,0.50,0.50}{##1}}}
\expandafter\def\csname PY@tok@cs\endcsname{\let\PY@it=\textit\def\PY@tc##1{\textcolor[rgb]{0.25,0.50,0.50}{##1}}}

\def\PYZbs{\char`\\}
\def\PYZus{\char`\_}
\def\PYZob{\char`\{}
\def\PYZcb{\char`\}}
\def\PYZca{\char`\^}
\def\PYZam{\char`\&}
\def\PYZlt{\char`\<}
\def\PYZgt{\char`\>}
\def\PYZsh{\char`\#}
\def\PYZpc{\char`\%}
\def\PYZdl{\char`\$}
\def\PYZhy{\char`\-}
\def\PYZsq{\char`\'}
\def\PYZdq{\char`\"}
\def\PYZti{\char`\~}
% for compatibility with earlier versions
\def\PYZat{@}
\def\PYZlb{[}
\def\PYZrb{]}
\makeatother


    % Exact colors from NB
    \definecolor{incolor}{rgb}{0.0, 0.0, 0.5}
    \definecolor{outcolor}{rgb}{0.545, 0.0, 0.0}



    
    % Prevent overflowing lines due to hard-to-break entities
    \sloppy 
    % Setup hyperref package
    \hypersetup{
      breaklinks=true,  % so long urls are correctly broken across lines
      colorlinks=true,
      urlcolor=urlcolor,
      linkcolor=linkcolor,
      citecolor=citecolor,
      }
    % Slightly bigger margins than the latex defaults
    
    \geometry{verbose,tmargin=1in,bmargin=1in,lmargin=1in,rmargin=1in}
    
    

    \begin{document}
    
    
    \maketitle
    
    

    
    \hypertarget{introduction}{%
\section{Introduction}\label{introduction}}

Least Squares method is a very useful method in approximating functions.
They can be used to approximate signals with noise to a very good extent
as far as the noise is not much.

    \hypertarget{least-squares-method}{%
\section{Least Squares Method}\label{least-squares-method}}

The coefficients can also be obtained by this method.This method
basically fits the variables in a set of equations where the error which
is squared is minimized. \\
\subsection{Derivation of Least Squares Method using linear algebra: }
 \\We want to find the solution for Ax=b \\But due to more
number of equations than the number of variables we may not get a
perfect solution and hence we find a solution \(x\) which minimizes the
error \(\epsilon\)=\(b-Ax\).

      
    \begin{center}
    \adjustimage{max size={0.9\linewidth}{0.9\paperheight}}{output_2_0.png}
    \end{center}
    { \hspace*{\fill} \\}
    

    In the figure above \(b-Ax\) is minimum when it is perpendicular to
\(Ax\). \\Thus, \(A^T.(b-Ax)=0\) \(\longrightarrow\) \(A\) is
perpendicular to \(b-Ax\)\\ \(A^T.b=(A^T.A).x\)\\
\(\therefore x=(A^T.A)^{-1}.A^T.b\)

    \hypertarget{importing-libraries}{%
\section{Importing Libraries}\label{importing-libraries}}

    \begin{Verbatim}[commandchars=\\\{\}]
{\color{incolor}In [{\color{incolor}2}]:} \PY{k+kn}{import} \PY{n+nn}{numpy} \PY{k}{as} \PY{n+nn}{np}
        \PY{k+kn}{import} \PY{n+nn}{math}
        \PY{k+kn}{import} \PY{n+nn}{matplotlib}\PY{n+nn}{.}\PY{n+nn}{pyplot} \PY{k}{as} \PY{n+nn}{plt}
        \PY{k+kn}{import} \PY{n+nn}{scipy}\PY{n+nn}{.}\PY{n+nn}{special} \PY{k}{as} \PY{n+nn}{sp}
\end{Verbatim}


    \begin{Verbatim}[commandchars=\\\{\}]
{\color{incolor}In [{\color{incolor}3}]:} \PY{n}{x}\PY{o}{=}\PY{n}{np}\PY{o}{.}\PY{n}{linspace}\PY{p}{(}\PY{l+m+mf}{0.5}\PY{p}{,}\PY{l+m+mi}{20}\PY{p}{,}\PY{l+m+mi}{40}\PY{p}{)}
\end{Verbatim}


    \hypertarget{define-a-function-for-getting-acosxbsinxapprox-j_vx}{%
\section{\texorpdfstring{Define a function for getting
\(Acos(x)+Bsin(x)\approx J_v(x)\)}{Define a function for getting Acos(x)+Bsin(x)\textbackslash{}approx J\_v(x)}}\label{define-a-function-for-getting-acosxbsinxapprox-j_vx}}

    \begin{Verbatim}[commandchars=\\\{\}]
{\color{incolor}In [{\color{incolor}4}]:} \PY{k}{def} \PY{n+nf}{method\PYZus{}1}\PY{p}{(}\PY{n}{x}\PY{p}{,}\PY{n}{x0}\PY{p}{)}\PY{p}{:}
            \PY{n}{A\PYZus{}arr}\PY{o}{=}\PY{p}{[}\PY{p}{]}
            \PY{n}{x\PYZus{}new}\PY{o}{=}\PY{n}{x}\PY{p}{[}\PY{n+nb}{int}\PY{p}{(}\PY{n}{np}\PY{o}{.}\PY{n}{where}\PY{p}{(}\PY{n}{x}\PY{o}{==}\PY{n}{x0}\PY{p}{)}\PY{p}{[}\PY{l+m+mi}{0}\PY{p}{]}\PY{p}{)}\PY{p}{:}\PY{p}{]}
            \PY{n}{J}\PY{o}{=}\PY{n}{sp}\PY{o}{.}\PY{n}{jv}\PY{p}{(}\PY{l+m+mi}{1}\PY{p}{,}\PY{n}{x\PYZus{}new}\PY{p}{)}
            \PY{k}{for} \PY{n}{element} \PY{o+ow}{in} \PY{n}{x\PYZus{}new}\PY{p}{:}
                \PY{n}{cos}\PY{o}{=}\PY{n}{math}\PY{o}{.}\PY{n}{cos}\PY{p}{(}\PY{n}{element}\PY{p}{)}
                \PY{n}{sin}\PY{o}{=}\PY{n}{math}\PY{o}{.}\PY{n}{sin}\PY{p}{(}\PY{n}{element}\PY{p}{)}
                \PY{n}{A\PYZus{}arr}\PY{o}{.}\PY{n}{append}\PY{p}{(}\PY{p}{[}\PY{n}{cos}\PY{p}{,}\PY{n}{sin}\PY{p}{]}\PY{p}{)}
            \PY{k}{return} \PY{n}{A\PYZus{}arr}\PY{p}{,}\PY{n}{J}
\end{Verbatim}


    \begin{Verbatim}[commandchars=\\\{\}]
{\color{incolor}In [{\color{incolor}5}]:} \PY{c+c1}{\PYZsh{}A,J=method\PYZus{}1(x,0.5)}
        \PY{n}{AB1\PYZus{}array}\PY{o}{=}\PY{p}{[}\PY{p}{]}  \PY{c+c1}{\PYZsh{} will contain the A and B values}
        \PY{k}{for} \PY{n}{x\PYZus{}0} \PY{o+ow}{in} \PY{n}{x}\PY{p}{:}
            \PY{k}{if} \PY{n}{x\PYZus{}0}\PY{o}{\PYZlt{}}\PY{o}{=}\PY{l+m+mi}{18}\PY{p}{:}
                \PY{n}{A}\PY{p}{,}\PY{n}{J}\PY{o}{=}\PY{n}{method\PYZus{}1}\PY{p}{(}\PY{n}{x}\PY{p}{,}\PY{n}{x\PYZus{}0}\PY{p}{)}
                \PY{n}{s}\PY{o}{=}\PY{n}{np}\PY{o}{.}\PY{n}{linalg}\PY{o}{.}\PY{n}{inv}\PY{p}{(}\PY{n}{np}\PY{o}{.}\PY{n}{dot}\PY{p}{(}\PY{n}{np}\PY{o}{.}\PY{n}{array}\PY{p}{(}\PY{n}{A}\PY{p}{)}\PY{o}{.}\PY{n}{T}\PY{p}{,}\PY{n}{A}\PY{p}{)}\PY{p}{)}
                \PY{n}{cl}\PY{o}{=}\PY{p}{(}\PY{n}{np}\PY{o}{.}\PY{n}{dot}\PY{p}{(}\PY{n}{s}\PY{p}{,}\PY{n}{np}\PY{o}{.}\PY{n}{dot}\PY{p}{(}\PY{n}{np}\PY{o}{.}\PY{n}{array}\PY{p}{(}\PY{n}{A}\PY{p}{)}\PY{o}{.}\PY{n}{T}\PY{p}{,}\PY{n}{J}\PY{p}{)}\PY{p}{)}\PY{p}{)}
                \PY{c+c1}{\PYZsh{}cl=np.linalg.lstsq(A,J)[0] }
                \PY{n}{AB1\PYZus{}array}\PY{o}{.}\PY{n}{append}\PY{p}{(}\PY{n}{cl}\PY{p}{)}
\end{Verbatim}


    \hypertarget{function-for-calculating-the-value-of-v-from-phi}{%
\section{\texorpdfstring{Function for calculating the value of \(v\)
from
\(\phi\)}{Function for calculating the value of v from \textbackslash{}phi}}\label{function-for-calculating-the-value-of-v-from-phi}}

We have \(Acos(x)+Bsin(x)\)\\ Dividing and multiplying by
\(\sqrt{A^{2}+B^{2}}\) We get
\(\frac{Acos(x)}{\sqrt{A^{2}+B^{2}}}+\frac{Bsin(x)}{\sqrt{A^{2}+B^{2}}}\)\\
Therefore we get \(\sqrt{A^{2}+B^{2}}{cos(x+\phi)}\) where
\(\phi=\arccos(\frac{A}{\sqrt{A^{2}+B^{2}}})\)

    \begin{Verbatim}[commandchars=\\\{\}]
{\color{incolor}In [{\color{incolor}6}]:} \PY{k}{def} \PY{n+nf}{calculate\PYZus{}v}\PY{p}{(}\PY{n}{AB\PYZus{}array}\PY{p}{)}\PY{p}{:}
            \PY{n}{v\PYZus{}arr}\PY{o}{=}\PY{p}{[}\PY{p}{]}
            \PY{k}{for} \PY{n}{elements} \PY{o+ow}{in} \PY{n}{AB\PYZus{}array}\PY{p}{:}
                \PY{n}{phi}\PY{o}{=}\PY{n}{np}\PY{o}{.}\PY{n}{arccos}\PY{p}{(}\PY{n}{elements}\PY{p}{[}\PY{l+m+mi}{0}\PY{p}{]}\PY{o}{/}\PY{n}{math}\PY{o}{.}\PY{n}{sqrt}\PY{p}{(}\PY{n}{elements}\PY{p}{[}\PY{l+m+mi}{0}\PY{p}{]}\PY{o}{*}\PY{o}{*}\PY{l+m+mi}{2}\PY{o}{+}\PY{n}{elements}\PY{p}{[}\PY{l+m+mi}{1}\PY{p}{]}\PY{o}{*}\PY{o}{*}\PY{l+m+mi}{2}\PY{p}{)}\PY{p}{)}
                \PY{n}{v}\PY{o}{=}\PY{p}{(}\PY{n}{phi}\PY{o}{\PYZhy{}}\PY{n}{np}\PY{o}{.}\PY{n}{pi}\PY{o}{/}\PY{l+m+mi}{4}\PY{p}{)}\PY{o}{*}\PY{l+m+mi}{2}\PY{o}{/}\PY{n}{np}\PY{o}{.}\PY{n}{pi}
                \PY{n}{v\PYZus{}arr}\PY{o}{.}\PY{n}{append}\PY{p}{(}\PY{n}{v}\PY{p}{)}
            \PY{k}{return} \PY{n}{v\PYZus{}arr}    
\end{Verbatim}


    \begin{Verbatim}[commandchars=\\\{\}]
{\color{incolor}In [{\color{incolor}7}]:} \PY{n}{v\PYZus{}arr1}\PY{o}{=}\PY{n}{calculate\PYZus{}v}\PY{p}{(}\PY{n}{AB1\PYZus{}array}\PY{p}{)} \PY{c+c1}{\PYZsh{} has values of v for different x0}
\end{Verbatim}


    \begin{Verbatim}[commandchars=\\\{\}]
{\color{incolor}In [{\color{incolor}8}]:} \PY{n}{x\PYZus{}0\PYZus{}arr}\PY{o}{=}\PY{n}{np}\PY{o}{.}\PY{n}{linspace}\PY{p}{(}\PY{l+m+mf}{0.5}\PY{p}{,}\PY{l+m+mi}{18}\PY{p}{,}\PY{l+m+mi}{36}\PY{p}{)}
\end{Verbatim}


    \hypertarget{define-a-function-for-getting-fracacosxsqrtxfracbsinxsqrtxapprox-j_vx}{%
\section{\texorpdfstring{Define a function for getting
\(\frac{Acos(x)}{\sqrt{x}}+\frac{Bsin(x)}{\sqrt{x}}\approx J_v(x)\)}{Define a function for getting \textbackslash{}frac\{Acos(x)\}\{\textbackslash{}sqrt\{x\}\}+\textbackslash{}frac\{Bsin(x)\}\{\textbackslash{}sqrt\{x\}\}\textbackslash{}approx J\_v(x)}}\label{define-a-function-for-getting-fracacosxsqrtxfracbsinxsqrtxapprox-j_vx}}

    \begin{Verbatim}[commandchars=\\\{\}]
{\color{incolor}In [{\color{incolor}9}]:} \PY{k}{def} \PY{n+nf}{method\PYZus{}2}\PY{p}{(}\PY{n}{x}\PY{p}{,}\PY{n}{x0}\PY{p}{)}\PY{p}{:} 
            \PY{l+s+sd}{\PYZsq{}\PYZsq{}\PYZsq{}}
        \PY{l+s+sd}{    returns the arrays A and J in the equation Ax=J}
        \PY{l+s+sd}{    \PYZsq{}\PYZsq{}\PYZsq{}}
            \PY{n}{A\PYZus{}arr}\PY{o}{=}\PY{p}{[}\PY{p}{]}
            \PY{n}{x\PYZus{}new}\PY{o}{=}\PY{n}{x}\PY{p}{[}\PY{n+nb}{int}\PY{p}{(}\PY{n}{np}\PY{o}{.}\PY{n}{where}\PY{p}{(}\PY{n}{x}\PY{o}{==}\PY{n}{x0}\PY{p}{)}\PY{p}{[}\PY{l+m+mi}{0}\PY{p}{]}\PY{p}{)}\PY{p}{:}\PY{p}{]}
            \PY{n}{J}\PY{o}{=}\PY{n}{sp}\PY{o}{.}\PY{n}{jv}\PY{p}{(}\PY{l+m+mi}{1}\PY{p}{,}\PY{n}{x\PYZus{}new}\PY{p}{)}
            \PY{k}{for} \PY{n}{element} \PY{o+ow}{in} \PY{n}{x\PYZus{}new}\PY{p}{:}
                \PY{n}{cos}\PY{o}{=}\PY{n}{math}\PY{o}{.}\PY{n}{cos}\PY{p}{(}\PY{n}{element}\PY{p}{)}\PY{o}{/}\PY{n}{math}\PY{o}{.}\PY{n}{sqrt}\PY{p}{(}\PY{n}{element}\PY{p}{)}
                \PY{n}{sin}\PY{o}{=}\PY{n}{math}\PY{o}{.}\PY{n}{sin}\PY{p}{(}\PY{n}{element}\PY{p}{)}\PY{o}{/}\PY{n}{math}\PY{o}{.}\PY{n}{sqrt}\PY{p}{(}\PY{n}{element}\PY{p}{)}
                \PY{n}{A\PYZus{}arr}\PY{o}{.}\PY{n}{append}\PY{p}{(}\PY{p}{[}\PY{n}{cos}\PY{p}{,}\PY{n}{sin}\PY{p}{]}\PY{p}{)}
            \PY{k}{return} \PY{n}{A\PYZus{}arr}\PY{p}{,}\PY{n}{J}
\end{Verbatim}


    \begin{Verbatim}[commandchars=\\\{\}]
{\color{incolor}In [{\color{incolor}10}]:} \PY{n}{AB2\PYZus{}array}\PY{o}{=}\PY{p}{[}\PY{p}{]}
         \PY{k}{for} \PY{n}{x\PYZus{}0} \PY{o+ow}{in} \PY{n}{x}\PY{p}{:}
             \PY{k}{if} \PY{n}{x\PYZus{}0}\PY{o}{\PYZlt{}}\PY{o}{=}\PY{l+m+mi}{18}\PY{p}{:}
                 \PY{n}{A}\PY{p}{,}\PY{n}{J}\PY{o}{=}\PY{n}{method\PYZus{}2}\PY{p}{(}\PY{n}{x}\PY{p}{,}\PY{n}{x\PYZus{}0}\PY{p}{)}
                 \PY{n}{A}\PY{o}{=}\PY{n}{np}\PY{o}{.}\PY{n}{array}\PY{p}{(}\PY{n}{A}\PY{p}{)}
                 \PY{n}{s}\PY{o}{=}\PY{n}{np}\PY{o}{.}\PY{n}{linalg}\PY{o}{.}\PY{n}{inv}\PY{p}{(}\PY{n}{np}\PY{o}{.}\PY{n}{dot}\PY{p}{(}\PY{n}{A}\PY{o}{.}\PY{n}{T}\PY{p}{,}\PY{n}{A}\PY{p}{)}\PY{p}{)}
                 \PY{n}{cl}\PY{o}{=}\PY{p}{(}\PY{n}{np}\PY{o}{.}\PY{n}{dot}\PY{p}{(}\PY{n}{s}\PY{p}{,}\PY{n}{np}\PY{o}{.}\PY{n}{dot}\PY{p}{(}\PY{n}{A}\PY{o}{.}\PY{n}{T}\PY{p}{,}\PY{n}{J}\PY{p}{)}\PY{p}{)}\PY{p}{)}
                 \PY{n}{AB2\PYZus{}array}\PY{o}{.}\PY{n}{append}\PY{p}{(}\PY{n}{cl}\PY{p}{)}
\end{Verbatim}


    \hypertarget{plot-of-function-obtained-by-method-1}{%
\subsection{Plot of function obtained by method
1}\label{plot-of-function-obtained-by-method-1}}

The amplitudes do not match because in method 1 we do not take it into
account.

    \begin{Verbatim}[commandchars=\\\{\}]
{\color{incolor}In [{\color{incolor}11}]:} \PY{n}{A\PYZus{}arr}\PY{o}{=}\PY{p}{[}\PY{p}{]}
         \PY{k}{for} \PY{n}{element} \PY{o+ow}{in} \PY{n}{x}\PY{p}{:}
             \PY{n}{cos}\PY{o}{=}\PY{n}{math}\PY{o}{.}\PY{n}{cos}\PY{p}{(}\PY{n}{element}\PY{p}{)}
             \PY{n}{sin}\PY{o}{=}\PY{n}{math}\PY{o}{.}\PY{n}{sin}\PY{p}{(}\PY{n}{element}\PY{p}{)}
             \PY{n}{A\PYZus{}arr}\PY{o}{.}\PY{n}{append}\PY{p}{(}\PY{p}{[}\PY{n}{cos}\PY{p}{,}\PY{n}{sin}\PY{p}{]}\PY{p}{)}
         \PY{n}{J}\PY{o}{=}\PY{n}{sp}\PY{o}{.}\PY{n}{jv}\PY{p}{(}\PY{l+m+mi}{1}\PY{p}{,}\PY{n}{x}\PY{p}{)}    
         \PY{n}{cl}\PY{o}{=}\PY{n}{np}\PY{o}{.}\PY{n}{linalg}\PY{o}{.}\PY{n}{lstsq}\PY{p}{(}\PY{n}{A\PYZus{}arr}\PY{p}{,}\PY{n}{J}\PY{p}{)}\PY{p}{[}\PY{l+m+mi}{0}\PY{p}{]}
         \PY{n}{a}\PY{o}{=}\PY{n}{cl}\PY{p}{[}\PY{l+m+mi}{0}\PY{p}{]}\PY{o}{*}\PY{n}{np}\PY{o}{.}\PY{n}{cos}\PY{p}{(}\PY{n}{x}\PY{p}{)}
         \PY{n}{b}\PY{o}{=}\PY{n}{cl}\PY{p}{[}\PY{l+m+mi}{1}\PY{p}{]}\PY{o}{*}\PY{n}{np}\PY{o}{.}\PY{n}{sin}\PY{p}{(}\PY{n}{x}\PY{p}{)}
         \PY{n}{c}\PY{o}{=}\PY{n}{a}\PY{o}{+}\PY{n}{b}
         \PY{n}{plt}\PY{o}{.}\PY{n}{plot}\PY{p}{(}\PY{n}{x}\PY{p}{,}\PY{n}{sp}\PY{o}{.}\PY{n}{jv}\PY{p}{(}\PY{l+m+mi}{1}\PY{p}{,}\PY{n}{x}\PY{p}{)}\PY{p}{,}\PY{l+s+s1}{\PYZsq{}}\PY{l+s+s1}{ro}\PY{l+s+s1}{\PYZsq{}}\PY{p}{,}\PY{n}{label}\PY{o}{=}\PY{l+s+s1}{\PYZsq{}}\PY{l+s+s1}{Jv}\PY{l+s+s1}{\PYZsq{}}\PY{p}{)}
         \PY{n}{plt}\PY{o}{.}\PY{n}{plot}\PY{p}{(}\PY{n}{x}\PY{p}{,}\PY{n}{c}\PY{p}{,}\PY{l+s+s1}{\PYZsq{}}\PY{l+s+s1}{gx}\PY{l+s+s1}{\PYZsq{}}\PY{p}{,}\PY{n}{label}\PY{o}{=}\PY{l+s+s1}{\PYZsq{}}\PY{l+s+s1}{lstsq}\PY{l+s+s1}{\PYZsq{}}\PY{p}{)} \PY{c+c1}{\PYZsh{}obtained function by method 1}
         \PY{n}{plt}\PY{o}{.}\PY{n}{title}\PY{p}{(}\PY{l+s+s1}{\PYZsq{}}\PY{l+s+s1}{Plot of function obtained by method 1}\PY{l+s+s1}{\PYZsq{}}\PY{p}{)}
         \PY{n}{plt}\PY{o}{.}\PY{n}{xlabel}\PY{p}{(}\PY{l+s+s1}{\PYZsq{}}\PY{l+s+s1}{\PYZdl{}x\PYZdl{}}\PY{l+s+s1}{\PYZsq{}}\PY{p}{)}
         \PY{n}{plt}\PY{o}{.}\PY{n}{ylabel}\PY{p}{(}\PY{l+s+s1}{\PYZsq{}}\PY{l+s+s1}{\PYZdl{}J\PYZus{}v(x)\PYZdl{}}\PY{l+s+s1}{\PYZsq{}}\PY{p}{)}
         \PY{n}{plt}\PY{o}{.}\PY{n}{legend}\PY{p}{(}\PY{p}{)}
         \PY{n}{plt}\PY{o}{.}\PY{n}{grid}\PY{p}{(}\PY{p}{)}
         \PY{n}{plt}\PY{o}{.}\PY{n}{show}\PY{p}{(}\PY{p}{)}
\end{Verbatim}


    \begin{center}
    \adjustimage{max size={0.9\linewidth}{0.9\paperheight}}{output_18_0.png}
    \end{center}
    { \hspace*{\fill} \\}
    
    \hypertarget{plot-of-function-obtained-by-method-2}{%
\subsection{Plot of function obtained by method
2}\label{plot-of-function-obtained-by-method-2}}

The plot matches quite nicely and gives us a very good approximate of
the Bessel's function.

    \begin{Verbatim}[commandchars=\\\{\}]
{\color{incolor}In [{\color{incolor}12}]:} \PY{n}{A\PYZus{}arr}\PY{o}{=}\PY{p}{[}\PY{p}{]}
         \PY{k}{for} \PY{n}{element} \PY{o+ow}{in} \PY{n}{x}\PY{p}{:}
             \PY{n}{cos}\PY{o}{=}\PY{n}{math}\PY{o}{.}\PY{n}{cos}\PY{p}{(}\PY{n}{element}\PY{p}{)}\PY{o}{/}\PY{n}{math}\PY{o}{.}\PY{n}{sqrt}\PY{p}{(}\PY{n}{element}\PY{p}{)}
             \PY{n}{sin}\PY{o}{=}\PY{n}{math}\PY{o}{.}\PY{n}{sin}\PY{p}{(}\PY{n}{element}\PY{p}{)}\PY{o}{/}\PY{n}{math}\PY{o}{.}\PY{n}{sqrt}\PY{p}{(}\PY{n}{element}\PY{p}{)}
             \PY{n}{A\PYZus{}arr}\PY{o}{.}\PY{n}{append}\PY{p}{(}\PY{p}{[}\PY{n}{cos}\PY{p}{,}\PY{n}{sin}\PY{p}{]}\PY{p}{)}
         \PY{n}{J}\PY{o}{=}\PY{n}{sp}\PY{o}{.}\PY{n}{jv}\PY{p}{(}\PY{l+m+mi}{1}\PY{p}{,}\PY{n}{x}\PY{p}{)}    
         \PY{n}{cl}\PY{o}{=}\PY{n}{np}\PY{o}{.}\PY{n}{linalg}\PY{o}{.}\PY{n}{lstsq}\PY{p}{(}\PY{n}{A\PYZus{}arr}\PY{p}{,}\PY{n}{J}\PY{p}{)}\PY{p}{[}\PY{l+m+mi}{0}\PY{p}{]}
         \PY{n}{a}\PY{o}{=}\PY{n}{cl}\PY{p}{[}\PY{l+m+mi}{0}\PY{p}{]}\PY{o}{*}\PY{n}{np}\PY{o}{.}\PY{n}{cos}\PY{p}{(}\PY{n}{x}\PY{p}{)}\PY{o}{/}\PY{n}{np}\PY{o}{.}\PY{n}{sqrt}\PY{p}{(}\PY{n}{x}\PY{p}{)}
         \PY{n}{b}\PY{o}{=}\PY{n}{cl}\PY{p}{[}\PY{l+m+mi}{1}\PY{p}{]}\PY{o}{*}\PY{n}{np}\PY{o}{.}\PY{n}{sin}\PY{p}{(}\PY{n}{x}\PY{p}{)}\PY{o}{/}\PY{n}{np}\PY{o}{.}\PY{n}{sqrt}\PY{p}{(}\PY{n}{x}\PY{p}{)}
         \PY{n}{c}\PY{o}{=}\PY{n}{a}\PY{o}{+}\PY{n}{b}
         \PY{n}{plt}\PY{o}{.}\PY{n}{plot}\PY{p}{(}\PY{n}{x}\PY{p}{,}\PY{n}{sp}\PY{o}{.}\PY{n}{jv}\PY{p}{(}\PY{l+m+mi}{1}\PY{p}{,}\PY{n}{x}\PY{p}{)}\PY{p}{,}\PY{l+s+s1}{\PYZsq{}}\PY{l+s+s1}{ro}\PY{l+s+s1}{\PYZsq{}}\PY{p}{,}\PY{n}{label}\PY{o}{=}\PY{l+s+s1}{\PYZsq{}}\PY{l+s+s1}{Jv}\PY{l+s+s1}{\PYZsq{}}\PY{p}{)}
         \PY{n}{plt}\PY{o}{.}\PY{n}{plot}\PY{p}{(}\PY{n}{x}\PY{p}{,}\PY{n}{c}\PY{p}{,}\PY{l+s+s1}{\PYZsq{}}\PY{l+s+s1}{gx}\PY{l+s+s1}{\PYZsq{}}\PY{p}{,}\PY{n}{label}\PY{o}{=}\PY{l+s+s1}{\PYZsq{}}\PY{l+s+s1}{lstsq}\PY{l+s+s1}{\PYZsq{}}\PY{p}{)} \PY{c+c1}{\PYZsh{}obtained function by method 2}
         \PY{n}{plt}\PY{o}{.}\PY{n}{title}\PY{p}{(}\PY{l+s+s1}{\PYZsq{}}\PY{l+s+s1}{Plot of function obtained by method 2}\PY{l+s+s1}{\PYZsq{}}\PY{p}{)}
         \PY{n}{plt}\PY{o}{.}\PY{n}{xlabel}\PY{p}{(}\PY{l+s+s1}{\PYZsq{}}\PY{l+s+s1}{\PYZdl{}x\PYZdl{}}\PY{l+s+s1}{\PYZsq{}}\PY{p}{)}
         \PY{n}{plt}\PY{o}{.}\PY{n}{ylabel}\PY{p}{(}\PY{l+s+s1}{\PYZsq{}}\PY{l+s+s1}{\PYZdl{}J\PYZus{}v(x)\PYZdl{}}\PY{l+s+s1}{\PYZsq{}}\PY{p}{)}
         \PY{n}{plt}\PY{o}{.}\PY{n}{legend}\PY{p}{(}\PY{p}{)}
         \PY{n}{plt}\PY{o}{.}\PY{n}{grid}\PY{p}{(}\PY{p}{)}
         \PY{n}{plt}\PY{o}{.}\PY{n}{show}\PY{p}{(}\PY{p}{)}
\end{Verbatim}


    \begin{center}
    \adjustimage{max size={0.9\linewidth}{0.9\paperheight}}{output_20_0.png}
    \end{center}
    { \hspace*{\fill} \\}
    
    \begin{Verbatim}[commandchars=\\\{\}]
{\color{incolor}In [{\color{incolor}13}]:} \PY{n}{v\PYZus{}arr2}\PY{o}{=}\PY{n}{calculate\PYZus{}v}\PY{p}{(}\PY{n}{AB2\PYZus{}array}\PY{p}{)} \PY{c+c1}{\PYZsh{} has values of v for different x0}
\end{Verbatim}


    \hypertarget{plot-of-v-vs-x_0}{%
\subsection{\texorpdfstring{Plot of \(v\) vs
\(x_0\)}{Plot of v vs x\_0}}\label{plot-of-v-vs-x_0}}

For method 1 (green dots) the value of v increases as \(x_0\)
increases.And then it oscillates a little bit.This is because in method
one we don't take the amplitude into account thus leading to error. For
method 2 (blue dots) the value of v increases and then saturates and
also it doesn't oscillate much like that of method 1.

    \begin{Verbatim}[commandchars=\\\{\}]
{\color{incolor}In [{\color{incolor}14}]:} \PY{n}{plt}\PY{o}{.}\PY{n}{plot}\PY{p}{(}\PY{n}{x\PYZus{}0\PYZus{}arr}\PY{p}{,}\PY{n}{v\PYZus{}arr2}\PY{p}{,}\PY{l+s+s1}{\PYZsq{}}\PY{l+s+s1}{bo}\PY{l+s+s1}{\PYZsq{}}\PY{p}{,}\PY{n}{label}\PY{o}{=}\PY{l+s+s1}{\PYZsq{}}\PY{l+s+s1}{model 2}\PY{l+s+s1}{\PYZsq{}}\PY{p}{)}
         \PY{n}{plt}\PY{o}{.}\PY{n}{plot}\PY{p}{(}\PY{n}{x\PYZus{}0\PYZus{}arr}\PY{p}{,}\PY{n}{v\PYZus{}arr1}\PY{p}{,}\PY{l+s+s1}{\PYZsq{}}\PY{l+s+s1}{go}\PY{l+s+s1}{\PYZsq{}}\PY{p}{,}\PY{n}{label}\PY{o}{=}\PY{l+s+s1}{\PYZsq{}}\PY{l+s+s1}{model 1}\PY{l+s+s1}{\PYZsq{}}\PY{p}{)}
         \PY{n}{plt}\PY{o}{.}\PY{n}{title}\PY{p}{(}\PY{l+s+s1}{\PYZsq{}}\PY{l+s+s1}{Plot of \PYZdl{}v\PYZdl{} vs \PYZdl{}x\PYZus{}0\PYZdl{}}\PY{l+s+s1}{\PYZsq{}}\PY{p}{)}
         \PY{n}{plt}\PY{o}{.}\PY{n}{xlabel}\PY{p}{(}\PY{l+s+s1}{\PYZsq{}}\PY{l+s+s1}{\PYZdl{}x\PYZus{}0\PYZdl{}}\PY{l+s+s1}{\PYZsq{}}\PY{p}{)}
         \PY{n}{plt}\PY{o}{.}\PY{n}{ylabel}\PY{p}{(}\PY{l+s+s1}{\PYZsq{}}\PY{l+s+s1}{\PYZdl{}v\PYZdl{}}\PY{l+s+s1}{\PYZsq{}}\PY{p}{)}
         \PY{n}{plt}\PY{o}{.}\PY{n}{legend}\PY{p}{(}\PY{p}{)}
         \PY{n}{plt}\PY{o}{.}\PY{n}{grid}\PY{p}{(}\PY{p}{)}
         \PY{n}{plt}\PY{o}{.}\PY{n}{show}\PY{p}{(}\PY{p}{)}
\end{Verbatim}


    \begin{center}
    \adjustimage{max size={0.9\linewidth}{0.9\paperheight}}{output_23_0.png}
    \end{center}
    { \hspace*{\fill} \\}
    
    \begin{Verbatim}[commandchars=\\\{\}]
{\color{incolor}In [{\color{incolor}15}]:} \PY{k}{def} \PY{n+nf}{calcnu}\PY{p}{(}\PY{n}{x}\PY{p}{,}\PY{n}{x0}\PY{p}{,}\PY{n}{eps}\PY{p}{,}\PY{n}{model}\PY{p}{)}\PY{p}{:}
             \PY{n}{AB\PYZus{}array}\PY{o}{=}\PY{p}{[}\PY{p}{]}
             \PY{k}{for} \PY{n}{x\PYZus{}0} \PY{o+ow}{in} \PY{n}{x}\PY{p}{:}
                 \PY{k}{if} \PY{n}{x\PYZus{}0}\PY{o}{\PYZlt{}}\PY{o}{=}\PY{n}{x0}\PY{p}{:}
                     \PY{n}{A}\PY{p}{,}\PY{n}{J}\PY{o}{=}\PY{n}{model}\PY{p}{(}\PY{n}{x}\PY{p}{,}\PY{n}{x\PYZus{}0}\PY{p}{)}
                     \PY{n}{noise}\PY{o}{=}\PY{n}{eps}\PY{o}{*}\PY{n}{np}\PY{o}{.}\PY{n}{random}\PY{o}{.}\PY{n}{randn}\PY{p}{(}\PY{n+nb}{len}\PY{p}{(}\PY{n}{J}\PY{p}{)}\PY{p}{)}
                     \PY{n}{J}\PY{o}{+}\PY{o}{=}\PY{n}{noise} \PY{c+c1}{\PYZsh{}add noise to the function}
                     \PY{n}{A}\PY{o}{=}\PY{n}{np}\PY{o}{.}\PY{n}{array}\PY{p}{(}\PY{n}{A}\PY{p}{)}
                     \PY{n}{s}\PY{o}{=}\PY{n}{np}\PY{o}{.}\PY{n}{linalg}\PY{o}{.}\PY{n}{inv}\PY{p}{(}\PY{n}{np}\PY{o}{.}\PY{n}{dot}\PY{p}{(}\PY{n}{A}\PY{o}{.}\PY{n}{T}\PY{p}{,}\PY{n}{A}\PY{p}{)}\PY{p}{)}
                     \PY{n}{cl}\PY{o}{=}\PY{p}{(}\PY{n}{np}\PY{o}{.}\PY{n}{dot}\PY{p}{(}\PY{n}{s}\PY{p}{,}\PY{n}{np}\PY{o}{.}\PY{n}{dot}\PY{p}{(}\PY{n}{A}\PY{o}{.}\PY{n}{T}\PY{p}{,}\PY{n}{J}\PY{p}{)}\PY{p}{)}\PY{p}{)}
                     \PY{c+c1}{\PYZsh{}cl=np.linalg.lstsq(A,J)[0]  \PYZsh{}change to least square}
                     \PY{n}{AB\PYZus{}array}\PY{o}{.}\PY{n}{append}\PY{p}{(}\PY{n}{cl}\PY{p}{)}
             \PY{k}{return} \PY{n}{AB\PYZus{}array}
             
             
\end{Verbatim}


    \begin{Verbatim}[commandchars=\\\{\}]
{\color{incolor}In [{\color{incolor}16}]:} \PY{n}{AB3\PYZus{}array}\PY{o}{=}\PY{n}{calcnu}\PY{p}{(}\PY{n}{x}\PY{p}{,}\PY{l+m+mi}{18}\PY{p}{,}\PY{l+m+mf}{0.01}\PY{p}{,}\PY{n}{method\PYZus{}1}\PY{p}{)} \PY{c+c1}{\PYZsh{} has values of A and B for model 1 with noise}
         \PY{n}{AB4\PYZus{}array}\PY{o}{=}\PY{n}{calcnu}\PY{p}{(}\PY{n}{x}\PY{p}{,}\PY{l+m+mi}{18}\PY{p}{,}\PY{l+m+mf}{0.01}\PY{p}{,}\PY{n}{method\PYZus{}2}\PY{p}{)} \PY{c+c1}{\PYZsh{} has values of A and B for model 2 with noise}
         \PY{n}{v\PYZus{}arr3}\PY{o}{=}\PY{n}{calculate\PYZus{}v}\PY{p}{(}\PY{n}{AB3\PYZus{}array}\PY{p}{)}
         \PY{n}{v\PYZus{}arr4}\PY{o}{=}\PY{n}{calculate\PYZus{}v}\PY{p}{(}\PY{n}{AB4\PYZus{}array}\PY{p}{)}
\end{Verbatim}


    \hypertarget{plot-of-v-vs-x_0-with-noise-epsilon-0.01}{%
\section{\texorpdfstring{Plot of \(v\) vs \(x_0\) with noise
(\(\epsilon = 0.01\))}{Plot of v vs x\_0 with noise (\textbackslash{}epsilon = 0.01)}}\label{plot-of-v-vs-x_0-with-noise-epsilon-0.01}}

    \begin{Verbatim}[commandchars=\\\{\}]
{\color{incolor}In [{\color{incolor}17}]:} \PY{n}{AB3\PYZus{}array}\PY{o}{=}\PY{n}{calcnu}\PY{p}{(}\PY{n}{x}\PY{p}{,}\PY{l+m+mi}{18}\PY{p}{,}\PY{l+m+mf}{0.01}\PY{p}{,}\PY{n}{method\PYZus{}1}\PY{p}{)} \PY{c+c1}{\PYZsh{} has values of A and B for model 1 with noise}
         \PY{n}{AB4\PYZus{}array}\PY{o}{=}\PY{n}{calcnu}\PY{p}{(}\PY{n}{x}\PY{p}{,}\PY{l+m+mi}{18}\PY{p}{,}\PY{l+m+mf}{0.01}\PY{p}{,}\PY{n}{method\PYZus{}2}\PY{p}{)} \PY{c+c1}{\PYZsh{} has values of A and B for model 2 with noise}
         \PY{n}{v\PYZus{}arr3}\PY{o}{=}\PY{n}{calculate\PYZus{}v}\PY{p}{(}\PY{n}{AB3\PYZus{}array}\PY{p}{)}
         \PY{n}{v\PYZus{}arr4}\PY{o}{=}\PY{n}{calculate\PYZus{}v}\PY{p}{(}\PY{n}{AB4\PYZus{}array}\PY{p}{)}
         \PY{n}{plt}\PY{o}{.}\PY{n}{plot}\PY{p}{(}\PY{n}{x\PYZus{}0\PYZus{}arr}\PY{p}{,}\PY{n}{v\PYZus{}arr3}\PY{p}{,}\PY{l+s+s1}{\PYZsq{}}\PY{l+s+s1}{bo}\PY{l+s+s1}{\PYZsq{}}\PY{p}{,}\PY{n}{label}\PY{o}{=}\PY{l+s+s1}{\PYZsq{}}\PY{l+s+s1}{method 1 (noise=0.01)}\PY{l+s+s1}{\PYZsq{}}\PY{p}{)}
         \PY{n}{plt}\PY{o}{.}\PY{n}{plot}\PY{p}{(}\PY{n}{x\PYZus{}0\PYZus{}arr}\PY{p}{,}\PY{n}{v\PYZus{}arr4}\PY{p}{,}\PY{l+s+s1}{\PYZsq{}}\PY{l+s+s1}{ro}\PY{l+s+s1}{\PYZsq{}}\PY{p}{,}\PY{n}{label}\PY{o}{=}\PY{l+s+s1}{\PYZsq{}}\PY{l+s+s1}{method 2 (noise=0.01)}\PY{l+s+s1}{\PYZsq{}}\PY{p}{)}
         \PY{n}{plt}\PY{o}{.}\PY{n}{title}\PY{p}{(}\PY{l+s+s1}{\PYZsq{}}\PY{l+s+s1}{Plot of \PYZdl{}v\PYZdl{} vs \PYZdl{}x\PYZus{}0\PYZdl{} with noise=0.01}\PY{l+s+s1}{\PYZsq{}}\PY{p}{)}
         \PY{n}{plt}\PY{o}{.}\PY{n}{xlabel}\PY{p}{(}\PY{l+s+s1}{\PYZsq{}}\PY{l+s+s1}{\PYZdl{}x\PYZus{}0\PYZdl{}}\PY{l+s+s1}{\PYZsq{}}\PY{p}{)}
         \PY{n}{plt}\PY{o}{.}\PY{n}{ylabel}\PY{p}{(}\PY{l+s+s1}{\PYZsq{}}\PY{l+s+s1}{\PYZdl{}v\PYZdl{}}\PY{l+s+s1}{\PYZsq{}}\PY{p}{)}
         \PY{n}{plt}\PY{o}{.}\PY{n}{legend}\PY{p}{(}\PY{p}{)}
         \PY{n}{plt}\PY{o}{.}\PY{n}{grid}\PY{p}{(}\PY{p}{)}
         \PY{n}{plt}\PY{o}{.}\PY{n}{show}\PY{p}{(}\PY{p}{)}
\end{Verbatim}


    \begin{center}
    \adjustimage{max size={0.9\linewidth}{0.9\paperheight}}{output_27_0.png}
    \end{center}
    { \hspace*{\fill} \\}
    
    \begin{Verbatim}[commandchars=\\\{\}]
{\color{incolor}In [{\color{incolor}21}]:} \PY{n}{plt}\PY{o}{.}\PY{n}{plot}\PY{p}{(}\PY{n}{x\PYZus{}0\PYZus{}arr}\PY{p}{,}\PY{n}{v\PYZus{}arr2}\PY{p}{,}\PY{l+s+s1}{\PYZsq{}}\PY{l+s+s1}{bo}\PY{l+s+s1}{\PYZsq{}}\PY{p}{,}\PY{n}{label}\PY{o}{=}\PY{l+s+s1}{\PYZsq{}}\PY{l+s+s1}{model 2}\PY{l+s+s1}{\PYZsq{}}\PY{p}{)}
         \PY{n}{plt}\PY{o}{.}\PY{n}{plot}\PY{p}{(}\PY{n}{x\PYZus{}0\PYZus{}arr}\PY{p}{,}\PY{n}{v\PYZus{}arr1}\PY{p}{,}\PY{l+s+s1}{\PYZsq{}}\PY{l+s+s1}{go}\PY{l+s+s1}{\PYZsq{}}\PY{p}{,}\PY{n}{label}\PY{o}{=}\PY{l+s+s1}{\PYZsq{}}\PY{l+s+s1}{model 1}\PY{l+s+s1}{\PYZsq{}}\PY{p}{)}
         \PY{c+c1}{\PYZsh{}plt.plot(x\PYZus{}0\PYZus{}arr,v\PYZus{}arr3,\PYZsq{}bo\PYZsq{},label=\PYZsq{}method 1 (noise=0.01)\PYZsq{})}
         \PY{n}{plt}\PY{o}{.}\PY{n}{plot}\PY{p}{(}\PY{n}{x\PYZus{}0\PYZus{}arr}\PY{p}{,}\PY{n}{v\PYZus{}arr4}\PY{p}{,}\PY{l+s+s1}{\PYZsq{}}\PY{l+s+s1}{ro}\PY{l+s+s1}{\PYZsq{}}\PY{p}{,}\PY{n}{label}\PY{o}{=}\PY{l+s+s1}{\PYZsq{}}\PY{l+s+s1}{method 2 (noise=0.01)}\PY{l+s+s1}{\PYZsq{}}\PY{p}{)}
         \PY{n}{plt}\PY{o}{.}\PY{n}{title}\PY{p}{(}\PY{l+s+s1}{\PYZsq{}}\PY{l+s+s1}{Plot of \PYZdl{}v\PYZdl{} vs \PYZdl{}x\PYZus{}0\PYZdl{} }\PY{l+s+s1}{\PYZsq{}}\PY{p}{)}
         \PY{n}{plt}\PY{o}{.}\PY{n}{xlabel}\PY{p}{(}\PY{l+s+s1}{\PYZsq{}}\PY{l+s+s1}{\PYZdl{}x\PYZus{}0\PYZdl{}}\PY{l+s+s1}{\PYZsq{}}\PY{p}{)}
         \PY{n}{plt}\PY{o}{.}\PY{n}{ylabel}\PY{p}{(}\PY{l+s+s1}{\PYZsq{}}\PY{l+s+s1}{\PYZdl{}v\PYZdl{}}\PY{l+s+s1}{\PYZsq{}}\PY{p}{)}
         \PY{n}{plt}\PY{o}{.}\PY{n}{legend}\PY{p}{(}\PY{n}{loc}\PY{o}{=}\PY{l+s+s1}{\PYZsq{}}\PY{l+s+s1}{lower right}\PY{l+s+s1}{\PYZsq{}}\PY{p}{)}
         \PY{n}{plt}\PY{o}{.}\PY{n}{grid}\PY{p}{(}\PY{p}{)}
         \PY{n}{plt}\PY{o}{.}\PY{n}{show}\PY{p}{(}\PY{p}{)}
\end{Verbatim}


    \begin{center}
    \adjustimage{max size={0.9\linewidth}{0.9\paperheight}}{output_28_0.png}
    \end{center}
    { \hspace*{\fill} \\}
    
    \hypertarget{plot-v-vs-x_0-for-different-values-of-noise}{%
\section{\texorpdfstring{Plot \(v\) vs \(x_0\) for different values of
noise}{Plot v vs x\_0 for different values of noise}}\label{plot-v-vs-x_0-for-different-values-of-noise}}

    \begin{Verbatim}[commandchars=\\\{\}]
{\color{incolor}In [{\color{incolor}22}]:} \PY{c+c1}{\PYZsh{}plotting values of v vs x\PYZus{}0 for different values of epsilon}
         \PY{n}{epsilon}\PY{o}{=}\PY{p}{[}\PY{l+m+mi}{0}\PY{p}{,}\PY{l+m+mf}{0.001}\PY{p}{,}\PY{l+m+mf}{0.01}\PY{p}{,}\PY{l+m+mf}{0.1}\PY{p}{,}\PY{l+m+mi}{1}\PY{p}{]}
         \PY{k}{def} \PY{n+nf}{plot}\PY{p}{(}\PY{n}{x}\PY{p}{,}\PY{n}{x\PYZus{}0}\PY{p}{,}\PY{n}{epsilon}\PY{p}{,}\PY{n}{method}\PY{p}{,}\PY{n}{method\PYZus{}number}\PY{p}{)}\PY{p}{:}
             \PY{n}{i}\PY{o}{=}\PY{l+m+mi}{0}
             \PY{k}{for} \PY{n}{eps} \PY{o+ow}{in} \PY{n}{epsilon}\PY{p}{:}
                 \PY{n}{AB1\PYZus{}array}\PY{o}{=}\PY{n}{calcnu}\PY{p}{(}\PY{n}{x}\PY{p}{,}\PY{n}{x\PYZus{}0}\PY{p}{,}\PY{n}{eps}\PY{p}{,}\PY{n}{method}\PY{p}{)} \PY{c+c1}{\PYZsh{} has values of A and B for model 1 with noise}
                 \PY{n}{v\PYZus{}arr1}\PY{o}{=}\PY{n}{calculate\PYZus{}v}\PY{p}{(}\PY{n}{AB1\PYZus{}array}\PY{p}{)}
                 \PY{c+c1}{\PYZsh{}plt.subplot(5,2,i+1)}
                 \PY{n}{plt}\PY{o}{.}\PY{n}{plot}\PY{p}{(}\PY{n}{x\PYZus{}0\PYZus{}arr}\PY{p}{[}\PY{p}{:}\PY{o}{\PYZhy{}}\PY{p}{(}\PY{n+nb}{int}\PY{p}{(}\PY{n}{x\PYZus{}0}\PY{o}{/}\PY{l+m+mi}{10}\PY{p}{)}\PY{p}{)}\PY{p}{]}\PY{p}{,}\PY{n}{v\PYZus{}arr1}\PY{p}{[}\PY{p}{:}\PY{o}{\PYZhy{}}\PY{p}{(}\PY{n+nb}{int}\PY{p}{(}\PY{n}{x\PYZus{}0}\PY{o}{/}\PY{l+m+mi}{10}\PY{p}{)}\PY{p}{)}\PY{p}{]}\PY{p}{,}\PY{l+s+s1}{\PYZsq{}}\PY{l+s+s1}{bo}\PY{l+s+s1}{\PYZsq{}}\PY{p}{,}\PY{n}{label}\PY{o}{=}\PY{l+s+s1}{\PYZsq{}}\PY{l+s+s1}{method }\PY{l+s+si}{\PYZpc{}d}\PY{l+s+s1}{ (noise=}\PY{l+s+si}{\PYZpc{}f}\PY{l+s+s1}{)}\PY{l+s+s1}{\PYZsq{}}\PY{o}{\PYZpc{}}\PY{p}{(}\PY{n}{method\PYZus{}number}\PY{p}{,}\PY{n}{eps}\PY{p}{)}\PY{p}{)}
                 \PY{n}{plt}\PY{o}{.}\PY{n}{title}\PY{p}{(}\PY{l+s+s1}{\PYZsq{}}\PY{l+s+s1}{Plot of \PYZdl{}v\PYZdl{} vs \PYZdl{}x\PYZus{}0\PYZdl{} with noise=}\PY{l+s+si}{\PYZpc{}f}\PY{l+s+s1}{ }\PY{l+s+s1}{\PYZsq{}}\PY{o}{\PYZpc{}}\PY{k}{eps})
                 \PY{n}{plt}\PY{o}{.}\PY{n}{xlabel}\PY{p}{(}\PY{l+s+s1}{\PYZsq{}}\PY{l+s+s1}{\PYZdl{}x\PYZus{}0\PYZdl{}}\PY{l+s+s1}{\PYZsq{}}\PY{p}{)}
                 \PY{n}{plt}\PY{o}{.}\PY{n}{ylabel}\PY{p}{(}\PY{l+s+s1}{\PYZsq{}}\PY{l+s+s1}{\PYZdl{}v\PYZdl{}}\PY{l+s+s1}{\PYZsq{}}\PY{p}{)}
                 \PY{n}{plt}\PY{o}{.}\PY{n}{legend}\PY{p}{(}\PY{n}{loc}\PY{o}{=}\PY{l+s+s1}{\PYZsq{}}\PY{l+s+s1}{lower right}\PY{l+s+s1}{\PYZsq{}}\PY{p}{)}
                 \PY{n}{plt}\PY{o}{.}\PY{n}{grid}\PY{p}{(}\PY{p}{)}
                 \PY{n}{plt}\PY{o}{.}\PY{n}{show}\PY{p}{(}\PY{p}{)}
\end{Verbatim}


    \begin{Verbatim}[commandchars=\\\{\}]
{\color{incolor}In [{\color{incolor}23}]:} \PY{n}{plot}\PY{p}{(}\PY{n}{x\PYZus{}0\PYZus{}arr}\PY{p}{,}\PY{l+m+mi}{18}\PY{p}{,}\PY{n}{epsilon}\PY{p}{,}\PY{n}{method\PYZus{}1}\PY{p}{,}\PY{l+m+mi}{1}\PY{p}{)}
         \PY{n}{plot}\PY{p}{(}\PY{n}{x\PYZus{}0\PYZus{}arr}\PY{p}{,}\PY{l+m+mi}{18}\PY{p}{,}\PY{n}{epsilon}\PY{p}{,}\PY{n}{method\PYZus{}2}\PY{p}{,}\PY{l+m+mi}{2}\PY{p}{)}
\end{Verbatim}


    \begin{center}
    \adjustimage{max size={0.9\linewidth}{0.9\paperheight}}{output_31_0.png}
    \end{center}
    { \hspace*{\fill} \\}
    
    \begin{center}
    \adjustimage{max size={0.9\linewidth}{0.9\paperheight}}{output_31_1.png}
    \end{center}
    { \hspace*{\fill} \\}
    
    \begin{center}
    \adjustimage{max size={0.9\linewidth}{0.9\paperheight}}{output_31_2.png}
    \end{center}
    { \hspace*{\fill} \\}
    
    \begin{center}
    \adjustimage{max size={0.9\linewidth}{0.9\paperheight}}{output_31_3.png}
    \end{center}
    { \hspace*{\fill} \\}
    
    \begin{center}
    \adjustimage{max size={0.9\linewidth}{0.9\paperheight}}{output_31_4.png}
    \end{center}
    { \hspace*{\fill} \\}
    
   

    \begin{center}
    \adjustimage{max size={0.9\linewidth}{0.9\paperheight}}{output_31_6.png}
    \end{center}
    { \hspace*{\fill} \\}
    
    \begin{center}
    \adjustimage{max size={0.9\linewidth}{0.9\paperheight}}{output_31_7.png}
    \end{center}
    { \hspace*{\fill} \\}
    
    \begin{center}
    \adjustimage{max size={0.9\linewidth}{0.9\paperheight}}{output_31_8.png}
    \end{center}
    { \hspace*{\fill} \\}
    
    \begin{center}
    \adjustimage{max size={0.9\linewidth}{0.9\paperheight}}{output_31_9.png}
    \end{center}
    { \hspace*{\fill} \\}
    
    \begin{center}
    \adjustimage{max size={0.9\linewidth}{0.9\paperheight}}{output_31_10.png}
    \end{center}
    { \hspace*{\fill} \\}
    
    \hypertarget{plotting-with-different-number-of-measurements-with-epsilon0.01}{%
\section{\texorpdfstring{Plotting with different number of measurements
(with
\(\epsilon=0.01\))}{Plotting with different number of measurements (with \textbackslash{}epsilon=0.01)}}\label{plotting-with-different-number-of-measurements-with-epsilon0.01}}

The plots show that for sufficiently high number of measurements the
value of \(v\) converges to 1. Thus it shows that the number of
measuremnts is also an important factor in fitting functions with least
squares method.

    \begin{Verbatim}[commandchars=\\\{\}]
{\color{incolor}In [{\color{incolor}24}]:} \PY{n}{sizes}\PY{o}{=}\PY{p}{[}\PY{l+m+mi}{10}\PY{p}{,}\PY{l+m+mi}{50}\PY{p}{,}\PY{l+m+mi}{100}\PY{p}{,}\PY{l+m+mi}{250}\PY{p}{,}\PY{l+m+mi}{500}\PY{p}{,}\PY{l+m+mi}{1000}\PY{p}{,}\PY{l+m+mi}{10000}\PY{p}{]}
         \PY{k}{for} \PY{n}{size} \PY{o+ow}{in} \PY{n}{sizes}\PY{p}{:}
             \PY{n}{x\PYZus{}array}\PY{o}{=}\PY{n}{np}\PY{o}{.}\PY{n}{linspace}\PY{p}{(}\PY{l+m+mf}{0.5}\PY{p}{,}\PY{l+m+mi}{20}\PY{p}{,}\PY{n}{size}\PY{p}{)}
             \PY{n}{AB1\PYZus{}array}\PY{o}{=}\PY{n}{calcnu}\PY{p}{(}\PY{n}{x\PYZus{}array}\PY{p}{,}\PY{l+m+mi}{20}\PY{p}{,}\PY{l+m+mf}{0.01}\PY{p}{,}\PY{n}{method\PYZus{}2}\PY{p}{)} \PY{c+c1}{\PYZsh{} has values of A and B for model 1 with noise}
             \PY{n}{v\PYZus{}arr1}\PY{o}{=}\PY{n}{calculate\PYZus{}v}\PY{p}{(}\PY{n}{AB1\PYZus{}array}\PY{p}{)}
             \PY{n}{plt}\PY{o}{.}\PY{n}{plot}\PY{p}{(}\PY{n}{x\PYZus{}array}\PY{p}{[}\PY{p}{:}\PY{o}{\PYZhy{}}\PY{p}{(}\PY{n+nb}{int}\PY{p}{(}\PY{n}{x\PYZus{}0}\PY{o}{/}\PY{l+m+mi}{10}\PY{p}{)}\PY{p}{)}\PY{p}{]}\PY{p}{,}\PY{n}{v\PYZus{}arr1}\PY{p}{[}\PY{p}{:}\PY{o}{\PYZhy{}}\PY{p}{(}\PY{n+nb}{int}\PY{p}{(}\PY{n}{x\PYZus{}0}\PY{o}{/}\PY{l+m+mi}{10}\PY{p}{)}\PY{p}{)}\PY{p}{]}\PY{p}{,}\PY{l+s+s1}{\PYZsq{}}\PY{l+s+s1}{ro}\PY{l+s+s1}{\PYZsq{}}\PY{p}{,}\PY{n}{label}\PY{o}{=}\PY{l+s+s1}{\PYZsq{}}\PY{l+s+s1}{method 2}\PY{l+s+s1}{\PYZsq{}}\PY{p}{)}
             \PY{n}{plt}\PY{o}{.}\PY{n}{title}\PY{p}{(}\PY{l+s+s1}{\PYZsq{}}\PY{l+s+s1}{Plot of \PYZdl{}v\PYZdl{} vs \PYZdl{}x\PYZus{}0\PYZdl{} with number of measurements=}\PY{l+s+si}{\PYZpc{}d}\PY{l+s+s1}{\PYZsq{}}\PY{o}{\PYZpc{}}\PY{k}{size})
             \PY{n}{plt}\PY{o}{.}\PY{n}{xlabel}\PY{p}{(}\PY{l+s+s1}{\PYZsq{}}\PY{l+s+s1}{\PYZdl{}x\PYZus{}0\PYZdl{}}\PY{l+s+s1}{\PYZsq{}}\PY{p}{)}
             \PY{n}{plt}\PY{o}{.}\PY{n}{ylabel}\PY{p}{(}\PY{l+s+s1}{\PYZsq{}}\PY{l+s+s1}{\PYZdl{}v\PYZdl{}}\PY{l+s+s1}{\PYZsq{}}\PY{p}{)}
             \PY{n}{plt}\PY{o}{.}\PY{n}{legend}\PY{p}{(}\PY{n}{loc}\PY{o}{=}\PY{l+s+s1}{\PYZsq{}}\PY{l+s+s1}{lower right}\PY{l+s+s1}{\PYZsq{}}\PY{p}{)}
             \PY{n}{plt}\PY{o}{.}\PY{n}{grid}\PY{p}{(}\PY{p}{)}
             \PY{n}{plt}\PY{o}{.}\PY{n}{show}\PY{p}{(}\PY{p}{)}
\end{Verbatim}


    \begin{center}
    \adjustimage{max size={0.9\linewidth}{0.9\paperheight}}{output_33_0.png}
    \end{center}
    { \hspace*{\fill} \\}
    
    \begin{center}
    \adjustimage{max size={0.9\linewidth}{0.9\paperheight}}{output_33_1.png}
    \end{center}
    { \hspace*{\fill} \\}
    
   

    \begin{center}
    \adjustimage{max size={0.9\linewidth}{0.9\paperheight}}{output_33_3.png}
    \end{center}
    { \hspace*{\fill} \\}
    
    \begin{center}
    \adjustimage{max size={0.9\linewidth}{0.9\paperheight}}{output_33_4.png}
    \end{center}
    { \hspace*{\fill} \\}
    
    \begin{center}
    \adjustimage{max size={0.9\linewidth}{0.9\paperheight}}{output_33_5.png}
    \end{center}
    { \hspace*{\fill} \\}
    
    \begin{center}
    \adjustimage{max size={0.9\linewidth}{0.9\paperheight}}{output_33_6.png}
    \end{center}
    { \hspace*{\fill} \\}
    
    \begin{center}
    \adjustimage{max size={0.9\linewidth}{0.9\paperheight}}{output_33_7.png}
    \end{center}
    { \hspace*{\fill} \\}
    
    \hypertarget{discussions-and-conclusions}{%
\section{Discussions and
Conclusions}\label{discussions-and-conclusions}}

1:Model Selection:\\ We tried out two methods to fit the data.The first
one namely \(Acos(x)+Bsin(x)\approx J_v(x)\) did not fit well because of
the amplitude factor not taken into account. But the second method
namely
\(\frac{Acos(x)}{\sqrt{x}}+\frac{Bsin(x)}{\sqrt{x}}\approx J_v(x)\)
turned out to be a very good approximate function for the Bessel's
function of the first kind. \\2:Effect of noise: \\The value of noise in the
data hampers our fits by least squares method.As the graphs suggest the
function is not captured properly by the least squares method. Thus we
have to use some method to remove th noise from the data before fitting
it with least squares.This can be done by using various filters.\\
3:Effect of increasing the nuber of measurements: \\By increasing the
nuber of measurements the value of \(v\) starts to approach the desired
value(here it is equal to 1). This is because of the increase in the
datapoints which makes the least squares approximator get a better
estimate of the function.\\
* As this assignment was about least squares method I have implemented the least squares method and have not used the inbuilt function 'lstsq'

    % Add a bibliography block to the postdoc
    
    
    
    \end{document}
