
% Default to the notebook output style

    


% Inherit from the specified cell style.




    
\documentclass[11pt]{article}

    
    
    \usepackage[T1]{fontenc}
    % Nicer default font (+ math font) than Computer Modern for most use cases
    \usepackage{mathpazo}

    % Basic figure setup, for now with no caption control since it's done
    % automatically by Pandoc (which extracts ![](path) syntax from Markdown).
    \usepackage{graphicx}
    % We will generate all images so they have a width \maxwidth. This means
    % that they will get their normal width if they fit onto the page, but
    % are scaled down if they would overflow the margins.
    \makeatletter
    \def\maxwidth{\ifdim\Gin@nat@width>\linewidth\linewidth
    \else\Gin@nat@width\fi}
    \makeatother
    \let\Oldincludegraphics\includegraphics
    % Set max figure width to be 80% of text width, for now hardcoded.
    \renewcommand{\includegraphics}[1]{\Oldincludegraphics[width=.8\maxwidth]{#1}}
    % Ensure that by default, figures have no caption (until we provide a
    % proper Figure object with a Caption API and a way to capture that
    % in the conversion process - todo).
    \usepackage{caption}
    \DeclareCaptionLabelFormat{nolabel}{}
    \captionsetup{labelformat=nolabel}

    \usepackage{adjustbox} % Used to constrain images to a maximum size 
    \usepackage{xcolor} % Allow colors to be defined
    \usepackage{enumerate} % Needed for markdown enumerations to work
    \usepackage{geometry} % Used to adjust the document margins
    \usepackage{amsmath} % Equations
    \usepackage{amssymb} % Equations
    \usepackage{textcomp} % defines textquotesingle
    % Hack from http://tex.stackexchange.com/a/47451/13684:
    \AtBeginDocument{%
        \def\PYZsq{\textquotesingle}% Upright quotes in Pygmentized code
    }
    \usepackage{upquote} % Upright quotes for verbatim code
    \usepackage{eurosym} % defines \euro
    \usepackage[mathletters]{ucs} % Extended unicode (utf-8) support
    \usepackage[utf8x]{inputenc} % Allow utf-8 characters in the tex document
    \usepackage{fancyvrb} % verbatim replacement that allows latex
    \usepackage{grffile} % extends the file name processing of package graphics 
                         % to support a larger range 
    % The hyperref package gives us a pdf with properly built
    % internal navigation ('pdf bookmarks' for the table of contents,
    % internal cross-reference links, web links for URLs, etc.)
    \usepackage{hyperref}
    \usepackage{longtable} % longtable support required by pandoc >1.10
    \usepackage{booktabs}  % table support for pandoc > 1.12.2
    \usepackage[inline]{enumitem} % IRkernel/repr support (it uses the enumerate* environment)
    \usepackage[normalem]{ulem} % ulem is needed to support strikethroughs (\sout)
                                % normalem makes italics be italics, not underlines
    

    
    
    % Colors for the hyperref package
    \definecolor{urlcolor}{rgb}{0,.145,.698}
    \definecolor{linkcolor}{rgb}{.71,0.21,0.01}
    \definecolor{citecolor}{rgb}{.12,.54,.11}

    % ANSI colors
    \definecolor{ansi-black}{HTML}{3E424D}
    \definecolor{ansi-black-intense}{HTML}{282C36}
    \definecolor{ansi-red}{HTML}{E75C58}
    \definecolor{ansi-red-intense}{HTML}{B22B31}
    \definecolor{ansi-green}{HTML}{00A250}
    \definecolor{ansi-green-intense}{HTML}{007427}
    \definecolor{ansi-yellow}{HTML}{DDB62B}
    \definecolor{ansi-yellow-intense}{HTML}{B27D12}
    \definecolor{ansi-blue}{HTML}{208FFB}
    \definecolor{ansi-blue-intense}{HTML}{0065CA}
    \definecolor{ansi-magenta}{HTML}{D160C4}
    \definecolor{ansi-magenta-intense}{HTML}{A03196}
    \definecolor{ansi-cyan}{HTML}{60C6C8}
    \definecolor{ansi-cyan-intense}{HTML}{258F8F}
    \definecolor{ansi-white}{HTML}{C5C1B4}
    \definecolor{ansi-white-intense}{HTML}{A1A6B2}

    % commands and environments needed by pandoc snippets
    % extracted from the output of `pandoc -s`
    \providecommand{\tightlist}{%
      \setlength{\itemsep}{0pt}\setlength{\parskip}{0pt}}
    \DefineVerbatimEnvironment{Highlighting}{Verbatim}{commandchars=\\\{\}}
    % Add ',fontsize=\small' for more characters per line
    \newenvironment{Shaded}{}{}
    \newcommand{\KeywordTok}[1]{\textcolor[rgb]{0.00,0.44,0.13}{\textbf{{#1}}}}
    \newcommand{\DataTypeTok}[1]{\textcolor[rgb]{0.56,0.13,0.00}{{#1}}}
    \newcommand{\DecValTok}[1]{\textcolor[rgb]{0.25,0.63,0.44}{{#1}}}
    \newcommand{\BaseNTok}[1]{\textcolor[rgb]{0.25,0.63,0.44}{{#1}}}
    \newcommand{\FloatTok}[1]{\textcolor[rgb]{0.25,0.63,0.44}{{#1}}}
    \newcommand{\CharTok}[1]{\textcolor[rgb]{0.25,0.44,0.63}{{#1}}}
    \newcommand{\StringTok}[1]{\textcolor[rgb]{0.25,0.44,0.63}{{#1}}}
    \newcommand{\CommentTok}[1]{\textcolor[rgb]{0.38,0.63,0.69}{\textit{{#1}}}}
    \newcommand{\OtherTok}[1]{\textcolor[rgb]{0.00,0.44,0.13}{{#1}}}
    \newcommand{\AlertTok}[1]{\textcolor[rgb]{1.00,0.00,0.00}{\textbf{{#1}}}}
    \newcommand{\FunctionTok}[1]{\textcolor[rgb]{0.02,0.16,0.49}{{#1}}}
    \newcommand{\RegionMarkerTok}[1]{{#1}}
    \newcommand{\ErrorTok}[1]{\textcolor[rgb]{1.00,0.00,0.00}{\textbf{{#1}}}}
    \newcommand{\NormalTok}[1]{{#1}}
    
    % Additional commands for more recent versions of Pandoc
    \newcommand{\ConstantTok}[1]{\textcolor[rgb]{0.53,0.00,0.00}{{#1}}}
    \newcommand{\SpecialCharTok}[1]{\textcolor[rgb]{0.25,0.44,0.63}{{#1}}}
    \newcommand{\VerbatimStringTok}[1]{\textcolor[rgb]{0.25,0.44,0.63}{{#1}}}
    \newcommand{\SpecialStringTok}[1]{\textcolor[rgb]{0.73,0.40,0.53}{{#1}}}
    \newcommand{\ImportTok}[1]{{#1}}
    \newcommand{\DocumentationTok}[1]{\textcolor[rgb]{0.73,0.13,0.13}{\textit{{#1}}}}
    \newcommand{\AnnotationTok}[1]{\textcolor[rgb]{0.38,0.63,0.69}{\textbf{\textit{{#1}}}}}
    \newcommand{\CommentVarTok}[1]{\textcolor[rgb]{0.38,0.63,0.69}{\textbf{\textit{{#1}}}}}
    \newcommand{\VariableTok}[1]{\textcolor[rgb]{0.10,0.09,0.49}{{#1}}}
    \newcommand{\ControlFlowTok}[1]{\textcolor[rgb]{0.00,0.44,0.13}{\textbf{{#1}}}}
    \newcommand{\OperatorTok}[1]{\textcolor[rgb]{0.40,0.40,0.40}{{#1}}}
    \newcommand{\BuiltInTok}[1]{{#1}}
    \newcommand{\ExtensionTok}[1]{{#1}}
    \newcommand{\PreprocessorTok}[1]{\textcolor[rgb]{0.74,0.48,0.00}{{#1}}}
    \newcommand{\AttributeTok}[1]{\textcolor[rgb]{0.49,0.56,0.16}{{#1}}}
    \newcommand{\InformationTok}[1]{\textcolor[rgb]{0.38,0.63,0.69}{\textbf{\textit{{#1}}}}}
    \newcommand{\WarningTok}[1]{\textcolor[rgb]{0.38,0.63,0.69}{\textbf{\textit{{#1}}}}}
    
    
    % Define a nice break command that doesn't care if a line doesn't already
    % exist.
    \def\br{\hspace*{\fill} \\* }
    % Math Jax compatability definitions
    \def\gt{>}
    \def\lt{<}
    % Document parameters
    \title{Assignment 8}
    \author{Siddharth Nayak EE16B073}
    
    

    % Pygments definitions
    
\makeatletter
\def\PY@reset{\let\PY@it=\relax \let\PY@bf=\relax%
    \let\PY@ul=\relax \let\PY@tc=\relax%
    \let\PY@bc=\relax \let\PY@ff=\relax}
\def\PY@tok#1{\csname PY@tok@#1\endcsname}
\def\PY@toks#1+{\ifx\relax#1\empty\else%
    \PY@tok{#1}\expandafter\PY@toks\fi}
\def\PY@do#1{\PY@bc{\PY@tc{\PY@ul{%
    \PY@it{\PY@bf{\PY@ff{#1}}}}}}}
\def\PY#1#2{\PY@reset\PY@toks#1+\relax+\PY@do{#2}}

\expandafter\def\csname PY@tok@w\endcsname{\def\PY@tc##1{\textcolor[rgb]{0.73,0.73,0.73}{##1}}}
\expandafter\def\csname PY@tok@c\endcsname{\let\PY@it=\textit\def\PY@tc##1{\textcolor[rgb]{0.25,0.50,0.50}{##1}}}
\expandafter\def\csname PY@tok@cp\endcsname{\def\PY@tc##1{\textcolor[rgb]{0.74,0.48,0.00}{##1}}}
\expandafter\def\csname PY@tok@k\endcsname{\let\PY@bf=\textbf\def\PY@tc##1{\textcolor[rgb]{0.00,0.50,0.00}{##1}}}
\expandafter\def\csname PY@tok@kp\endcsname{\def\PY@tc##1{\textcolor[rgb]{0.00,0.50,0.00}{##1}}}
\expandafter\def\csname PY@tok@kt\endcsname{\def\PY@tc##1{\textcolor[rgb]{0.69,0.00,0.25}{##1}}}
\expandafter\def\csname PY@tok@o\endcsname{\def\PY@tc##1{\textcolor[rgb]{0.40,0.40,0.40}{##1}}}
\expandafter\def\csname PY@tok@ow\endcsname{\let\PY@bf=\textbf\def\PY@tc##1{\textcolor[rgb]{0.67,0.13,1.00}{##1}}}
\expandafter\def\csname PY@tok@nb\endcsname{\def\PY@tc##1{\textcolor[rgb]{0.00,0.50,0.00}{##1}}}
\expandafter\def\csname PY@tok@nf\endcsname{\def\PY@tc##1{\textcolor[rgb]{0.00,0.00,1.00}{##1}}}
\expandafter\def\csname PY@tok@nc\endcsname{\let\PY@bf=\textbf\def\PY@tc##1{\textcolor[rgb]{0.00,0.00,1.00}{##1}}}
\expandafter\def\csname PY@tok@nn\endcsname{\let\PY@bf=\textbf\def\PY@tc##1{\textcolor[rgb]{0.00,0.00,1.00}{##1}}}
\expandafter\def\csname PY@tok@ne\endcsname{\let\PY@bf=\textbf\def\PY@tc##1{\textcolor[rgb]{0.82,0.25,0.23}{##1}}}
\expandafter\def\csname PY@tok@nv\endcsname{\def\PY@tc##1{\textcolor[rgb]{0.10,0.09,0.49}{##1}}}
\expandafter\def\csname PY@tok@no\endcsname{\def\PY@tc##1{\textcolor[rgb]{0.53,0.00,0.00}{##1}}}
\expandafter\def\csname PY@tok@nl\endcsname{\def\PY@tc##1{\textcolor[rgb]{0.63,0.63,0.00}{##1}}}
\expandafter\def\csname PY@tok@ni\endcsname{\let\PY@bf=\textbf\def\PY@tc##1{\textcolor[rgb]{0.60,0.60,0.60}{##1}}}
\expandafter\def\csname PY@tok@na\endcsname{\def\PY@tc##1{\textcolor[rgb]{0.49,0.56,0.16}{##1}}}
\expandafter\def\csname PY@tok@nt\endcsname{\let\PY@bf=\textbf\def\PY@tc##1{\textcolor[rgb]{0.00,0.50,0.00}{##1}}}
\expandafter\def\csname PY@tok@nd\endcsname{\def\PY@tc##1{\textcolor[rgb]{0.67,0.13,1.00}{##1}}}
\expandafter\def\csname PY@tok@s\endcsname{\def\PY@tc##1{\textcolor[rgb]{0.73,0.13,0.13}{##1}}}
\expandafter\def\csname PY@tok@sd\endcsname{\let\PY@it=\textit\def\PY@tc##1{\textcolor[rgb]{0.73,0.13,0.13}{##1}}}
\expandafter\def\csname PY@tok@si\endcsname{\let\PY@bf=\textbf\def\PY@tc##1{\textcolor[rgb]{0.73,0.40,0.53}{##1}}}
\expandafter\def\csname PY@tok@se\endcsname{\let\PY@bf=\textbf\def\PY@tc##1{\textcolor[rgb]{0.73,0.40,0.13}{##1}}}
\expandafter\def\csname PY@tok@sr\endcsname{\def\PY@tc##1{\textcolor[rgb]{0.73,0.40,0.53}{##1}}}
\expandafter\def\csname PY@tok@ss\endcsname{\def\PY@tc##1{\textcolor[rgb]{0.10,0.09,0.49}{##1}}}
\expandafter\def\csname PY@tok@sx\endcsname{\def\PY@tc##1{\textcolor[rgb]{0.00,0.50,0.00}{##1}}}
\expandafter\def\csname PY@tok@m\endcsname{\def\PY@tc##1{\textcolor[rgb]{0.40,0.40,0.40}{##1}}}
\expandafter\def\csname PY@tok@gh\endcsname{\let\PY@bf=\textbf\def\PY@tc##1{\textcolor[rgb]{0.00,0.00,0.50}{##1}}}
\expandafter\def\csname PY@tok@gu\endcsname{\let\PY@bf=\textbf\def\PY@tc##1{\textcolor[rgb]{0.50,0.00,0.50}{##1}}}
\expandafter\def\csname PY@tok@gd\endcsname{\def\PY@tc##1{\textcolor[rgb]{0.63,0.00,0.00}{##1}}}
\expandafter\def\csname PY@tok@gi\endcsname{\def\PY@tc##1{\textcolor[rgb]{0.00,0.63,0.00}{##1}}}
\expandafter\def\csname PY@tok@gr\endcsname{\def\PY@tc##1{\textcolor[rgb]{1.00,0.00,0.00}{##1}}}
\expandafter\def\csname PY@tok@ge\endcsname{\let\PY@it=\textit}
\expandafter\def\csname PY@tok@gs\endcsname{\let\PY@bf=\textbf}
\expandafter\def\csname PY@tok@gp\endcsname{\let\PY@bf=\textbf\def\PY@tc##1{\textcolor[rgb]{0.00,0.00,0.50}{##1}}}
\expandafter\def\csname PY@tok@go\endcsname{\def\PY@tc##1{\textcolor[rgb]{0.53,0.53,0.53}{##1}}}
\expandafter\def\csname PY@tok@gt\endcsname{\def\PY@tc##1{\textcolor[rgb]{0.00,0.27,0.87}{##1}}}
\expandafter\def\csname PY@tok@err\endcsname{\def\PY@bc##1{\setlength{\fboxsep}{0pt}\fcolorbox[rgb]{1.00,0.00,0.00}{1,1,1}{\strut ##1}}}
\expandafter\def\csname PY@tok@kc\endcsname{\let\PY@bf=\textbf\def\PY@tc##1{\textcolor[rgb]{0.00,0.50,0.00}{##1}}}
\expandafter\def\csname PY@tok@kd\endcsname{\let\PY@bf=\textbf\def\PY@tc##1{\textcolor[rgb]{0.00,0.50,0.00}{##1}}}
\expandafter\def\csname PY@tok@kn\endcsname{\let\PY@bf=\textbf\def\PY@tc##1{\textcolor[rgb]{0.00,0.50,0.00}{##1}}}
\expandafter\def\csname PY@tok@kr\endcsname{\let\PY@bf=\textbf\def\PY@tc##1{\textcolor[rgb]{0.00,0.50,0.00}{##1}}}
\expandafter\def\csname PY@tok@bp\endcsname{\def\PY@tc##1{\textcolor[rgb]{0.00,0.50,0.00}{##1}}}
\expandafter\def\csname PY@tok@fm\endcsname{\def\PY@tc##1{\textcolor[rgb]{0.00,0.00,1.00}{##1}}}
\expandafter\def\csname PY@tok@vc\endcsname{\def\PY@tc##1{\textcolor[rgb]{0.10,0.09,0.49}{##1}}}
\expandafter\def\csname PY@tok@vg\endcsname{\def\PY@tc##1{\textcolor[rgb]{0.10,0.09,0.49}{##1}}}
\expandafter\def\csname PY@tok@vi\endcsname{\def\PY@tc##1{\textcolor[rgb]{0.10,0.09,0.49}{##1}}}
\expandafter\def\csname PY@tok@vm\endcsname{\def\PY@tc##1{\textcolor[rgb]{0.10,0.09,0.49}{##1}}}
\expandafter\def\csname PY@tok@sa\endcsname{\def\PY@tc##1{\textcolor[rgb]{0.73,0.13,0.13}{##1}}}
\expandafter\def\csname PY@tok@sb\endcsname{\def\PY@tc##1{\textcolor[rgb]{0.73,0.13,0.13}{##1}}}
\expandafter\def\csname PY@tok@sc\endcsname{\def\PY@tc##1{\textcolor[rgb]{0.73,0.13,0.13}{##1}}}
\expandafter\def\csname PY@tok@dl\endcsname{\def\PY@tc##1{\textcolor[rgb]{0.73,0.13,0.13}{##1}}}
\expandafter\def\csname PY@tok@s2\endcsname{\def\PY@tc##1{\textcolor[rgb]{0.73,0.13,0.13}{##1}}}
\expandafter\def\csname PY@tok@sh\endcsname{\def\PY@tc##1{\textcolor[rgb]{0.73,0.13,0.13}{##1}}}
\expandafter\def\csname PY@tok@s1\endcsname{\def\PY@tc##1{\textcolor[rgb]{0.73,0.13,0.13}{##1}}}
\expandafter\def\csname PY@tok@mb\endcsname{\def\PY@tc##1{\textcolor[rgb]{0.40,0.40,0.40}{##1}}}
\expandafter\def\csname PY@tok@mf\endcsname{\def\PY@tc##1{\textcolor[rgb]{0.40,0.40,0.40}{##1}}}
\expandafter\def\csname PY@tok@mh\endcsname{\def\PY@tc##1{\textcolor[rgb]{0.40,0.40,0.40}{##1}}}
\expandafter\def\csname PY@tok@mi\endcsname{\def\PY@tc##1{\textcolor[rgb]{0.40,0.40,0.40}{##1}}}
\expandafter\def\csname PY@tok@il\endcsname{\def\PY@tc##1{\textcolor[rgb]{0.40,0.40,0.40}{##1}}}
\expandafter\def\csname PY@tok@mo\endcsname{\def\PY@tc##1{\textcolor[rgb]{0.40,0.40,0.40}{##1}}}
\expandafter\def\csname PY@tok@ch\endcsname{\let\PY@it=\textit\def\PY@tc##1{\textcolor[rgb]{0.25,0.50,0.50}{##1}}}
\expandafter\def\csname PY@tok@cm\endcsname{\let\PY@it=\textit\def\PY@tc##1{\textcolor[rgb]{0.25,0.50,0.50}{##1}}}
\expandafter\def\csname PY@tok@cpf\endcsname{\let\PY@it=\textit\def\PY@tc##1{\textcolor[rgb]{0.25,0.50,0.50}{##1}}}
\expandafter\def\csname PY@tok@c1\endcsname{\let\PY@it=\textit\def\PY@tc##1{\textcolor[rgb]{0.25,0.50,0.50}{##1}}}
\expandafter\def\csname PY@tok@cs\endcsname{\let\PY@it=\textit\def\PY@tc##1{\textcolor[rgb]{0.25,0.50,0.50}{##1}}}

\def\PYZbs{\char`\\}
\def\PYZus{\char`\_}
\def\PYZob{\char`\{}
\def\PYZcb{\char`\}}
\def\PYZca{\char`\^}
\def\PYZam{\char`\&}
\def\PYZlt{\char`\<}
\def\PYZgt{\char`\>}
\def\PYZsh{\char`\#}
\def\PYZpc{\char`\%}
\def\PYZdl{\char`\$}
\def\PYZhy{\char`\-}
\def\PYZsq{\char`\'}
\def\PYZdq{\char`\"}
\def\PYZti{\char`\~}
% for compatibility with earlier versions
\def\PYZat{@}
\def\PYZlb{[}
\def\PYZrb{]}
\makeatother


    % Exact colors from NB
    \definecolor{incolor}{rgb}{0.0, 0.0, 0.5}
    \definecolor{outcolor}{rgb}{0.545, 0.0, 0.0}



    
    % Prevent overflowing lines due to hard-to-break entities
    \sloppy 
    % Setup hyperref package
    \hypersetup{
      breaklinks=true,  % so long urls are correctly broken across lines
      colorlinks=true,
      urlcolor=urlcolor,
      linkcolor=linkcolor,
      citecolor=citecolor,
      }
    % Slightly bigger margins than the latex defaults
    
    \geometry{verbose,tmargin=1in,bmargin=1in,lmargin=1in,rmargin=1in}
    
    

    \begin{document}
    
    
    \maketitle
    
    

    
    \hypertarget{introduction}{%
\section{Introduction}\label{introduction}}

In this assignment simulations have been carried out by solving the
circuits symbolically. We just have to feed in a matrix and we can can
simulate the circuit.In this assignment a Highpass and a Lowpass circuit
has been simulated.

    \hypertarget{import-libraries}{%
\section{Import Libraries}\label{import-libraries}}

    \begin{Verbatim}[commandchars=\\\{\}]
{\color{incolor}In [{\color{incolor}1}]:} \PY{k+kn}{from} \PY{n+nn}{sympy} \PY{k}{import} \PY{o}{*}
        \PY{k+kn}{import} \PY{n+nn}{numpy} \PY{k}{as} \PY{n+nn}{np}
        \PY{k+kn}{import} \PY{n+nn}{matplotlib}\PY{n+nn}{.}\PY{n+nn}{pyplot} \PY{k}{as} \PY{n+nn}{plt}
        \PY{k+kn}{import} \PY{n+nn}{scipy}\PY{n+nn}{.}\PY{n+nn}{signal} \PY{k}{as} \PY{n+nn}{sp}
\end{Verbatim}


    \hypertarget{lowpass-filter}{%
\section{Lowpass filter}\label{lowpass-filter}}

In the circuit given below we write the nodal equations and form a
matrix.

       \begin{center}
    \adjustimage{max size={0.9\linewidth}{0.9\paperheight}}{output_4_0.png}
    \end{center}
    { \hspace*{\fill} \\}\\
  
  
    

The circuit equations are: \\
    \(V_m=\frac{V_m}{G_1}\)\\
\(V_p=V1\frac{1}{1+j\omega R_2.C_2}\) \\
\(V_0=G_2(V_p-V_m)\)\\
\(\frac{V_i-V_1}{R1}+\frac{V_p-V_1}{R2}+j\omega C1(V_0-V_1)=0\)\\
Therefore the matrix becomes:\\
\(\begin{pmatrix}  0 & 0 & 1 & -\frac{1}{G_1} \\  -\frac{1}{1+sC_2R_2} & 1 & 0 & 0 \\  0 & -G_2 & G_2 & 1\\  -\frac{1}{R_1}-\frac{1}{R_2}-sC_1 & \frac{1}{R_2} & 0 & sC_1  \end{pmatrix}  .  \begin{pmatrix}  V_1\\  V_p\\  V_m\\  V_0  \end{pmatrix} = \begin{pmatrix}  0\\  0\\  0\\  V_i(s)/R_1  \end{pmatrix}\)\\
\\ * Assuming \(G_2\) is the gain of the opamp
\\ * Image courtesy:The Art of
Electronics by Paul Horowitz and Winfield Hill

    \begin{Verbatim}[commandchars=\\\{\}]
{\color{incolor}In [{\color{incolor}3}]:} \PY{n}{s}\PY{o}{=}\PY{n}{symbols}\PY{p}{(}\PY{l+s+s1}{\PYZsq{}}\PY{l+s+s1}{s}\PY{l+s+s1}{\PYZsq{}}\PY{p}{)}
        \PY{k}{def} \PY{n+nf}{lowpass}\PY{p}{(}\PY{n}{R1}\PY{p}{,}\PY{n}{R2}\PY{p}{,}\PY{n}{C1}\PY{p}{,}\PY{n}{C2}\PY{p}{,}\PY{n}{G\PYZus{}1}\PY{p}{,}\PY{n}{G\PYZus{}2}\PY{p}{,}\PY{n}{Vi}\PY{p}{)}\PY{p}{:}
            \PY{n}{s}\PY{o}{=}\PY{n}{symbols}\PY{p}{(}\PY{l+s+s1}{\PYZsq{}}\PY{l+s+s1}{s}\PY{l+s+s1}{\PYZsq{}}\PY{p}{)}
            \PY{n}{A}\PY{o}{=}\PY{n}{Matrix}\PY{p}{(}\PY{p}{[}\PY{p}{[}\PY{l+m+mi}{0}\PY{p}{,}\PY{l+m+mi}{0}\PY{p}{,}\PY{l+m+mi}{1}\PY{p}{,}\PY{o}{\PYZhy{}}\PY{l+m+mi}{1}\PY{o}{/}\PY{n}{G\PYZus{}1}\PY{p}{]}\PY{p}{,}\PY{p}{[}\PY{o}{\PYZhy{}}\PY{l+m+mi}{1}\PY{o}{/}\PY{p}{(}\PY{l+m+mi}{1}\PY{o}{+}\PY{n}{s}\PY{o}{*}\PY{n}{R2}\PY{o}{*}\PY{n}{C2}\PY{p}{)}\PY{p}{,}\PY{l+m+mi}{1}\PY{p}{,}\PY{l+m+mi}{0}\PY{p}{,}\PY{l+m+mi}{0}\PY{p}{]}\PY{p}{,}\PY{p}{[}\PY{l+m+mi}{0}\PY{p}{,}\PY{o}{\PYZhy{}}\PY{n}{G\PYZus{}2}\PY{p}{,}\PY{n}{G\PYZus{}2}\PY{p}{,}\PY{l+m+mi}{1}\PY{p}{]}\PY{p}{,}\PY{p}{[}\PY{o}{\PYZhy{}}\PY{l+m+mi}{1}\PY{o}{/}\PY{n}{R1}\PY{o}{\PYZhy{}}\PY{l+m+mi}{1}\PY{o}{/}\PY{n}{R2}\PY{o}{\PYZhy{}}\PY{n}{s}\PY{o}{*}\PY{n}{C1}\PY{p}{,}\PY{l+m+mi}{1}\PY{o}{/}\PY{n}{R2}\PY{p}{,}\PY{l+m+mi}{0}\PY{p}{,}\PY{n}{s}\PY{o}{*}\PY{n}{C1}\PY{p}{]}\PY{p}{]}\PY{p}{)}
            \PY{n}{b}\PY{o}{=}\PY{n}{Matrix}\PY{p}{(}\PY{p}{[}\PY{p}{[}\PY{l+m+mi}{0}\PY{p}{]}\PY{p}{,}\PY{p}{[}\PY{l+m+mi}{0}\PY{p}{]}\PY{p}{,}\PY{p}{[}\PY{l+m+mi}{0}\PY{p}{]}\PY{p}{,}\PY{p}{[}\PY{o}{\PYZhy{}}\PY{n}{Vi}\PY{o}{/}\PY{n}{R1}\PY{p}{]}\PY{p}{]}\PY{p}{)}
            \PY{n}{V}\PY{o}{=}\PY{n}{A}\PY{o}{.}\PY{n}{inv}\PY{p}{(}\PY{p}{)}\PY{o}{*}\PY{n}{b}
            \PY{k}{return} \PY{n}{A}\PY{p}{,}\PY{n}{b}\PY{p}{,}\PY{n}{V}
        
        \PY{n}{A}\PY{p}{,}\PY{n}{b}\PY{p}{,}\PY{n}{V}\PY{o}{=}\PY{n}{lowpass}\PY{p}{(}\PY{l+m+mi}{10000}\PY{p}{,}\PY{l+m+mi}{10000}\PY{p}{,}\PY{l+m+mf}{1e\PYZhy{}9}\PY{p}{,}\PY{l+m+mf}{1e\PYZhy{}9}\PY{p}{,}\PY{l+m+mf}{1.586}\PY{p}{,}\PY{l+m+mi}{1000}\PY{p}{,}\PY{l+m+mi}{1}\PY{p}{)}
        \PY{n}{V0}\PY{o}{=}\PY{n}{V}\PY{p}{[}\PY{l+m+mi}{3}\PY{p}{]}
\end{Verbatim}


    \begin{Verbatim}[commandchars=\\\{\}]
{\color{incolor}In [{\color{incolor}4}]:} \PY{k}{def} \PY{n+nf}{plot}\PY{p}{(}\PY{n}{V0}\PY{p}{,}\PY{n}{upperlimit}\PY{p}{,}\PY{n}{title}\PY{p}{)}\PY{p}{:}
            \PY{l+s+sd}{\PYZsq{}\PYZsq{}\PYZsq{}}
        \PY{l+s+sd}{    Plots the transfer function of V0}
        \PY{l+s+sd}{    upperlimit=maximum omega}
        \PY{l+s+sd}{    title= title for the plot}
        \PY{l+s+sd}{    \PYZsq{}\PYZsq{}\PYZsq{}}
            \PY{n}{w}\PY{o}{=}\PY{n}{np}\PY{o}{.}\PY{n}{logspace}\PY{p}{(}\PY{l+m+mi}{0}\PY{p}{,}\PY{n}{upperlimit}\PY{p}{,}\PY{n}{upperlimit}\PY{o}{*}\PY{l+m+mi}{10}\PY{o}{+}\PY{l+m+mi}{1}\PY{p}{)}
            \PY{n}{ss}\PY{o}{=}\PY{l+m+mi}{1}\PY{n}{j}\PY{o}{*}\PY{n}{w}
            \PY{n}{s}\PY{o}{=}\PY{n}{symbols}\PY{p}{(}\PY{l+s+s1}{\PYZsq{}}\PY{l+s+s1}{s}\PY{l+s+s1}{\PYZsq{}}\PY{p}{)}
            \PY{n}{hf}\PY{o}{=}\PY{n}{lambdify}\PY{p}{(}\PY{n}{s}\PY{p}{,}\PY{n}{V0}\PY{p}{,}\PY{l+s+s1}{\PYZsq{}}\PY{l+s+s1}{numpy}\PY{l+s+s1}{\PYZsq{}}\PY{p}{)}
            \PY{n}{v}\PY{o}{=}\PY{n}{hf}\PY{p}{(}\PY{n}{ss}\PY{p}{)}
            \PY{n}{plt}\PY{o}{.}\PY{n}{loglog}\PY{p}{(}\PY{n}{w}\PY{p}{,}\PY{n+nb}{abs}\PY{p}{(}\PY{n}{v}\PY{p}{)}\PY{p}{,}\PY{n}{lw}\PY{o}{=}\PY{l+m+mi}{2}\PY{p}{)}
            \PY{n}{plt}\PY{o}{.}\PY{n}{xlabel}\PY{p}{(}\PY{l+s+s1}{\PYZsq{}}\PY{l+s+s1}{\PYZdl{}}\PY{l+s+s1}{\PYZbs{}}\PY{l+s+s1}{omega\PYZdl{}}\PY{l+s+s1}{\PYZsq{}}\PY{p}{)}
            \PY{n}{plt}\PY{o}{.}\PY{n}{ylabel}\PY{p}{(}\PY{l+s+s1}{\PYZsq{}}\PY{l+s+s1}{\PYZdl{}|H(e\PYZca{}}\PY{l+s+s1}{\PYZob{}}\PY{l+s+s1}{j }\PY{l+s+s1}{\PYZbs{}}\PY{l+s+s1}{omega\PYZcb{})|\PYZdl{}}\PY{l+s+s1}{\PYZsq{}}\PY{p}{)}
            \PY{n}{plt}\PY{o}{.}\PY{n}{title}\PY{p}{(}\PY{n}{title}\PY{p}{)}
            \PY{n}{plt}\PY{o}{.}\PY{n}{grid}\PY{p}{(}\PY{k+kc}{True}\PY{p}{)}
            \PY{n}{plt}\PY{o}{.}\PY{n}{show}\PY{p}{(}\PY{p}{)}
\end{Verbatim}


    \hypertarget{plot-of-transfer-function-of-lowpass-filter}{%
\section{Plot of transfer function of lowpass
filter}\label{plot-of-transfer-function-of-lowpass-filter}}

This filter has a cutoff at \(\omega=10^5\). After this it falls at
20dB/dec.

    \begin{Verbatim}[commandchars=\\\{\}]
{\color{incolor}In [{\color{incolor}5}]:} \PY{n}{plot}\PY{p}{(}\PY{n}{V0}\PY{p}{,}\PY{l+m+mi}{8}\PY{p}{,}\PY{l+s+s1}{\PYZsq{}}\PY{l+s+s1}{Transfer Function}\PY{l+s+s1}{\PYZsq{}}\PY{p}{)}
\end{Verbatim}


    \begin{center}
    \adjustimage{max size={0.9\linewidth}{0.9\paperheight}}{output_9_0.png}
    \end{center}
    { \hspace*{\fill} \\}
    
    \begin{Verbatim}[commandchars=\\\{\}]
{\color{incolor}In [{\color{incolor}6}]:} \PY{k}{def} \PY{n+nf}{plot\PYZus{}graph}\PY{p}{(}\PY{n}{V0}\PY{p}{,}\PY{n}{t}\PY{p}{,}\PY{n}{V\PYZus{}i}\PY{p}{,}\PY{n}{V\PYZus{}i\PYZus{}name}\PY{p}{)}\PY{p}{:}
            \PY{l+s+sd}{\PYZsq{}\PYZsq{}\PYZsq{}}
        \PY{l+s+sd}{    Plots the time response}
        \PY{l+s+sd}{    V0=transfer function}
        \PY{l+s+sd}{    t=time array}
        \PY{l+s+sd}{    V\PYZus{}i=input voltge}
        \PY{l+s+sd}{    V\PYZus{}i\PYZus{}name= name of input voltage to be shown in the graph}
        \PY{l+s+sd}{    \PYZsq{}\PYZsq{}\PYZsq{}}
            \PY{n}{expr\PYZus{}num}\PY{p}{,} \PY{n}{expr\PYZus{}den} \PY{o}{=} \PY{n}{V0}\PY{o}{.}\PY{n}{as\PYZus{}numer\PYZus{}denom}\PY{p}{(}\PY{p}{)}
            \PY{n}{Hnum} \PY{o}{=} \PY{n}{Poly}\PY{p}{(}\PY{n}{expr\PYZus{}num}\PY{p}{)}\PY{o}{.}\PY{n}{coeffs}\PY{p}{(}\PY{p}{)}
            \PY{n}{Hden} \PY{o}{=} \PY{p}{(}\PY{n}{Poly}\PY{p}{(}\PY{n}{expr\PYZus{}den}\PY{p}{)}\PY{p}{)}\PY{o}{.}\PY{n}{coeffs}\PY{p}{(}\PY{p}{)}
            \PY{n}{transferX} \PY{o}{=} \PY{n}{sp}\PY{o}{.}\PY{n}{lti}\PY{p}{(}\PY{n}{np}\PY{o}{.}\PY{n}{array}\PY{p}{(}\PY{n}{Hnum}\PY{p}{,} \PY{n}{dtype}\PY{o}{=}\PY{n+nb}{float}\PY{p}{)}\PY{p}{,} \PY{n}{np}\PY{o}{.}\PY{n}{array}\PY{p}{(}\PY{n}{Hden}\PY{p}{,} \PY{n}{dtype}\PY{o}{=}\PY{n+nb}{float}\PY{p}{)}\PY{p}{)}
            \PY{n}{t}\PY{p}{,}\PY{n}{x}\PY{p}{,}\PY{n}{svec} \PY{o}{=} \PY{n}{sp}\PY{o}{.}\PY{n}{lsim}\PY{p}{(}\PY{n}{transferX}\PY{p}{,}\PY{n}{V\PYZus{}i}\PY{p}{,}\PY{n}{t}\PY{p}{)}
            \PY{n}{plt}\PY{o}{.}\PY{n}{plot}\PY{p}{(}\PY{n}{t}\PY{p}{,}\PY{n}{x}\PY{p}{)}
            \PY{n}{plt}\PY{o}{.}\PY{n}{title}\PY{p}{(}\PY{l+s+s1}{\PYZsq{}}\PY{l+s+s1}{Response to input }\PY{l+s+si}{\PYZpc{}s}\PY{l+s+s1}{\PYZsq{}}\PY{o}{\PYZpc{}}\PY{k}{V\PYZus{}i\PYZus{}name})
            \PY{n}{plt}\PY{o}{.}\PY{n}{xlabel}\PY{p}{(}\PY{l+s+s1}{\PYZsq{}}\PY{l+s+s1}{\PYZdl{}t\PYZdl{}}\PY{l+s+s1}{\PYZsq{}}\PY{p}{)}
            \PY{n}{plt}\PY{o}{.}\PY{n}{ylabel}\PY{p}{(}\PY{l+s+s1}{\PYZsq{}}\PY{l+s+s1}{\PYZdl{}V\PYZus{}0(t)\PYZdl{}}\PY{l+s+s1}{\PYZsq{}}\PY{p}{)}
            \PY{n}{plt}\PY{o}{.}\PY{n}{grid}\PY{p}{(}\PY{p}{)}
            \PY{n}{plt}\PY{o}{.}\PY{n}{show}\PY{p}{(}\PY{p}{)}
\end{Verbatim}


    The Laplace transform of the unit response for the lowpass filter falls
initially at 20dB/dec till \(\omega=10^5\) and then it falls at 40
dB/dec

    \begin{Verbatim}[commandchars=\\\{\}]
{\color{incolor}In [{\color{incolor}7}]:} \PY{n}{V\PYZus{}i}\PY{o}{=}\PY{l+m+mi}{1}\PY{o}{/}\PY{n}{s}
        \PY{n}{A}\PY{p}{,}\PY{n}{b}\PY{p}{,}\PY{n}{V}\PY{o}{=}\PY{n}{lowpass}\PY{p}{(}\PY{l+m+mi}{10000}\PY{p}{,}\PY{l+m+mi}{10000}\PY{p}{,}\PY{l+m+mf}{1e\PYZhy{}9}\PY{p}{,}\PY{l+m+mf}{1e\PYZhy{}9}\PY{p}{,}\PY{l+m+mf}{1.586}\PY{p}{,}\PY{l+m+mi}{1000}\PY{p}{,}\PY{n}{V\PYZus{}i}\PY{p}{)}
        \PY{n}{V0}\PY{o}{=}\PY{n}{V}\PY{p}{[}\PY{l+m+mi}{3}\PY{p}{]}
        \PY{n}{plot}\PY{p}{(}\PY{n}{V0}\PY{p}{,}\PY{l+m+mi}{8}\PY{p}{,}\PY{l+s+s1}{\PYZsq{}}\PY{l+s+s1}{Unit step response of laplace transform}\PY{l+s+s1}{\PYZsq{}}\PY{p}{)}
\end{Verbatim}


    \begin{center}
    \adjustimage{max size={0.9\linewidth}{0.9\paperheight}}{output_12_0.png}
    \end{center}
    { \hspace*{\fill} \\}
    
    The unit response of the lowpass filter reaches it's steady state after
a small overshoot.This overshoot is because of the quality factor being
greater than \(\frac{1}{\sqrt{2}}\) in the second oreder system.

    \begin{Verbatim}[commandchars=\\\{\}]
{\color{incolor}In [{\color{incolor}8}]:} \PY{n}{t}\PY{o}{=}\PY{n}{np}\PY{o}{.}\PY{n}{arange}\PY{p}{(}\PY{l+m+mi}{0}\PY{p}{,}\PY{l+m+mf}{1e\PYZhy{}3}\PY{p}{,}\PY{l+m+mf}{1e\PYZhy{}8}\PY{p}{)}
        \PY{n}{V\PYZus{}i} \PY{o}{=} \PY{n}{np}\PY{o}{.}\PY{n}{ones}\PY{p}{(}\PY{n}{t}\PY{o}{.}\PY{n}{size}\PY{p}{)} \PY{c+c1}{\PYZsh{}unit step}
        \PY{n}{plot\PYZus{}graph}\PY{p}{(}\PY{n}{V0}\PY{p}{,}\PY{n}{t}\PY{p}{,}\PY{n}{V\PYZus{}i}\PY{p}{,}\PY{l+s+s1}{\PYZsq{}}\PY{l+s+s1}{unit step}\PY{l+s+s1}{\PYZsq{}}\PY{p}{)}
\end{Verbatim}


    \begin{center}
    \adjustimage{max size={0.9\linewidth}{0.9\paperheight}}{output_14_0.png}
    \end{center}
    { \hspace*{\fill} \\}
    
    \hypertarget{sinusoidal-input-voltage}{%
\section{Sinusoidal input voltage}\label{sinusoidal-input-voltage}}

\(V_i(t)=(sin(2000\pi t)+cos(2*10^6\pi t))u_0{t}\) In the input we have
to frequency components:\(10^3\) and \(10^6\). This causes the system to
respond to the \(10^3\) component and \(10^6\) component but the second
component is attenuated heavily.

    \begin{Verbatim}[commandchars=\\\{\}]
{\color{incolor}In [{\color{incolor}9}]:} \PY{n}{s}\PY{o}{=}\PY{n}{symbols}\PY{p}{(}\PY{l+s+s1}{\PYZsq{}}\PY{l+s+s1}{s}\PY{l+s+s1}{\PYZsq{}}\PY{p}{)}
        \PY{n}{w\PYZus{}0}\PY{o}{=}\PY{l+m+mi}{2000}\PY{o}{*}\PY{n}{np}\PY{o}{.}\PY{n}{pi}
        \PY{n}{sin\PYZus{}s}\PY{o}{=}\PY{n}{w\PYZus{}0}\PY{o}{/}\PY{p}{(}\PY{p}{(}\PY{n}{s}\PY{o}{*}\PY{o}{*}\PY{l+m+mi}{2}\PY{o}{+}\PY{n}{w\PYZus{}0}\PY{o}{*}\PY{o}{*}\PY{l+m+mi}{2}\PY{p}{)}\PY{p}{)}
        \PY{n}{w\PYZus{}0}\PY{o}{=}\PY{l+m+mi}{2}\PY{o}{*}\PY{n}{pi}\PY{o}{*}\PY{l+m+mf}{1e6}
        \PY{n}{cos\PYZus{}s}\PY{o}{=}\PY{n}{s}\PY{o}{/}\PY{p}{(}\PY{p}{(}\PY{n}{s}\PY{o}{*}\PY{o}{*}\PY{l+m+mi}{2}\PY{o}{+}\PY{n}{w\PYZus{}0}\PY{o}{*}\PY{o}{*}\PY{l+m+mi}{2}\PY{p}{)}\PY{p}{)}
        \PY{n}{V\PYZus{}i}\PY{o}{=}\PY{n}{cos\PYZus{}s}\PY{o}{+}\PY{n}{sin\PYZus{}s}
        \PY{n}{A}\PY{p}{,}\PY{n}{b}\PY{p}{,}\PY{n}{V}\PY{o}{=}\PY{n}{lowpass}\PY{p}{(}\PY{l+m+mi}{10000}\PY{p}{,}\PY{l+m+mi}{10000}\PY{p}{,}\PY{l+m+mf}{1e\PYZhy{}9}\PY{p}{,}\PY{l+m+mf}{1e\PYZhy{}9}\PY{p}{,}\PY{l+m+mf}{1.586}\PY{p}{,}\PY{l+m+mi}{1000}\PY{p}{,}\PY{n}{V\PYZus{}i}\PY{p}{)}
        \PY{n}{V0}\PY{o}{=}\PY{n}{V}\PY{p}{[}\PY{l+m+mi}{3}\PY{p}{]}
        \PY{n}{plot}\PY{p}{(}\PY{n}{V0}\PY{p}{,}\PY{l+m+mi}{8}\PY{p}{,}\PY{l+s+s1}{\PYZsq{}}\PY{l+s+s1}{Laplace transform for the given input}\PY{l+s+s1}{\PYZsq{}}\PY{p}{)}
\end{Verbatim}


    \begin{center}
    \adjustimage{max size={0.9\linewidth}{0.9\paperheight}}{output_16_0.png}
    \end{center}
    { \hspace*{\fill} \\}
    
    \begin{Verbatim}[commandchars=\\\{\}]
{\color{incolor}In [{\color{incolor}10}]:} \PY{n}{A}\PY{p}{,}\PY{n}{b}\PY{p}{,}\PY{n}{V}\PY{o}{=}\PY{n}{lowpass}\PY{p}{(}\PY{l+m+mi}{10000}\PY{p}{,}\PY{l+m+mi}{10000}\PY{p}{,}\PY{l+m+mf}{1e\PYZhy{}9}\PY{p}{,}\PY{l+m+mf}{1e\PYZhy{}9}\PY{p}{,}\PY{l+m+mf}{1.586}\PY{p}{,}\PY{l+m+mi}{1000}\PY{p}{,}\PY{l+m+mi}{1}\PY{p}{)}
         \PY{n}{V0}\PY{o}{=}\PY{n}{V}\PY{p}{[}\PY{l+m+mi}{3}\PY{p}{]}
         \PY{n}{t}\PY{o}{=}\PY{n}{np}\PY{o}{.}\PY{n}{arange}\PY{p}{(}\PY{l+m+mi}{0}\PY{p}{,}\PY{l+m+mi}{2}\PY{o}{*}\PY{l+m+mf}{1e\PYZhy{}3}\PY{p}{,}\PY{l+m+mf}{1e\PYZhy{}7}\PY{p}{)}
         \PY{n}{u} \PY{o}{=} \PY{n}{np}\PY{o}{.}\PY{n}{ones}\PY{p}{(}\PY{n}{t}\PY{o}{.}\PY{n}{size}\PY{p}{)}
         \PY{n}{V\PYZus{}i}\PY{o}{=}\PY{p}{(}\PY{n}{np}\PY{o}{.}\PY{n}{sin}\PY{p}{(}\PY{l+m+mi}{2000}\PY{o}{*}\PY{n}{np}\PY{o}{.}\PY{n}{pi}\PY{o}{*}\PY{n}{t}\PY{p}{)}\PY{o}{+}\PY{n}{np}\PY{o}{.}\PY{n}{cos}\PY{p}{(}\PY{l+m+mi}{2}\PY{o}{*}\PY{l+m+mf}{1e6}\PY{o}{*}\PY{n}{np}\PY{o}{.}\PY{n}{pi}\PY{o}{*}\PY{n}{t}\PY{p}{)}\PY{p}{)}\PY{o}{*}\PY{n}{u}
         \PY{n}{plot\PYZus{}graph}\PY{p}{(}\PY{n}{V0}\PY{p}{,}\PY{n}{t}\PY{p}{,}\PY{n}{V\PYZus{}i}\PY{p}{,}\PY{l+s+s1}{\PYZsq{}}\PY{l+s+s1}{\PYZdl{}(sin(2x10\PYZca{}3}\PY{l+s+s1}{\PYZbs{}}\PY{l+s+s1}{pi t)+cos(2x10\PYZca{}6}\PY{l+s+s1}{\PYZbs{}}\PY{l+s+s1}{pi t))u\PYZus{}0}\PY{l+s+si}{\PYZob{}t\PYZcb{}}\PY{l+s+s1}{\PYZdl{} }\PY{l+s+se}{\PYZbs{}n}\PY{l+s+s1}{ (longterm)}\PY{l+s+s1}{\PYZsq{}}\PY{p}{)}
\end{Verbatim}


    \begin{center}
    \adjustimage{max size={0.9\linewidth}{0.9\paperheight}}{output_17_0.png}
    \end{center}
    { \hspace*{\fill} \\}
    
    The small ripples in th transient correspond to the frequency component
with \(\omega=10^6\)

    \begin{Verbatim}[commandchars=\\\{\}]
{\color{incolor}In [{\color{incolor}11}]:} \PY{n}{t}\PY{o}{=}\PY{n}{np}\PY{o}{.}\PY{n}{arange}\PY{p}{(}\PY{l+m+mi}{0}\PY{p}{,}\PY{l+m+mf}{1e\PYZhy{}5}\PY{p}{,}\PY{l+m+mf}{1e\PYZhy{}8}\PY{p}{)}
         \PY{n}{u} \PY{o}{=} \PY{n}{np}\PY{o}{.}\PY{n}{ones}\PY{p}{(}\PY{n}{t}\PY{o}{.}\PY{n}{size}\PY{p}{)}
         \PY{n}{V\PYZus{}i}\PY{o}{=}\PY{p}{(}\PY{n}{np}\PY{o}{.}\PY{n}{sin}\PY{p}{(}\PY{l+m+mi}{2000}\PY{o}{*}\PY{n}{np}\PY{o}{.}\PY{n}{pi}\PY{o}{*}\PY{n}{t}\PY{p}{)}\PY{o}{+}\PY{n}{np}\PY{o}{.}\PY{n}{cos}\PY{p}{(}\PY{l+m+mi}{2}\PY{o}{*}\PY{l+m+mf}{1e6}\PY{o}{*}\PY{n}{np}\PY{o}{.}\PY{n}{pi}\PY{o}{*}\PY{n}{t}\PY{p}{)}\PY{p}{)}\PY{o}{*}\PY{n}{u}
         \PY{n}{plot\PYZus{}graph}\PY{p}{(}\PY{n}{V0}\PY{p}{,}\PY{n}{t}\PY{p}{,}\PY{n}{V\PYZus{}i}\PY{p}{,}\PY{l+s+s1}{\PYZsq{}}\PY{l+s+s1}{\PYZdl{}(sin(2x10\PYZca{}3}\PY{l+s+s1}{\PYZbs{}}\PY{l+s+s1}{pi t)+cos(2x10\PYZca{}6}\PY{l+s+s1}{\PYZbs{}}\PY{l+s+s1}{pi t))u\PYZus{}0}\PY{l+s+si}{\PYZob{}t\PYZcb{}}\PY{l+s+s1}{\PYZdl{} }\PY{l+s+se}{\PYZbs{}n}\PY{l+s+s1}{ (transient)}\PY{l+s+s1}{\PYZsq{}}\PY{p}{)}
\end{Verbatim}


    \begin{center}
    \adjustimage{max size={0.9\linewidth}{0.9\paperheight}}{output_19_0.png}
    \end{center}
    { \hspace*{\fill} \\}
    
    \hypertarget{highpass-filter}{%
\section{Highpass Filter}\label{highpass-filter}}

In the circuit given below we write the nodal equations and form a
matrix.

    \begin{Verbatim}[commandchars=\\\{\}]
{\color{incolor}In [{\color{incolor}12}]:} \PY{k+kn}{from} \PY{n+nn}{IPython}\PY{n+nn}{.}\PY{n+nn}{display} \PY{k}{import} \PY{n}{Image}
         \PY{n}{Image}\PY{p}{(}\PY{n}{filename}\PY{o}{=}\PY{l+s+s1}{\PYZsq{}}\PY{l+s+s1}{/Users/siddharthnayak/Desktop/highpass.png}\PY{l+s+s1}{\PYZsq{}}\PY{p}{)}
\end{Verbatim}

\texttt{\color{outcolor}Out[{\color{outcolor}12}]:}
    
    \begin{center}
    \adjustimage{max size={0.9\linewidth}{0.9\paperheight}}{output_21_0.png}
    \end{center}
    { \hspace*{\fill} \\}
    

The circuit equations are: \\
    \(V_m=\frac{V_m}{G_1}\)
\(V_p=V1\frac{1}{1+j\omega R_2.C_2}\) \\
\(V_0=G_2(V_p-V_m)\)\\
\(\frac{V_i-V_1}{R1}+\frac{V_p-V_1}{R2}+\frac{(V_0-V_1)}{j\omega C1}=0\)\\
Therefore the matrix becomes:\\\\
\(\begin{pmatrix}  0 & 0 & 1 & -\frac{1}{G_1} \\  -\frac{1}{1+sC_2R_2} & 1 & 0 & 0 \\  0 & -G_2 & G_2 & 1\\  -sC_1-sC_2-\frac{1}{R_1} & sC_2 & 0 & \frac{1}{R_1}  \end{pmatrix}  .  \begin{pmatrix}  V_1\\  V_p\\  V_m\\  V_0  \end{pmatrix} = \begin{pmatrix}  0\\  0\\  0\\  V_i(s)sC_1  \end{pmatrix}\)\\
\\ * Assuming \(G_2\) is the gain of the opamp.
\\ * Image courtesy:The Art of
Electronics by Paul Horowitz and Winfield Hill

    \begin{Verbatim}[commandchars=\\\{\}]
{\color{incolor}In [{\color{incolor}13}]:} \PY{k}{def} \PY{n+nf}{highpass}\PY{p}{(}\PY{n}{R1}\PY{p}{,}\PY{n}{R2}\PY{p}{,}\PY{n}{C1}\PY{p}{,}\PY{n}{C2}\PY{p}{,}\PY{n}{G1}\PY{p}{,}\PY{n}{G2}\PY{p}{,}\PY{n}{Vi}\PY{p}{)}\PY{p}{:}
             \PY{n}{s}\PY{o}{=}\PY{n}{symbols}\PY{p}{(}\PY{l+s+s1}{\PYZsq{}}\PY{l+s+s1}{s}\PY{l+s+s1}{\PYZsq{}}\PY{p}{)}
             \PY{n}{A} \PY{o}{=} \PY{n}{Matrix}\PY{p}{(}\PY{p}{[}\PY{p}{[}\PY{l+m+mi}{0}\PY{p}{,} \PY{l+m+mi}{0}\PY{p}{,} \PY{l+m+mi}{1}\PY{p}{,} \PY{o}{\PYZhy{}}\PY{l+m+mi}{1}\PY{o}{/}\PY{n}{G1}\PY{p}{]}\PY{p}{,}\PY{p}{[}\PY{l+m+mi}{0}\PY{p}{,} \PY{o}{\PYZhy{}}\PY{n}{G2}\PY{p}{,} \PY{n}{G2}\PY{p}{,} \PY{o}{\PYZhy{}}\PY{l+m+mi}{1}\PY{p}{]}\PY{p}{,}\PY{p}{[}\PY{n}{s}\PY{o}{*}\PY{n}{C2}\PY{o}{*}\PY{n}{R2}\PY{p}{,} \PY{o}{\PYZhy{}}\PY{p}{(}\PY{n}{s}\PY{o}{*}\PY{n}{C2}\PY{o}{*}\PY{n}{R2}\PY{o}{+}\PY{l+m+mi}{1}\PY{p}{)}\PY{p}{,} \PY{l+m+mi}{0}\PY{p}{,} \PY{l+m+mi}{0}\PY{p}{]}\PY{p}{,}\PY{p}{[}\PY{n}{s}\PY{o}{*}\PY{n}{C1}\PY{o}{+}\PY{n}{s}\PY{o}{*}\PY{n}{C2}\PY{o}{+}\PY{l+m+mi}{1}\PY{o}{/}\PY{n}{R1}\PY{p}{,} \PY{o}{\PYZhy{}}\PY{p}{(}\PY{n}{s}\PY{o}{*}\PY{n}{C2}\PY{p}{)}\PY{p}{,} \PY{l+m+mi}{0}\PY{p}{,} \PY{o}{\PYZhy{}}\PY{l+m+mi}{1}\PY{o}{/}\PY{n}{R1}\PY{p}{]}\PY{p}{]}\PY{p}{)}
             \PY{n}{b} \PY{o}{=} \PY{n}{Matrix}\PY{p}{(}\PY{p}{[}\PY{l+m+mi}{0}\PY{p}{,} \PY{l+m+mi}{0}\PY{p}{,} \PY{l+m+mi}{0}\PY{p}{,} \PY{n}{Vi}\PY{o}{*}\PY{n}{s}\PY{o}{*}\PY{n}{C1}\PY{p}{]}\PY{p}{)}
             \PY{n}{V}\PY{o}{=}\PY{n}{A}\PY{o}{.}\PY{n}{inv}\PY{p}{(}\PY{p}{)}\PY{o}{*}\PY{n}{b}
             \PY{k}{return} \PY{n}{A}\PY{p}{,}\PY{n}{b}\PY{p}{,}\PY{n}{V}\PY{p}{[}\PY{l+m+mi}{3}\PY{p}{]}
\end{Verbatim}


    \hypertarget{obtaining-the-transfer-function-and-extracting-the-coefficients-from-the-sympy-polynomial.}{%
\section{Obtaining the transfer function and extracting the coefficients
from the Sympy
polynomial.}\label{obtaining-the-transfer-function-and-extracting-the-coefficients-from-the-sympy-polynomial.}}

    \begin{Verbatim}[commandchars=\\\{\}]
{\color{incolor}In [{\color{incolor}14}]:} \PY{n}{s} \PY{o}{=} \PY{n}{symbols}\PY{p}{(}\PY{l+s+s1}{\PYZsq{}}\PY{l+s+s1}{s}\PY{l+s+s1}{\PYZsq{}}\PY{p}{)}
         \PY{n}{t} \PY{o}{=} \PY{n}{np}\PY{o}{.}\PY{n}{linspace}\PY{p}{(}\PY{l+m+mi}{0}\PY{p}{,}\PY{l+m+mf}{1e\PYZhy{}3}\PY{p}{,}\PY{l+m+mi}{1001}\PY{p}{)}
         \PY{n}{A}\PY{p}{,} \PY{n}{b}\PY{p}{,} \PY{n}{Vo} \PY{o}{=} \PY{n}{highpass}\PY{p}{(}\PY{l+m+mi}{10000}\PY{p}{,}\PY{l+m+mi}{10000}\PY{p}{,}\PY{l+m+mf}{1e\PYZhy{}9}\PY{p}{,}\PY{l+m+mf}{1e\PYZhy{}9}\PY{p}{,}\PY{l+m+mf}{1.586}\PY{p}{,}\PY{l+m+mi}{1000}\PY{p}{,}\PY{l+m+mi}{1}\PY{p}{)} \PY{c+c1}{\PYZsh{} Vo consists of the transfer function of the system \PYZsh{}}
         
         \PY{n}{num}\PY{p}{,} \PY{n}{den} \PY{o}{=} \PY{n}{simplify}\PY{p}{(}\PY{n}{Vo}\PY{p}{)}\PY{o}{.}\PY{n}{as\PYZus{}numer\PYZus{}denom}\PY{p}{(}\PY{p}{)}
         \PY{n}{p\PYZus{}num\PYZus{}den} \PY{o}{=} \PY{n}{poly}\PY{p}{(}\PY{n}{num}\PY{p}{,}\PY{n}{s}\PY{p}{)}\PY{p}{,} \PY{n}{poly}\PY{p}{(}\PY{n}{den}\PY{p}{,}\PY{n}{s}\PY{p}{)} \PY{c+c1}{\PYZsh{} Polynomials \PYZsh{}}
         \PY{n}{c\PYZus{}num\PYZus{}den} \PY{o}{=} \PY{p}{[}\PY{n}{expand}\PY{p}{(}\PY{n}{p}\PY{p}{)}\PY{o}{.}\PY{n}{all\PYZus{}coeffs}\PY{p}{(}\PY{p}{)} \PY{k}{for} \PY{n}{p} \PY{o+ow}{in} \PY{n}{p\PYZus{}num\PYZus{}den}\PY{p}{]} \PY{c+c1}{\PYZsh{} Coefficients \PYZsh{}}
         \PY{n}{l\PYZus{}num}\PY{p}{,} \PY{n}{l\PYZus{}den} \PY{o}{=} \PY{p}{[}\PY{n}{lambdify}\PY{p}{(}\PY{p}{(}\PY{p}{)}\PY{p}{,}\PY{n}{c}\PY{p}{)}\PY{p}{(}\PY{p}{)} \PY{k}{for} \PY{n}{c} \PY{o+ow}{in} \PY{n}{c\PYZus{}num\PYZus{}den}\PY{p}{]} \PY{c+c1}{\PYZsh{} Convert to floats \PYZsh{}}
         \PY{n}{H} \PY{o}{=} \PY{n}{sp}\PY{o}{.}\PY{n}{lti}\PY{p}{(}\PY{n}{l\PYZus{}num}\PY{p}{,} \PY{n}{l\PYZus{}den}\PY{p}{)} \PY{c+c1}{\PYZsh{} LTI transfer function of the circuit \PYZsh{}}
         
         \PY{n}{w} \PY{o}{=} \PY{n}{np}\PY{o}{.}\PY{n}{logspace}\PY{p}{(}\PY{l+m+mi}{0}\PY{p}{,} \PY{l+m+mi}{16}\PY{p}{,}\PY{l+m+mi}{1001}\PY{p}{)}
         \PY{n}{ss} \PY{o}{=} \PY{l+m+mi}{1}\PY{n}{j}\PY{o}{*}\PY{n}{w} \PY{c+c1}{\PYZsh{} Obtaining \PYZlt{}jw\PYZgt{} values to replace \PYZlt{}s\PYZgt{} in the laplace transform \PYZsh{}}
         \PY{n}{f} \PY{o}{=} \PY{n}{lambdify}\PY{p}{(}\PY{n}{s}\PY{p}{,} \PY{n}{Vo}\PY{p}{,} \PY{l+s+s1}{\PYZsq{}}\PY{l+s+s1}{numpy}\PY{l+s+s1}{\PYZsq{}}\PY{p}{)}
         \PY{n}{Vo} \PY{o}{=} \PY{n}{f}\PY{p}{(}\PY{n}{ss}\PY{p}{)}
\end{Verbatim}


    \hypertarget{plot-the-magnitude-response}{%
\section{Plot the magnitude
response}\label{plot-the-magnitude-response}}

    This Highpass filter has a cutoff at \(\omega=10^5\). i.e.~till
\(\omega=10^5\) the frequency components will be attenuated and after
that we have a gain of 1 for higher frequencies.

    \begin{Verbatim}[commandchars=\\\{\}]
{\color{incolor}In [{\color{incolor}15}]:} \PY{n}{plt}\PY{o}{.}\PY{n}{loglog}\PY{p}{(}\PY{n}{w}\PY{p}{,} \PY{n}{np}\PY{o}{.}\PY{n}{abs}\PY{p}{(}\PY{n}{Vo}\PY{p}{)}\PY{p}{,}\PY{n}{lw}\PY{o}{=}\PY{l+m+mi}{2}\PY{p}{)}
         \PY{n}{plt}\PY{o}{.}\PY{n}{xlabel}\PY{p}{(}\PY{l+s+s2}{\PYZdq{}}\PY{l+s+s2}{\PYZdl{}}\PY{l+s+s2}{\PYZbs{}}\PY{l+s+s2}{omega\PYZdl{}}\PY{l+s+s2}{\PYZdq{}}\PY{p}{)}
         \PY{n}{plt}\PY{o}{.}\PY{n}{ylabel}\PY{p}{(}\PY{l+s+s2}{\PYZdq{}}\PY{l+s+s2}{\PYZdl{}H(j}\PY{l+s+s2}{\PYZbs{}}\PY{l+s+s2}{omega)\PYZdl{}}\PY{l+s+s2}{\PYZdq{}}\PY{p}{)}
         \PY{n}{plt}\PY{o}{.}\PY{n}{grid}\PY{p}{(}\PY{k+kc}{True}\PY{p}{)}
         \PY{n}{plt}\PY{o}{.}\PY{n}{title}\PY{p}{(}\PY{l+s+s2}{\PYZdq{}}\PY{l+s+s2}{Magnitude Response of the Transfer Function}\PY{l+s+s2}{\PYZdq{}}\PY{p}{)}
         \PY{n}{plt}\PY{o}{.}\PY{n}{show}\PY{p}{(}\PY{p}{)}
\end{Verbatim}


    \begin{center}
    \adjustimage{max size={0.9\linewidth}{0.9\paperheight}}{output_28_0.png}
    \end{center}
    { \hspace*{\fill} \\}
    
    \hypertarget{obtaining-the-step-response-of-the-circuit}{%
\section{Obtaining the step response of the
circuit}\label{obtaining-the-step-response-of-the-circuit}}

The unit response has some initial values which decreases to zero with a
little overshoot.This is because the quality factor of the second order
system is greater than \(\frac{1}{\sqrt{2}}\). Also in a unit step all
the frequency components are present at t=0.At this point the low
frequency components are attenuates and the high frequency components
are amplified. After some time the frequency component is zero thus
output is zero.

    \begin{Verbatim}[commandchars=\\\{\}]
{\color{incolor}In [{\color{incolor}16}]:} \PY{n}{Vi\PYZus{}step} \PY{o}{=} \PY{l+m+mi}{1}\PY{o}{/}\PY{n}{s}
         \PY{n}{step} \PY{o}{=} \PY{n}{np}\PY{o}{.}\PY{n}{heaviside}\PY{p}{(}\PY{n}{t}\PY{p}{,} \PY{l+m+mi}{1}\PY{p}{)} \PY{c+c1}{\PYZsh{} Step input \PYZsh{}}
         \PY{n}{time}\PY{p}{,} \PY{n}{step\PYZus{}response}\PY{p}{,} \PY{n}{svec} \PY{o}{=} \PY{n}{sp}\PY{o}{.}\PY{n}{lsim}\PY{p}{(}\PY{n}{H}\PY{p}{,} \PY{n}{step}\PY{p}{,} \PY{n}{t}\PY{p}{)}
         \PY{n}{A}\PY{p}{,} \PY{n}{b}\PY{p}{,} \PY{n}{Vo} \PY{o}{=} \PY{n}{highpass}\PY{p}{(}\PY{l+m+mi}{10000}\PY{p}{,}\PY{l+m+mi}{10000}\PY{p}{,}\PY{l+m+mf}{1e\PYZhy{}9}\PY{p}{,}\PY{l+m+mf}{1e\PYZhy{}9}\PY{p}{,}\PY{l+m+mf}{1.586}\PY{p}{,}\PY{l+m+mi}{1000}\PY{p}{,}\PY{n}{Vi\PYZus{}step}\PY{p}{)}
         \PY{n}{f} \PY{o}{=} \PY{n}{lambdify}\PY{p}{(}\PY{n}{s}\PY{p}{,} \PY{n}{Vo}\PY{p}{,} \PY{l+s+s1}{\PYZsq{}}\PY{l+s+s1}{numpy}\PY{l+s+s1}{\PYZsq{}}\PY{p}{)}
         \PY{n}{Vo} \PY{o}{=} \PY{n}{f}\PY{p}{(}\PY{n}{ss}\PY{p}{)}
         
         
         \PY{n}{plt}\PY{o}{.}\PY{n}{loglog}\PY{p}{(}\PY{n}{w}\PY{p}{,} \PY{n}{np}\PY{o}{.}\PY{n}{abs}\PY{p}{(}\PY{n}{Vo}\PY{p}{)}\PY{p}{,} \PY{n}{lw}\PY{o}{=}\PY{l+m+mi}{2}\PY{p}{)}
         \PY{n}{plt}\PY{o}{.}\PY{n}{xlabel}\PY{p}{(}\PY{l+s+s2}{\PYZdq{}}\PY{l+s+s2}{\PYZdl{}}\PY{l+s+s2}{\PYZbs{}}\PY{l+s+s2}{omega\PYZdl{}}\PY{l+s+s2}{\PYZdq{}}\PY{p}{)}
         \PY{n}{plt}\PY{o}{.}\PY{n}{ylabel}\PY{p}{(}\PY{l+s+s2}{\PYZdq{}}\PY{l+s+s2}{\PYZdl{}V\PYZus{}o(j}\PY{l+s+s2}{\PYZbs{}}\PY{l+s+s2}{omega)\PYZdl{}}\PY{l+s+s2}{\PYZdq{}}\PY{p}{)}
         \PY{n}{plt}\PY{o}{.}\PY{n}{grid}\PY{p}{(}\PY{k+kc}{True}\PY{p}{)}
         \PY{n}{plt}\PY{o}{.}\PY{n}{title}\PY{p}{(}\PY{l+s+s2}{\PYZdq{}}\PY{l+s+s2}{Laplace Transform of step response}\PY{l+s+s2}{\PYZdq{}}\PY{p}{)}
         \PY{n}{plt}\PY{o}{.}\PY{n}{show}\PY{p}{(}\PY{p}{)}
         
         
         \PY{n}{plt}\PY{o}{.}\PY{n}{plot}\PY{p}{(}\PY{n}{t}\PY{p}{,} \PY{n}{step\PYZus{}response}\PY{p}{)}
         \PY{n}{plt}\PY{o}{.}\PY{n}{xlabel}\PY{p}{(}\PY{l+s+s2}{\PYZdq{}}\PY{l+s+s2}{Time}\PY{l+s+s2}{\PYZdq{}}\PY{p}{)}
         \PY{n}{plt}\PY{o}{.}\PY{n}{ylabel}\PY{p}{(}\PY{l+s+s2}{\PYZdq{}}\PY{l+s+s2}{\PYZdl{}V\PYZus{}o(t)\PYZdl{}}\PY{l+s+s2}{\PYZdq{}}\PY{p}{)}
         \PY{n}{plt}\PY{o}{.}\PY{n}{title}\PY{p}{(}\PY{l+s+s2}{\PYZdq{}}\PY{l+s+s2}{Step Response}\PY{l+s+s2}{\PYZdq{}}\PY{p}{)}
         \PY{n}{plt}\PY{o}{.}\PY{n}{grid}\PY{p}{(}\PY{k+kc}{True}\PY{p}{)}
         \PY{n}{plt}\PY{o}{.}\PY{n}{show}\PY{p}{(}\PY{p}{)}
\end{Verbatim}


    \begin{center}
    \adjustimage{max size={0.9\linewidth}{0.9\paperheight}}{output_30_0.png}
    \end{center}
    { \hspace*{\fill} \\}
    
    \begin{center}
    \adjustimage{max size={0.9\linewidth}{0.9\paperheight}}{output_30_1.png}
    \end{center}
    { \hspace*{\fill} \\}
    
    \hypertarget{decaying-sinusoidal-inputs-to-the-high-pass-filter}{%
\section{Decaying sinusoidal inputs to the high pass
filter}\label{decaying-sinusoidal-inputs-to-the-high-pass-filter}}

From the graphs it is clearly evident that the filter is responding only
to high frequency components.

    \begin{Verbatim}[commandchars=\\\{\}]
{\color{incolor}In [{\color{incolor}17}]:} \PY{n}{decay\PYZus{}sin\PYZus{}1}\PY{o}{=}\PY{n}{np}\PY{o}{.}\PY{n}{sin}\PY{p}{(}\PY{l+m+mf}{1e6}\PY{o}{*}\PY{n}{t}\PY{p}{)}\PY{o}{*}\PY{n}{np}\PY{o}{.}\PY{n}{exp}\PY{p}{(}\PY{o}{\PYZhy{}}\PY{l+m+mi}{5000}\PY{o}{*}\PY{n}{t}\PY{p}{)}\PY{o}{*}\PY{n}{np}\PY{o}{.}\PY{n}{heaviside}\PY{p}{(}\PY{n}{t}\PY{p}{,} \PY{l+m+mi}{1}\PY{p}{)} \PY{c+c1}{\PYZsh{} Decaying sinusoid input \PYZsh{}}
         \PY{n}{decay\PYZus{}sin\PYZus{}2}\PY{o}{=}\PY{n}{np}\PY{o}{.}\PY{n}{sin}\PY{p}{(}\PY{l+m+mi}{10}\PY{o}{*}\PY{n}{t}\PY{p}{)}\PY{o}{*}\PY{n}{np}\PY{o}{.}\PY{n}{exp}\PY{p}{(}\PY{o}{\PYZhy{}}\PY{l+m+mi}{5000}\PY{o}{*}\PY{n}{t}\PY{p}{)}\PY{o}{*}\PY{n}{np}\PY{o}{.}\PY{n}{heaviside}\PY{p}{(}\PY{n}{t}\PY{p}{,} \PY{l+m+mi}{1}\PY{p}{)} 
         \PY{n}{t1}\PY{o}{=}\PY{n}{np}\PY{o}{.}\PY{n}{arange}\PY{p}{(}\PY{l+m+mi}{0}\PY{p}{,}\PY{l+m+mf}{1e\PYZhy{}4}\PY{p}{,}\PY{l+m+mf}{1e\PYZhy{}7}\PY{p}{)}
         \PY{n}{decay\PYZus{}sin\PYZus{}3}\PY{o}{=}\PY{n}{np}\PY{o}{.}\PY{n}{sin}\PY{p}{(}\PY{l+m+mf}{1e6}\PY{o}{*}\PY{n}{t1}\PY{p}{)}\PY{o}{*}\PY{n}{np}\PY{o}{.}\PY{n}{heaviside}\PY{p}{(}\PY{n}{t1}\PY{p}{,} \PY{l+m+mi}{1}\PY{p}{)}
         \PY{n}{t1}\PY{o}{=}\PY{n}{np}\PY{o}{.}\PY{n}{arange}\PY{p}{(}\PY{l+m+mi}{0}\PY{p}{,}\PY{l+m+mf}{1e\PYZhy{}4}\PY{p}{,}\PY{l+m+mf}{1e\PYZhy{}7}\PY{p}{)}
         \PY{n}{decay\PYZus{}sin\PYZus{}4}\PY{o}{=}\PY{n}{np}\PY{o}{.}\PY{n}{sin}\PY{p}{(}\PY{l+m+mi}{10}\PY{o}{*}\PY{n}{t1}\PY{p}{)}\PY{o}{*}\PY{n}{np}\PY{o}{.}\PY{n}{heaviside}\PY{p}{(}\PY{n}{t1}\PY{p}{,} \PY{l+m+mi}{1}\PY{p}{)}
         
         
         \PY{n}{time}\PY{p}{,}\PY{n}{sin\PYZus{}response\PYZus{}1}\PY{p}{,}\PY{n}{svec}\PY{o}{=}\PY{n}{sp}\PY{o}{.}\PY{n}{lsim}\PY{p}{(}\PY{n}{H}\PY{p}{,}\PY{n}{decay\PYZus{}sin\PYZus{}1}\PY{p}{,}\PY{n}{t}\PY{p}{)} \PY{c+c1}{\PYZsh{} Simulate the output using lsim \PYZsh{}}
         \PY{n}{time}\PY{p}{,}\PY{n}{sin\PYZus{}response\PYZus{}2}\PY{p}{,}\PY{n}{svec}\PY{o}{=}\PY{n}{sp}\PY{o}{.}\PY{n}{lsim}\PY{p}{(}\PY{n}{H}\PY{p}{,}\PY{n}{decay\PYZus{}sin\PYZus{}2}\PY{p}{,}\PY{n}{t}\PY{p}{)} 
         \PY{n}{time1}\PY{p}{,}\PY{n}{sin\PYZus{}response\PYZus{}3}\PY{p}{,}\PY{n}{svec}\PY{o}{=}\PY{n}{sp}\PY{o}{.}\PY{n}{lsim}\PY{p}{(}\PY{n}{H}\PY{p}{,}\PY{n}{decay\PYZus{}sin\PYZus{}3}\PY{p}{,}\PY{n}{t1}\PY{p}{)} 
         \PY{n}{time1}\PY{p}{,}\PY{n}{sin\PYZus{}response\PYZus{}4}\PY{p}{,}\PY{n}{svec}\PY{o}{=}\PY{n}{sp}\PY{o}{.}\PY{n}{lsim}\PY{p}{(}\PY{n}{H}\PY{p}{,}\PY{n}{decay\PYZus{}sin\PYZus{}4}\PY{p}{,}\PY{n}{t1}\PY{p}{)} 
         
         \PY{n}{plt}\PY{o}{.}\PY{n}{plot}\PY{p}{(}\PY{n}{t}\PY{p}{,} \PY{n}{sin\PYZus{}response\PYZus{}1}\PY{p}{)}
         \PY{n}{plt}\PY{o}{.}\PY{n}{xlabel}\PY{p}{(}\PY{l+s+s2}{\PYZdq{}}\PY{l+s+s2}{Time(seconds)}\PY{l+s+s2}{\PYZdq{}}\PY{p}{)}
         \PY{n}{plt}\PY{o}{.}\PY{n}{ylabel}\PY{p}{(}\PY{l+s+s2}{\PYZdq{}}\PY{l+s+s2}{\PYZdl{}V\PYZus{}o(t)\PYZdl{}}\PY{l+s+s2}{\PYZdq{}}\PY{p}{)}
         \PY{n}{plt}\PY{o}{.}\PY{n}{title}\PY{p}{(}\PY{l+s+s2}{\PYZdq{}}\PY{l+s+s2}{Response to a decaying sine input : \PYZdl{}e\PYZca{}}\PY{l+s+s2}{\PYZob{}}\PY{l+s+s2}{\PYZhy{}5000t\PYZcb{}sin(10\PYZca{}6t)u\PYZus{}0(t)\PYZdl{}}\PY{l+s+s2}{\PYZdq{}}\PY{p}{)}
         \PY{n}{plt}\PY{o}{.}\PY{n}{grid}\PY{p}{(}\PY{k+kc}{True}\PY{p}{)}
         \PY{n}{plt}\PY{o}{.}\PY{n}{show}\PY{p}{(}\PY{p}{)}
         
         \PY{n}{plt}\PY{o}{.}\PY{n}{plot}\PY{p}{(}\PY{n}{t}\PY{p}{,} \PY{n}{sin\PYZus{}response\PYZus{}2}\PY{p}{)}
         \PY{n}{plt}\PY{o}{.}\PY{n}{xlabel}\PY{p}{(}\PY{l+s+s2}{\PYZdq{}}\PY{l+s+s2}{Time(seconds)}\PY{l+s+s2}{\PYZdq{}}\PY{p}{)}
         \PY{n}{plt}\PY{o}{.}\PY{n}{ylabel}\PY{p}{(}\PY{l+s+s2}{\PYZdq{}}\PY{l+s+s2}{\PYZdl{}V\PYZus{}o(t)\PYZdl{}}\PY{l+s+s2}{\PYZdq{}}\PY{p}{)}
         \PY{n}{plt}\PY{o}{.}\PY{n}{title}\PY{p}{(}\PY{l+s+s2}{\PYZdq{}}\PY{l+s+s2}{Response to a decaying sine input : \PYZdl{}e\PYZca{}}\PY{l+s+s2}{\PYZob{}}\PY{l+s+s2}{\PYZhy{}5000t\PYZcb{}sin(10t)u\PYZus{}0(t)\PYZdl{}}\PY{l+s+s2}{\PYZdq{}}\PY{p}{)}
         \PY{n}{plt}\PY{o}{.}\PY{n}{grid}\PY{p}{(}\PY{k+kc}{True}\PY{p}{)}
         \PY{n}{plt}\PY{o}{.}\PY{n}{show}\PY{p}{(}\PY{p}{)}
         
         \PY{n}{plt}\PY{o}{.}\PY{n}{plot}\PY{p}{(}\PY{n}{t1}\PY{p}{,} \PY{n}{sin\PYZus{}response\PYZus{}3}\PY{p}{)}
         \PY{n}{plt}\PY{o}{.}\PY{n}{xlabel}\PY{p}{(}\PY{l+s+s2}{\PYZdq{}}\PY{l+s+s2}{Time(seconds)}\PY{l+s+s2}{\PYZdq{}}\PY{p}{)}
         \PY{n}{plt}\PY{o}{.}\PY{n}{ylabel}\PY{p}{(}\PY{l+s+s2}{\PYZdq{}}\PY{l+s+s2}{\PYZdl{}V\PYZus{}o(t)\PYZdl{}}\PY{l+s+s2}{\PYZdq{}}\PY{p}{)}
         \PY{n}{plt}\PY{o}{.}\PY{n}{title}\PY{p}{(}\PY{l+s+s2}{\PYZdq{}}\PY{l+s+s2}{Response to a decaying sine input : \PYZdl{}sin(10\PYZca{}6t)u\PYZus{}0(t)\PYZdl{}}\PY{l+s+s2}{\PYZdq{}}\PY{p}{)}
         \PY{n}{plt}\PY{o}{.}\PY{n}{grid}\PY{p}{(}\PY{k+kc}{True}\PY{p}{)}
         \PY{n}{plt}\PY{o}{.}\PY{n}{show}\PY{p}{(}\PY{p}{)}
         
         \PY{n}{plt}\PY{o}{.}\PY{n}{plot}\PY{p}{(}\PY{n}{t1}\PY{p}{,} \PY{n}{sin\PYZus{}response\PYZus{}4}\PY{p}{)}
         \PY{n}{plt}\PY{o}{.}\PY{n}{xlabel}\PY{p}{(}\PY{l+s+s2}{\PYZdq{}}\PY{l+s+s2}{Time(seconds)}\PY{l+s+s2}{\PYZdq{}}\PY{p}{)}
         \PY{n}{plt}\PY{o}{.}\PY{n}{ylabel}\PY{p}{(}\PY{l+s+s2}{\PYZdq{}}\PY{l+s+s2}{\PYZdl{}V\PYZus{}o(t)\PYZdl{}}\PY{l+s+s2}{\PYZdq{}}\PY{p}{)}
         \PY{n}{plt}\PY{o}{.}\PY{n}{title}\PY{p}{(}\PY{l+s+s2}{\PYZdq{}}\PY{l+s+s2}{Response to a decaying sine input : \PYZdl{}sin(10t)u\PYZus{}0(t)\PYZdl{}}\PY{l+s+s2}{\PYZdq{}}\PY{p}{)}
         \PY{n}{plt}\PY{o}{.}\PY{n}{grid}\PY{p}{(}\PY{k+kc}{True}\PY{p}{)}
         \PY{n}{plt}\PY{o}{.}\PY{n}{show}\PY{p}{(}\PY{p}{)}
\end{Verbatim}


    \begin{center}
    \adjustimage{max size={0.9\linewidth}{0.9\paperheight}}{output_32_0.png}
    \end{center}
    { \hspace*{\fill} \\}
    
    \begin{center}
    \adjustimage{max size={0.9\linewidth}{0.9\paperheight}}{output_32_1.png}
    \end{center}
    { \hspace*{\fill} \\}
    
    \begin{center}
    \adjustimage{max size={0.9\linewidth}{0.9\paperheight}}{output_32_2.png}
    \end{center}
    { \hspace*{\fill} \\}
    
    \begin{center}
    \adjustimage{max size={0.9\linewidth}{0.9\paperheight}}{output_32_3.png}
    \end{center}
    { \hspace*{\fill} \\}
    
    \hypertarget{conclusions}{%
\section{Conclusions}\label{conclusions}}

Any linear electrical circuit can be simulated using the \texttt{sympy}
and \texttt{scipy.signal} module. \\Two types of circuits namely Highpass
filter and Lowpass Filter were simulated in this assignment.


    % Add a bibliography block to the postdoc
    
    
    
    \end{document}
